% Gemini theme
% https://github.com/anishathalye/gemini

\documentclass[final]{beamer}

% ====================
% Packages
% ====================

\usepackage{kotex}
\usepackage[T1]{fontenc}
\usepackage{lmodern}
\usepackage[orientation=portrait,size=a1,scale=1.2]{beamerposter}
\usetheme{gemini}
\usecolortheme{gemini}
\usepackage{graphicx}
\usepackage{booktabs}
\usepackage{tikz}
\usepackage{pgfplots}
\pgfplotsset{compat=1.14}
\usepackage{anyfontsize}

\usepackage{empheq}
\usepackage[many]{tcolorbox}

%%% Math settings
\usepackage{amssymb,amsmath,mathtools}
\usepackage[math-style=TeX,bold-style=TeX]{unicode-math}
\setlength{\fboxsep}{10pt}
\setlength{\fboxrule}{0.4pt}

%%% Font settings
\setmainfont{Libertinus Serif}
\setmathfont{Libertinus Math} % Before set*hangulfont
\setmainhangulfont{Noto Serif CJK KR}
\setsanshangulfont[BoldFont={* Bold}]{KoPubWorldDotum}

%%% PL constructs
\usepackage{ebproof}
\ebproofset{left label template=\textsc{[\inserttext]}}
\ebproofset{center=false}

%%% Custom commands
\newcommand*{\vbar}{|}
\newcommand*{\finto}{\xrightarrow{\text{\textrm{fin}}}}
\newcommand*{\istype}{\mathrel{⩴}}
\newcommand*{\ortype}{\mathrel{|}}
\newcommand*{\cons}{::}

\def\ovbarw{1.2mu}
\def\ovbarh{1}
\newcommand*{\ovbar}[1]{\mkern \ovbarw\overline{\mkern-\ovbarw{\smash{#1}\scalebox{1}[\ovbarh]{\vphantom{i}}}\mkern-\ovbarw}\mkern \ovbarw}
\newcommand*{\A}[1]{{#1}^{\#}}
\newcommand*{\Expr}{\text{Expr}}
\newcommand*{\ExprVar}{\text{Var}}
\newcommand*{\Module}{\text{Module}}
\newcommand*{\ModVar}{\text{ModVar}}
\newcommand*{\Time}{\mathbb{T}}
\newcommand*{\ATime}{\A{\Time}}
\newcommand*{\Ctx}{\text{Ctx}}
\newcommand*{\Value}{\text{Val}}
\newcommand*{\Mem}{\text{Mem}}
\newcommand*{\Left}{\text{L}}
\newcommand*{\Right}{\text{R}}
\newcommand*{\mem}{m}
\newcommand*{\AMem}{\A{\text{Mem}}}
\newcommand*{\State}{\text{State}}
\newcommand*{\AState}{\A{\text{State}}}
\newcommand*{\Result}{\text{Result}}
\newcommand*{\AResult}{\A{\text{Result}}}
\newcommand*{\Tick}{\text{Tick}}
\newcommand*{\semarrow}{\rightsquigarrow}
\newcommand*{\semlink}{\mathbin{\rotatebox[origin=c]{180}{$\propto$}}}
\newcommand*{\link}[2]{{#1}\semlink{#2}}
\newcommand*{\mt}{\mathsf{empty}}

\newcommand*{\doubleplus}{\ensuremath{\mathbin{+\mkern-3mu+}}}
\newcommand*{\project}{\text{\texttt{:>} }}
\newcommand*{\Exp}{\mathsf{Exp}}
\newcommand*{\Imp}{\mathsf{Imp}}
\newcommand*{\Fin}{\mathsf{Fin}}
\newcommand*{\Link}{\mathsf{Link}}
\newcommand*{\sembracket}[1]{\lBrack{#1}\rBrack}
\newcommand*{\fin}[2]{{#1}\xrightarrow{\text{fin}}{#2}}
\newcommand*{\addr}{\mathsf{addr}}
\newcommand*{\tick}{\mathsf{tick}}
\newcommand*{\modctx}{\mathsf{ctx}}
\newcommand*{\mapinject}[2]{{#2}[{#1}]}
\newcommand*{\inject}[2]{{#2}\langle{#1}\rangle}
\newcommand*{\deletepre}[2]{{#2}\overline{\doubleplus}{#1}}
\newcommand*{\deletemap}[2]{{#1}\overline{[{#2}]}}
\newcommand*{\delete}[2]{{#2}{\langle{#1}\rangle}^{-1}}
\newcommand*{\filter}{\mathsf{filter}}
\newcommand*{\Let}{\mathtt{let}}

% ====================
% Lengths
% ====================

% If you have N columns, choose \sepwidth and \colwidth such that
% (N+1)*\sepwidth + N*\colwidth = \paperwidth
\newlength{\sepwidth}
\newlength{\colwidth}
\setlength{\sepwidth}{0.025\paperwidth}
\setlength{\colwidth}{0.4625\paperwidth}

\newcommand{\separatorcolumn}{\begin{column}{\sepwidth}\end{column}}

% ====================
% Title
% ====================

\title{프로그램 조각별 따로분석의 이론적 틀}

\author{이준협 \and 이광근}

\institute[shortinst]{서울대학교 프로그래밍 연구실 (ROPAS)}

% ====================
% Footer (optional)
% ====================

%\footercontent{
%  \hfill
%  2023년 SIGPL 여름학교 포스터 발표
%  \hfill
%}
% (can be left out to remove footer)

% ====================
% Logo (optional)
% ====================

% use this to include logos on the left and/or right side of the header:
\logoright{\includegraphics[height=5cm]{ropas-symbol.png}}
\logoleft{\includegraphics[height=5cm]{snu-symbol.png}}

% ====================
% Body
% ====================

\begin{document}

\begin{frame}[t]
  \begin{columns}[t]
    \separatorcolumn

    \begin{column}{\colwidth}
      \begin{block}{풀고자 한 문제}
        프로그램 조각이 주어졌을 때, 분석을 해 놓고 기다리자!
        \begin{itemize}
          \item 목표: $\link{e_1}{e_2}$의 결과 어림잡기.
          \item 가정: $e_1$이 무엇을 내보내는지 일부만 알고 있음.
          \item 질문: $e_2$를 미리 분석해놓고, $e_1$을 \textbf{따로} 분석해서 \textbf{합치면} 전체 결과를 \textbf{안전하게} 어림잡을 수 있을까?
        \end{itemize}
        따로분석을 위한 의미구조를 엄밀하게 정의해보자!
      \end{block}

      \begin{block}{모듈이 있는 언어의 정의}
        \heading{겉모습 (Untyped $\lambda$+Modules)}
        \begin{figure}[h!]
          \large
          \begin{tabular}{rrcll}
            Identifiers & $x,M$ & $\in$         & $\ExprVar$                              \\
            Expression  & $e$   & $\rightarrow$ & $x$                & value identifier   \\
                        &       & $\vbar$       & $\lambda x.e$      & function           \\
                        &       & $\vbar$       & $e$ $e$            & application        \\
                        &       & $\vbar$       & $\link{e}{e}$      & linked expression  \\
                        &       & $\vbar$       & $\varepsilon$      & empty module       \\
                        &       & $\vbar$       & $M$                & module identifier  \\
                        &       & $\vbar$       & $\Let$ $x$ $e$ $e$ & binding expression \\
                        &       & $\vbar$       & $\Let$ $M$ $e$ $e$ & binding module     \\
          \end{tabular}
        \end{figure}
        \heading{속내용 ($\Time$, $\tick$에 대해 매개화된)}
        \begin{figure}[h!]
          \centering
          \normalsize
          \begin{tabular}{rrcl}
            Time            & $t$     & $\in$         & $\Time$                                                                                  \\
            Context         & $C$     & $\in$         & $\Ctx(\Time)$                                                                            \\
            Value(Expr)     & $v$     & $\in$         & $\Value(\Time) \triangleq \Expr\times\Ctx(\Time)$                                        \\
            Value(Expr/Mod) & $V$     & $\in$         & $\Value(\Time)\uplus\Ctx(\Time)$                                                         \\
            Memory          & $\mem$  & $\in$         & $\Mem(\Time) \triangleq \fin{\Time}{\Value(\Time)}$                                      \\
            State           & $s$     & $\in$         & $\State(\Time) \triangleq \Ctx(\Time)\times\Mem(\Time)\times\Time$                       \\
            Result          & $r$     & $\in$         & $\Result(\Time) \triangleq (\Value(\Time)\uplus\Ctx(\Time))\times\Mem(\Time)\times\Time$ \\
            Tick            & $\tick$ & $\in$         & $\Tick(\Time)\triangleq(\State(\Time)\times\ExprVar\times\Value(\Time))\rightarrow\Time$ \\
            Context         & $C$     & $\rightarrow$ & []                                                                                       \\
                            &         & $\vbar$       & $(x,t)\cons C$                                                                           \\
                            &         & $\vbar$       & $(M,C)\cons C$                                                                           \\
            Value(Expr)     & $v$     & $\rightarrow$ & $\langle \lambda x.e, C \rangle$
          \end{tabular}

          $\addr(C,x)$: $C$의 가장 위에 있는 $x$의 주소(시간).

          $\modctx(C,M)$: $C$의 가장 위에 있는 $M$의 값(환경).
        \end{figure}
        \heading{실행의미 (Operational)}
        \begin{figure}[h!]
          \large
          \begin{flushright}
            \fbox{$(e,C,\mem,t)\semarrow_\tick(V,\mem',t')\text{ or }(e',C',\mem',t')$}
          \end{flushright}
          \vspace{0pt} % -0.75em}
          \normalsize
          \[
            \begin{prooftree}
              \hypo{
                \begin{matrix}
                  t_{x}=\addr(C,x) \\
                  v=\mem(t_{x})
                \end{matrix}
              }
              \infer[left label=ExprVar]1{
              (x, C, \mem, t)
              \semarrow
              (v, \mem, t)
              }
            \end{prooftree}\qquad
            \begin{prooftree}
              \infer[left label=Fn]0{
              (\lambda x.e, C, \mem, t)
              \semarrow
              (\langle\lambda x.e, C\rangle, \mem, t)
              }
            \end{prooftree}
          \]
          \[
            \begin{prooftree}
              \infer[left label={AppL}]0{
              (e_{1}\:e_{2}, C, \mem, t)
              \semarrow
              (e_{1},C, \mem,t)
              }
            \end{prooftree}\qquad
            \begin{prooftree}
              \hypo{
                \begin{matrix}
                  (e_{1}, C, \mem, t)
                  \semarrow
                  (\langle\lambda x.e_{\lambda}, C_{\lambda}\rangle, \mem_{\lambda}, t_{\lambda})
                \end{matrix}
              }
              \infer[left label={AppR}]1{
              (e_{1}\:e_{2}, C, \mem, t)
              \semarrow
              (e_{2}, C, \mem_{\lambda}, t_{\lambda})
              }
            \end{prooftree}
          \]
          \[
            \begin{prooftree}
              \hypo{
                \begin{matrix}
                  (e_{1}, C, \mem, t)
                  \semarrow
                  (\langle\lambda x.e_{\lambda}, C_{\lambda}\rangle, \mem_{\lambda}, t_{\lambda}) \\
                  (e_{2}, C, \mem_{\lambda}, t_{\lambda})
                  \semarrow
                  (v, \mem_{a}, t_{a})
                \end{matrix}
              }
              \infer[left label={AppBody}]1{
              (e_{1}\:e_{2}, C, \mem, t)
              \semarrow
              (e_{\lambda}, (x, t_{a})\cons C_{\lambda}, \mem_{a}[t_{a}\mapsto v], \tick((C,\mem_{a},t_{a}),x,v))
              }
            \end{prooftree}
          \]
        \end{figure}
        \heading{실행의미 (Collecting)}
        \begin{figure}[h!]
          \large
          \fbox{$
              \mathsf{Step}(A)\triangleq
              \left\{\ell\semarrow_\tick\rho, (\rho,\tick)|
              \begin{prooftree}[center=true]
                \hypo{A'}\infer1{\ell\semarrow_\tick\rho}
              \end{prooftree}\wedge
              A'\subseteq A\wedge(\ell,\tick)\in A
              \right\}$}

          \fbox{$
          \sembracket{e}S\triangleq
          \mathsf{lfp}(\lambda X.\mathsf{Step}(X)\cup\{((e,s),\tick)|(s,\tick)\in S\})$}
        \end{figure}

        \begin{center}
          모르는 정보가 있어도, 불완전한 증명나무를 저장하고 있다.
        \end{center}
      \end{block}

    \end{column}

    \separatorcolumn

    \begin{column}{\colwidth}
      \begin{block}{``합치기''의 정의}
        \heading{정리 (Concrete Linking)}
        $|\sembracket{e_1}S|\cong S_1\rhd S_2$임이 확인됐을 때,
        \[|\sembracket{\link{e_1}{e_2}}S|\cong|S_1\semlink\sembracket{e_2}S_2|\]이다.
        \begin{itemize}
          \item $|\sembracket{e_1}S|$ : $e_1$이 내보내는 환경이
          \item $S_1\rhd S_2$ : 가정된 $S_2$와 부족했던 $S_1$으로 쪼개질 수 있다.
          \item $|\cdot|$ : 최종 상태에 관심이 있다.
          \item $\cong$ : ``동일한 이름들을 내보내는'' 의미구조면 된다.
        \end{itemize}
        \heading{$\cong$의 정의}
        $C,m$ : \underline{이름}($x,M$)으로만 접근 가능한 구조들.
        \vspace{-1em}
        \begin{figure}[h!]
          \large
          \begin{align*}
            \mathsf{Step}_m(G) & \triangleq  \{C\xrightarrow{x}t,t|C\in G\wedge t=\addr(C,x)\}      \\
                               & \cup \{C\xrightarrow{M}C',C'|C\in G\wedge C'=\modctx(C,M)\}        \\
                               & \cup \{t\xrightarrow{e}C,C|t\in G\wedge \langle e,C \rangle=m(t)\} \\
            \underline{C,m}    & \triangleq \mathsf{lfp}(\lambda X.\mathsf{Step}_m(X)\cup\{C\})
          \end{align*}
        \end{figure}
        $\underline{C,m}$ : 이름으로 뽑아낼 수 있는 모든 정보를 담은 rooted graph.

        $(C_1,\mem_1)\cong(C_2,\mem_2)\triangleq\underline{C_1,\mem_1}\cong\underline{C_2,\mem_2}$ : 그래프가 동형
        \heading{$\rhd$, $\semlink$의 정의}
        $(s_1,\tick_1)\rhd(s_2,\tick_2)\triangleq(s_+,\tick_+)$
        \begin{itemize}
          \item $s_+$는 $s_1=(C_1,\mem_1,t_1)$를 $s_2=(C_2,\mem_2,t_2)$에 \textbf{끼워넣은} 것.
          \item $\tick_+$는 $t\in\Time_1+\Time_2$를 $t\in\Time_1$의 경우 $\tick_1$으로, $t\in\Time_2$의 경우 \textbf{끼워넣기 전} 상태를 바탕으로 $\tick_2$로 증가시키는 함수.
        \end{itemize}
        \begin{center}
          \fbox{$S\semlink A\triangleq\underbrace{\mathsf{lfp}(\lambda X.\mathsf{Step}(X)\cup\overbrace{(S\rhd A)}^{\text{끼워넣고}})}_{\text{나머지를 채운다}}$}
        \end{center}
        \heading{도움정리 (Advance)}
        따로따로 나눌 수 있는 곳에서 출발하면, 미리 하고 합친 것과 같다.
        \[\sembracket{e}(S_1\rhd S_2)=S_1\semlink\sembracket{e}S_2\]
      \end{block}
      \begin{block}{분석을 위해서}
        $\tick$ : 무한히 증가하는 시간 생성 (이미 있는 주소와 안 겹치게)

        $\A\tick$ : 유한한 시간 생성

        $\alpha:\Time\rightarrow\A\Time$ : $\A\tick\circ \alpha=\alpha\circ\tick$이고, 

      \end{block}
    \end{column}

    \separatorcolumn
  \end{columns}
\end{frame}

\end{document}
