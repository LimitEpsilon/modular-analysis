%! TEX program = xelatex
\documentclass{article}
\usepackage{geometry}
\geometry{
  a4paper,
  total={170mm,257mm},
  left=20mm,
  top=20mm,
}
\usepackage{setspace}
\usepackage{graphicx}
\usepackage{xcolor}
\usepackage{kotex}
\usepackage{csquotes}

%%% Typesetting for listings
\usepackage{listings}

\definecolor{dkgreen}{rgb}{0,0.6,0}
\definecolor{ltblue}{rgb}{0,0.4,0.4}
\definecolor{dkviolet}{rgb}{0.3,0,0.5}

%%% lstlisting coq style (inspired from a file of Assia Mahboubi)
\lstdefinelanguage{Coq}{ 
    % Anything betweeen $ becomes LaTeX math mode
    mathescape=true,
    % Comments may or not include Latex commands
    texcl=false, 
    % Vernacular commands
    morekeywords=[1]{Section, Module, End, Require, Import, Export,
        Variable, Variables, Parameter, Parameters, Axiom, Hypothesis,
        Hypotheses, Notation, Local, Tactic, Reserved, Scope, Open, Close,
        Bind, Delimit, Definition, Let, Ltac, Fixpoint, CoFixpoint, Add,
        Morphism, Relation, Implicit, Arguments, Unset, Contextual,
        Strict, Prenex, Implicits, Inductive, CoInductive, Record,
        Structure, Canonical, Coercion, Context, Class, Global, Instance,
        Program, Infix, Theorem, Lemma, Corollary, Proposition, Fact,
        Remark, Example, Proof, Goal, Save, Qed, Defined, Hint, Resolve,
        Rewrite, View, Search, Show, Print, Printing, All, Eval, Check,
        Projections, inside, outside, Def},
    % Gallina
    morekeywords=[2]{forall, exists, exists2, fun, fix, cofix, struct,
        match, with, end, as, in, return, let, if, is, then, else, for, of,
        nosimpl, when},
    % Sorts
    morekeywords=[3]{Type, Prop, Set, true, false, option},
    % Various tactics, some are std Coq subsumed by ssr, for the manual purpose
    morekeywords=[4]{pose, set, move, case, elim, apply, clear, hnf,
        intro, intros, generalize, rename, pattern, after, destruct,
        induction, using, refine, inversion, injection, rewrite, congr,
        unlock, compute, ring, field, fourier, replace, fold, unfold,
        change, cutrewrite, simpl, have, suff, wlog, suffices, without,
        loss, nat_norm, assert, cut, trivial, revert, bool_congr, nat_congr,
        symmetry, transitivity, auto, split, left, right, autorewrite},
    % Terminators
    morekeywords=[5]{by, done, exact, reflexivity, tauto, romega, omega,
        assumption, solve, contradiction, discriminate},
    % Control
    morekeywords=[6]{do, last, first, try, idtac, repeat},
    % Comments delimiters, we do turn this off for the manual
    morecomment=[s]{(*}{*)},
    % Spaces are not displayed as a special character
    showstringspaces=false,
    % String delimiters
    morestring=[b]",
    morestring=[d],
    % Size of tabulations
    tabsize=3,
    % Enables ASCII chars 128 to 255
    extendedchars=false,
    % Case sensitivity
    sensitive=true,
    % Automatic breaking of long lines
    breaklines=false,
    % Default style fors listings
    basicstyle=\small,
    % Position of captions is bottom
    captionpos=b,
    % flexible columns
    columns=[l]flexible,
    % Style for (listings') identifiers
    identifierstyle={\ttfamily\color{black}},
    % Style for declaration keywords
    keywordstyle=[1]{\ttfamily\color{dkviolet}},
    % Style for gallina keywords
    keywordstyle=[2]{\ttfamily\color{dkgreen}},
    % Style for sorts keywords
    keywordstyle=[3]{\ttfamily\color{ltblue}},
    % Style for tactics keywords
    keywordstyle=[4]{\ttfamily\color{dkblue}},
    % Style for terminators keywords
    keywordstyle=[5]{\ttfamily\color{dkred}},
    %Style for iterators
    %keywordstyle=[6]{\ttfamily\color{dkpink}},
    % Style for strings
    stringstyle=\ttfamily,
    % Style for comments
    commentstyle={\ttfamily\color{dkgreen}},
    %moredelim=**[is][\ttfamily\color{red}]{/&}{&/},
    literate=
    {\\forall}{{\color{dkgreen}{$\forall\;$}}}1
    {\\exists}{{$\exists\;$}}1
    {\ge -}{{$\geftarrow\;$}}1
    {=>}{{$\Rightarrow\;$}}1
    {==}{{\code{==}\;}}1
    {==>}{{\code{==>}\;}}1
    %    {:>}{{\code{:>}\;}}1
    {->}{{$\rightarrow\;$}}1
    {\ge ->}{{$\geftrightarrow\;$}}1
    {\ge ==}{{$\geq\;$}}1
    {\#}{{$^\star$}}1 
    {\\o}{{$\circ\;$}}1 
    {\@}{{$\cdot$}}1 
    {\/\\}{{$\text{ and }\;$}}1
    {\\\/}{{$\vee\;$}}1
    {++}{{\code{++}}}1
    {~}{{\ }}1
    {\@\@}{{$@$}}1
    {\\mapsto}{{$\mapsto\;$}}1
    {\\hline}{{\rule{\linewidth}{0.5pt}}}1
    %
}[keywords,comments,strings]

%%% Math settings
\usepackage{amssymb,amsmath,amsthm,mathtools}
\usepackage[math-style=TeX,bold-style=TeX]{unicode-math}
\theoremstyle{definition}
\newtheorem{definition}{Definition}[section]
\newtheorem{example}{Example}[section]
\newtheorem{lem}{Lemma}[section]
\newtheorem{thm}{Theorem}[section]
\newtheorem{cor}{Corollary}[section]
\newtheorem{clm}{Claim}[section]

%%% Font settings
%\setmainfont{Libertinus Serif}
%\setsansfont{Libertinus Sans}[Scale=MatchUppercase]
%\setmonofont{JuliaMono}[Scale=MatchLowercase]
%\setmainhangulfont{Noto Serif CJK KR}
%\setmonohangulfont{D2Coding}

%%% PL constructs
\usepackage{galois}
\usepackage{ebproof}
\ebproofset{left label template=\textsc{[\inserttext]}}
\ebproofset{center=false}

%%% Custom commands
\newcommand*{\vbar}{|}
\newcommand*{\finto}{\xrightarrow{\text{\textrm{fin}}}}
\newcommand*{\istype}{\mathrel{⩴}}
\newcommand*{\ortype}{\mathrel{|}}
\newcommand*{\cons}{::}
\newcommand*{\pset}{\mathscr{P}}

\def\ovbarw{1.2mu}
\def\ovbarh{1}
\newcommand*{\ovbar}[1]{\mkern \ovbarw\overline{\mkern-\ovbarw{\smash{#1}\scalebox{1}[\ovbarh]{\vphantom{i}}}\mkern-\ovbarw}\mkern \ovbarw}
\newcommand*{\A}[1]{\overset{{\,_{\mbox{\Large .}}}}{#1}}
\newcommand*{\Abs}[1]{{#1}^{\#}}
\newcommand*{\Expr}{\text{Expr}}
\newcommand*{\ExprVar}{\text{Var}}
\newcommand*{\Module}{\text{Module}}
\newcommand*{\ModVar}{\text{ModVar}}
\newcommand*{\modid}{d}
\newcommand*{\Time}{\mathbb{T}}
\newcommand*{\Addr}{\text{Addr}}
\newcommand*{\ATime}{\A{\Time}}
\newcommand*{\Ctx}{\text{Ctx}}
\newcommand*{\ctx}{\sigma}
\newcommand*{\Value}{\text{Val}}
\newcommand*{\Mem}{\text{Heap}}
\newcommand*{\Left}{\text{Left}}
\newcommand*{\Right}{\text{Right}}
\newcommand*{\Sig}{\text{Sig}}
\newcommand*{\mem}{h}
\newcommand*{\AMem}{\A{\text{Heap}}}
\newcommand*{\Config}{\text{Config}}
\newcommand*{\State}{\text{State}}
\newcommand*{\Outcome}{\text{Outcome}}
\newcommand*{\AState}{\A{\text{State}}}
\newcommand*{\Result}{\text{Result}}
\newcommand*{\AResult}{\A{\text{Result}}}
\newcommand*{\Tick}{\text{Tick}}
\newcommand*{\Judge}{\text{Judge}}
\newcommand*{\lfp}{\mathsf{lfp}}
\newcommand*{\Step}{\mathsf{Step}}
\newcommand*{\Eval}{\mathsf{Eval}}
\newcommand*{\semarrow}{\hookrightarrow}
\newcommand*{\asemarrow}{\A{\rightsquigarrow}}
\newcommand*{\synlink}{\rtimes}
\newcommand*{\semlink}{\mathbin{\rotatebox[origin=c]{180}{$\propto$}}}
\newcommand*{\link}[2]{{#1}\rtimes{#2}}
\newcommand*{\mt}{\mathsf{emp}}
\newcommand*{\valid}{\checkmark}
\newcommand*{\Path}{\text{Path}}
\newcommand*{\equivalent}{\equiv}

\newcommand*{\doubleplus}{\ensuremath{\mathbin{+\mkern-3mu+}}}
\newcommand*{\project}{\text{\texttt{:>} }}
\newcommand*{\sembracket}[1]{\lBrack{#1}\rBrack}
\newcommand*{\fin}[2]{{#1}\xrightarrow{\text{fin}}{#2}}
\newcommand*{\addr}{\mathsf{addr}}
\newcommand*{\tick}{\mathsf{tick}}
\newcommand*{\fresh}{\mathsf{fresh}}
\newcommand*{\modctx}{\mathsf{ctx}}
\newcommand*{\inject}[2]{{#2}\langle{#1}\rangle}
\newcommand*{\Lete}{\mathtt{val}}
\newcommand*{\Letm}{\mathtt{mod}}

\newcommand*{\ValRel}[1]{\mathcal{V}\sembracket{#1}}
\newcommand*{\ExprRel}[1]{\mathcal{E}\sembracket{#1}}
\newcommand*{\CtxRel}[1]{\mathcal{C}\sembracket{#1}}
\newcommand*{\ModRel}[1]{\mathcal{M}\sembracket{#1}}
\newcommand*{\TyEnv}{\text{TyEnv}}
\newcommand*{\TyVar}{\text{TyVar}}
\newcommand*{\Type}{\text{Type}}
\newcommand*{\Subst}{\text{Subst}}
\newcommand*{\external}{\Gamma_{\text{ext}}}

\title{Control Flow Analysis as an Instance for Modular Analysis}
\author{Joonhyup Lee}
\begin{document}
\maketitle
\section{0CFA for the Simple Module Language}
\begin{figure}[htb]
  \centering
  \begin{tabular}{rrcll}
    Identifiers & $x,\modid$ & $\in$         & $\ExprVar$                                                             \\
    Expression  & $e$        & $\rightarrow$ & $x$ $\vbar$ $\lambda x.e$ $\vbar$ $e$ $e$ & untyped $\lambda$-calculus \\
                &            & $\vbar$       & $\link{m}{e}$                             & linked expression          \\
    Module      & $m$        & $\rightarrow$ & $\varepsilon$                             & empty module               \\
                &            & $\vbar$       & $\modid$                                  & module identifier          \\
                &            & $\vbar$       & $\Lete$ $x$ $e$ $m$                       & value binding              \\
                &            & $\vbar$       & $\Letm$ $\modid$ $m$ $m$                  & module binding             \\
  \end{tabular}
  \caption{Abstract syntax of the simple module language.}
  \label{fig:syntax}
\end{figure}
\subsection{Instrumented Operational Semantics}
\begin{figure}[h!]
  \centering
  \begin{tabular}{rrcll}
    Timestamp            & $t$    & $\in$         & $\Time$                                                          \\
    Address              & $\ell$ & $\in$         & $\Addr\triangleq\mathbb{N}\times\Time$                           \\
    Environment/Context  & $\ctx$ & $\in$         & $\Ctx\triangleq\fin{\ExprVar}{\Addr+\Ctx}$                       \\
    Value of expressions & $v$    & $\in$         & $\Value\triangleq\ExprVar\times\Expr\times\Ctx$                  \\
    Heap                 & $\mem$ & $\in$         & $\Mem\triangleq\fin{\Addr}{\Value}$                              \\
    State                & $s$    & $\in$         & $\State\triangleq\Ctx\times\Mem$                                 \\
    Context              & $\ctx$ & $\rightarrow$ & $\bullet$                                       & empty stack    \\
                         &        & $\vbar$       & $(x,\ell)\cons \ctx$                            & value binding  \\
                         &        & $\vbar$       & $(\modid,\ctx)\cons \ctx$                       & module binding \\
    Value of expressions & $v$    & $\rightarrow$ & $\langle \lambda x.e, \ctx \rangle$             & closure
  \end{tabular}
  \caption{Definition of the semantic domains.}
  \label{fig:simpdom}
\end{figure}

\begin{figure}[h!]
  \footnotesize
  \begin{flushright}
    \fbox{$(e,s)\Downarrow (v,\mem)$ and $(m,s)\Downarrow (\ctx,\mem)$}
  \end{flushright}
  \centering
  \vspace{0pt} % -0.75em}
  \[
    \begin{prooftree}
      \hypo{
        \begin{matrix}
          \ell=\ctx(x) \\
          v=\mem(\ell)
        \end{matrix}}
      \infer[left label=ExprID]1{
      (x, \ctx,\mem)
      \Downarrow
      (v,\mem)
      }
    \end{prooftree}\qquad
    \begin{prooftree}
      \infer[left label=Fn]0{
      (\lambda x.e, \ctx,\mem)
      \Downarrow
      (\langle\lambda x.e, \ctx\rangle,\mem)
      }
    \end{prooftree}\qquad
    \begin{prooftree}
      \hypo{
        \begin{matrix}
          (e_{1}, \ctx, \mem)
          \Downarrow
          (\langle\lambda x.e, \ctx_{\lambda}\rangle,\mem_\lambda) \\
          (e_{2}, \ctx, \mem_\lambda)
          \Downarrow
          (v, \mem_a)                                              \\
          (e, (x, \ell)\cons \ctx_{\lambda}, \mem_a[\ell\mapsto v])
          \Downarrow
          (v', \mem')
        \end{matrix}
      }
      \hypo{
        \begin{matrix}
          n\in\fresh(\mem_a) \\
          t=\tick(x)         \\
          \ell=(n,t)
        \end{matrix}
      }
      \infer[left label={App}]2{
      (e_{1}\:e_{2}, \ctx, \mem)
      \Downarrow
      (v', \mem')
      }
    \end{prooftree}
  \]

  \[
    \begin{prooftree}
      \hypo{
        \begin{matrix}
          (m_{1}, \ctx, \mem)
          \Downarrow
          (\ctx', \mem') \\
          (e_{2}, \ctx', \mem')
          \Downarrow
          (v',\mem'')
        \end{matrix}
      }
      \infer[left label=Link]1{
      (\link{m_{1}}{e_{2}}, \ctx, \mem)
      \Downarrow
      (v',\mem'')
      }
    \end{prooftree}\qquad
    \begin{prooftree}
      \infer[left label=Empty]0{
      (\varepsilon, \ctx, \mem)
      \Downarrow
      (\bullet, \mem)
      }
    \end{prooftree}\qquad
    \begin{prooftree}
      \hypo{\ctx'=\ctx(\modid)}
      \infer[left label=ModID]1{
      (\modid, \ctx, \mem)
      \Downarrow
      (\ctx', \mem)
      }
    \end{prooftree}
  \]

  \[
    \begin{prooftree}
      \hypo{
        \begin{matrix}
          (e_{1}, \ctx, \mem)
          \Downarrow
          (v, \mem') \\
          (m_{2}, (x, \ell)\cons \ctx, \mem'[\ell\mapsto v])
          \Downarrow
          (\ctx', \mem'')
        \end{matrix}
      }
      \hypo{
        \begin{matrix}
          n\in\fresh(\mem') \\
          t=\tick(x)        \\
          \ell=(n,t)
        \end{matrix}
      }
      \infer[left label=LetE]2{
      (\Lete\:x\:e_1\:m_2, \ctx, \mem)
      \Downarrow
      ((x,\ell)\cons\ctx', \mem'')
      }
    \end{prooftree}\qquad
    \begin{prooftree}
      \hypo{
        \begin{matrix}
          (m_{1}, \ctx, \mem)
          \Downarrow
          (\ctx', \mem') \\
          (m_{2}, (\modid, \ctx')\cons \ctx, \mem')
          \Downarrow
          (\ctx'', \mem'')
        \end{matrix}
      }
      \infer[left label=LetM]1{
      (\Letm\:\modid\:m_{1}\:m_{2}, \ctx, \mem)
      \Downarrow
      ((\modid,\ctx')\cons\ctx'', \mem'')
      }
    \end{prooftree}
  \]
  \caption{The big-step operational semantics.}
  \label{fig:simpeval}
\end{figure}
\subsection{Adding Intermediate Transitions}
\begin{figure}[h!]
  \footnotesize
  \begin{flushright}
    \fbox{$(e\text{ or }m,s)\semarrow (e\text{ or }m,s)$}
  \end{flushright}
  \centering
  \vspace{0pt} % -0.75em}
  \[
    \begin{prooftree}
      \infer[left label={AppL}]0{
      (e_{1}\:e_{2}, \ctx, \mem)
      \semarrow
      (e_{1}, \ctx, \mem)
      }
    \end{prooftree}\qquad
    \begin{prooftree}
      \hypo{
        \begin{matrix}
          (e_{1}, \ctx, \mem)
          \Downarrow
          (\langle\lambda x.e, \ctx_{\lambda}\rangle,\mem_\lambda)
        \end{matrix}
      }
      \infer[left label={AppR}]1{
      (e_{1}\:e_{2}, \ctx, \mem)
      \semarrow
      (e_{2}, \ctx, \mem_\lambda)
      }
    \end{prooftree}
  \]

  \[
    \begin{prooftree}
      \hypo{
        \begin{matrix}
          (e_{1}, \ctx, \mem)
          \Downarrow
          (\langle\lambda x.e, \ctx_{\lambda}\rangle,\mem_\lambda) \\
          (e_{2}, \ctx, \mem_\lambda)
          \Downarrow
          (v, \mem_a)
        \end{matrix}
      }
      \hypo{
        \begin{matrix}
          n\in\fresh(\mem_a) \\
          t=\tick(x)         \\
          \ell=(n,t)
        \end{matrix}
      }
      \infer[left label={AppBody}]2{
      (e_{1}\:e_{2}, \ctx, \mem)
      \semarrow
      (e, (x, \ell)\cons \ctx_{\lambda}, \mem_a[\ell\mapsto v])
      }
    \end{prooftree}
  \]

  \[
    \begin{prooftree}
      \infer[left label=LinkL]0{
      (\link{m_{1}}{e_{2}}, \ctx, \mem)
      \semarrow
      (m_{1}, \ctx, \mem)
      }
    \end{prooftree}\qquad
    \begin{prooftree}
      \hypo{
        \begin{matrix}
          (m_{1}, \ctx, \mem)
          \Downarrow
          (\ctx', \mem')
        \end{matrix}
      }
      \infer[left label=LinkR]1{
      (\link{m_{1}}{e_{2}}, \ctx, \mem)
      \semarrow
      (e_{2}, \ctx', \mem')
      }
    \end{prooftree}
  \]

  \[
    \begin{prooftree}
      \infer[left label=LetEL]0{
      (\Lete\:x\:e_1\:m_2, \ctx, \mem)
      \semarrow
      (e_{1}, \ctx, \mem)
      }
    \end{prooftree}\qquad
    \begin{prooftree}
      \hypo{
        \begin{matrix}
          (e_{1}, \ctx, \mem)
          \Downarrow
          (v, \mem')
        \end{matrix}
      }
      \hypo{
        \begin{matrix}
          n\in\fresh(\mem') \\
          t=\tick(x)        \\
          \ell=(n,t)
        \end{matrix}
      }
      \infer[left label=LetER]2{
      (\Lete\:x\:e_1\:m_2, \ctx, \mem)
      \semarrow
      (m_{2}, (x, \ell)\cons \ctx, \mem'[\ell\mapsto v])
      }
    \end{prooftree}
  \]

  \[
    \begin{prooftree}
      \infer[left label=LetML]0{
      (\Letm\:\modid\:m_{1}\:m_{2}, \ctx, \mem)
      \semarrow
      (m_{1}, \ctx, \mem)
      }
    \end{prooftree}\qquad
    \begin{prooftree}
      \hypo{
        \begin{matrix}
          (m_{1}, \ctx, \mem)
          \Downarrow
          (\ctx', \mem')
        \end{matrix}
      }
      \infer[left label=LetMR]1{
      (\Letm\:\modid\:m_{1}\:m_{2}, \ctx, \mem)
      \semarrow
      (m_{2}, (\modid, \ctx')\cons \ctx, \mem')
      }
    \end{prooftree}
  \]
  \caption{The transition relation for collecting all intermediate configurations.}
  \label{fig:simpreach}
\end{figure}

\subsection{Collecting Semantics}
\begin{figure}[h!]
  \centering
  \begin{tabular}{rrcll}
    Configuration & $c$ & $\in$       & $\Config\triangleq(\Expr+\Module)\times\State$ \\
    Result        & $r$ & $\in$       & $\Result\triangleq(\Value+\Ctx)\times\Mem$     \\
    Judgements    & $J$ & $\subseteq$ & $\Judge\triangleq\Config\:+\Downarrow$
  \end{tabular}
  \caption{Definition of additional semantic domains for collecting semantics.}
  \label{fig:coldom}
\end{figure}
\begin{align*}
  \Eval(J)        & \triangleq
  \left\{c\Downarrow r\middle|
  \begin{prooftree}[center=true]
    \hypo{P}\infer1{c\Downarrow r}
  \end{prooftree}\text{ and }
  P\subseteq J\text{ and }c\in J
  \right\}                     \\
  \Step(J)        & \triangleq
  \left\{c'\middle|
  \begin{prooftree}[center=true]
    \hypo{P}\infer1{c\semarrow c'}
  \end{prooftree}\text{ and }
  P\subseteq J\text{ and }c\in J
  \right\}                     \\
  \sembracket{e}S & \triangleq
  \lfp(\lambda X.\Step(X)\cup\Eval(X)\cup\{(e,s)|s\in S\})
\end{align*}
\subsection{Abstract Operational Semantics}
\begin{figure}[h!]
  \centering
  \begin{tabular}{rrcll}
    Abstract Context & $\A\ctx$ & $\in$         & $\A\Ctx\triangleq\fin{\ExprVar}{\Time+\A\Ctx}$                         \\
    Abstract Value   & $\A{v}$  & $\in$         & $\A\Value\triangleq\ExprVar\times\Expr\times\A\Ctx$                    \\
    Abstract Heap    & $\A\mem$ & $\in$         & $\A\Mem\triangleq\fin{\Time}{\pset(\A\Value)}$                         \\
    Abstract State   & $\A{s}$  & $\in$         & $\A\State\triangleq\A\Ctx\times\A\Mem$                                 \\
    Abstract Context & $\A\ctx$ & $\rightarrow$ & $\bullet$                                           & empty stack      \\
                     &          & $\vbar$       & $(x,t)\cons \A\ctx$                                 & value binding    \\
                     &          & $\vbar$       & $(\modid,\A\ctx)\cons \A\ctx$                       & module binding   \\
    Abstract Value   & $\A{v}$  & $\rightarrow$ & $\langle \lambda x.e, \A\ctx \rangle$               & abstract closure
  \end{tabular}
  \caption{Definition of the abstract semantic domains.}
  \label{fig:absdom}
\end{figure}
\begin{figure}[h!]
  \footnotesize
  \begin{flushright}
    \fbox{$(e,\A{s})\A\Downarrow (\A{v},\A\mem)$ and $(m,\A{s})\A\Downarrow (\A\ctx,\A\mem)$}
  \end{flushright}
  \centering
  \vspace{0pt} % -0.75em}
  \[
    \begin{prooftree}
      \hypo{
        \begin{matrix}
          t=\A\ctx(x) \\
          \A{v}\in\A\mem(t)
        \end{matrix}}
      \infer[left label=ExprID]1{
      (x, \A\ctx,\A\mem)
      \A\Downarrow
      (\A{v},\A\mem)
      }
    \end{prooftree}\qquad
    \begin{prooftree}
      \infer[left label=Fn]0{
      (\lambda x.e, \A\ctx,\A\mem)
      \A\Downarrow
      (\langle\lambda x.e, \A\ctx\rangle,\A\mem)
      }
    \end{prooftree}\qquad
    \begin{prooftree}
      \hypo{
        \begin{matrix}
          (e_{1}, \A\ctx, \A\mem)
          \A\Downarrow
          (\langle\lambda x.e, \A\ctx_{\lambda}\rangle,\A\mem_\lambda) \\
          (e_{2}, \A\ctx, \A\mem_\lambda)
          \A\Downarrow
          (\A{v}, \A\mem_a)                                            \\
          (e, (x, t)\cons \A\ctx_{\lambda}, \A\mem_a[t\A\mapsto\A{v}])
          \A\Downarrow
          (\A{v'}, \A{\mem'})
        \end{matrix}
      }
      \hypo{
        \begin{matrix}
          t=\tick(x)
        \end{matrix}
      }
      \infer[left label={App}]2{
      (e_{1}\:e_{2}, \A\ctx, \A\mem)
      \A\Downarrow
      (\A{v'}, \A{\mem'})
      }
    \end{prooftree}
  \]

  \[
    \begin{prooftree}
      \hypo{
        \begin{matrix}
          (m_{1}, \A\ctx, \A\mem)
          \A\Downarrow
          (\A{\ctx'}, \A{\mem'}) \\
          (e_{2}, \A{\ctx'}, \A{\mem'})
          \A\Downarrow
          (\A{v'},\A{\mem''})
        \end{matrix}
      }
      \infer[left label=Link]1{
      (\link{m_{1}}{e_{2}}, \A\ctx, \A\mem)
      \A\Downarrow
      (\A{v'},\A{\mem''})
      }
    \end{prooftree}\qquad
    \begin{prooftree}
      \infer[left label=Empty]0{
      (\varepsilon, \A\ctx, \A\mem)
      \A\Downarrow
      (\bullet, \A\mem)
      }
    \end{prooftree}\qquad
    \begin{prooftree}
      \hypo{\A\ctx'=\A\ctx(\modid)}
      \infer[left label=ModID]1{
      (\modid, \A\ctx, \A\mem)
      \A\Downarrow
      (\A{\ctx'}, \A\mem)
      }
    \end{prooftree}
  \]

  \[
    \begin{prooftree}
      \hypo{
        \begin{matrix}
          (e_{1}, \A\ctx, \A\mem)
          \A\Downarrow
          (\A{v}, \A{\mem'}) \\
          (m_{2}, (x, t)\cons \A\ctx, \A{\mem'}[t\A\mapsto\A{v}])
          \A\Downarrow
          (\A{\ctx'}, \A{\mem''})
        \end{matrix}
      }
      \hypo{
        \begin{matrix}
          t=\tick(x)
        \end{matrix}
      }
      \infer[left label=LetE]2{
      (\Lete\:x\:e_1\:m_2, \A\ctx, \A\mem)
      \A\Downarrow
      ((x,t)\cons\A{\ctx'}, \A{\mem''})
      }
    \end{prooftree}\qquad
    \begin{prooftree}
      \hypo{
        \begin{matrix}
          (m_{1}, \A{\ctx}, \A{\mem})
          \A\Downarrow
          (\A{\ctx'}, \A{\mem'}) \\
          (m_{2}, (\modid, \A{\ctx'})\cons \A\ctx, \A{\mem'})
          \A\Downarrow
          (\A{\ctx''}, \A{\mem''})
        \end{matrix}
      }
      \infer[left label=LetM]1{
      (\Letm\:\modid\:m_{1}\:m_{2}, \A\ctx, \A\mem)
      \A\Downarrow
      ((\modid,\A{\ctx'})\cons\A{\ctx''}, \A{\mem''})
      }
    \end{prooftree}
  \]
  \caption{The big-step abstract operational semantics.}
  \label{fig:abseval}
\end{figure}
\begin{figure}[h!]
  \footnotesize
  \begin{flushright}
    \fbox{$(e\text{ or }m,\A{s})\A\semarrow (e\text{ or }m,\A{s})$}
  \end{flushright}
  \centering
  \vspace{0pt} % -0.75em}
  \[
    \begin{prooftree}
      \infer[left label={AppL}]0{
      (e_{1}\:e_{2}, \A\ctx, \A\mem)
      \A\semarrow
      (e_{1}, \A\ctx, \A\mem)
      }
    \end{prooftree}\qquad
    \begin{prooftree}
      \hypo{
        \begin{matrix}
          (e_{1}, \A\ctx, \A\mem)
          \A\Downarrow
          (\langle\lambda x.e, \A\ctx_{\lambda}\rangle,\A\mem_\lambda)
        \end{matrix}
      }
      \infer[left label={AppR}]1{
      (e_{1}\:e_{2}, \A\ctx, \A\mem)
      \A\semarrow
      (e_{2}, \A\ctx, \A\mem_\lambda)
      }
    \end{prooftree}
  \]

  \[
    \begin{prooftree}
      \hypo{
        \begin{matrix}
          (e_{1}, \A\ctx, \A\mem)
          \A\Downarrow
          (\langle\lambda x.e, \A\ctx_{\lambda}\rangle,\A\mem_\lambda) \\
          (e_{2}, \A\ctx, \A\mem_\lambda)
          \A\Downarrow
          (\A{v}, \A\mem_a)
        \end{matrix}
      }
      \hypo{
        \begin{matrix}
          t=\tick(x)
        \end{matrix}
      }
      \infer[left label={AppBody}]2{
      (e_{1}\:e_{2}, \A\ctx, \A\mem)
      \A\semarrow
      (e, (x, t)\cons \A\ctx_{\lambda}, \A\mem_a[t\A\mapsto\A{v}])
      }
    \end{prooftree}
  \]

  \[
    \begin{prooftree}
      \infer[left label=LinkL]0{
      (\link{m_{1}}{e_{2}}, \A\ctx, \A\mem)
      \A\semarrow
      (m_{1}, \A\ctx, \A\mem)
      }
    \end{prooftree}\qquad
    \begin{prooftree}
      \hypo{
        \begin{matrix}
          (m_{1}, \A\ctx, \A\mem)
          \A\Downarrow
          (\A{\ctx'}, \A{\mem'})
        \end{matrix}
      }
      \infer[left label=LinkR]1{
      (\link{m_{1}}{e_{2}}, \A\ctx, \A\mem)
      \A\semarrow
      (e_{2}, \A{\ctx'}, \A{\mem'})
      }
    \end{prooftree}
  \]

  \[
    \begin{prooftree}
      \infer[left label=LetEL]0{
      (\Lete\:x\:e_1\:m_2, \A\ctx, \A\mem)
      \A\semarrow
      (e_{1}, \A\ctx, \A\mem)
      }
    \end{prooftree}\qquad
    \begin{prooftree}
      \hypo{
        \begin{matrix}
          (e_{1}, \A\ctx, \A\mem)
          \A\Downarrow
          (\A{v}, \A{\mem'})
        \end{matrix}
      }
      \hypo{
        \begin{matrix}
          t=\tick(x)
        \end{matrix}
      }
      \infer[left label=LetER]2{
      (\Lete\:x\:e_1\:m_2, \A\ctx, \A\mem)
      \A\semarrow
      (m_{2}, (x, t)\cons \A\ctx, \A{\mem'}[t\A\mapsto\A{v}])
      }
    \end{prooftree}
  \]

  \[
    \begin{prooftree}
      \infer[left label=LetML]0{
      (\Letm\:\modid\:m_{1}\:m_{2}, \A\ctx, \A\mem)
      \A\semarrow
      (m_{1}, \A\ctx, \A\mem)
      }
    \end{prooftree}\qquad
    \begin{prooftree}
      \hypo{
        \begin{matrix}
          (m_{1}, \A\ctx, \A\mem)
          \A\Downarrow
          (\A{\ctx'}, \A{\mem'})
        \end{matrix}
      }
      \infer[left label=LetMR]1{
      (\Letm\:\modid\:m_{1}\:m_{2}, \A\ctx, \A\mem)
      \A\semarrow
      (m_{2}, (\modid, \A{\ctx'})\cons \A\ctx, \A{\mem'})
      }
    \end{prooftree}
  \]
  \caption{The abstract transition relation for collecting all intermediate configurations.}
  \label{fig:absreach}
\end{figure}
\subsection{Abstract Semantics}
\begin{figure}[h!]
  \centering
  \begin{tabular}{rrcll}
    Abstract Configuration & $\A{c}$   & $\in$       & $\A\Config\triangleq(\Expr+\Module)\times\A\State$ \\
    Abstract Result        & $\A{r}$   & $\in$       & $\A\Result\triangleq(\A\Value+\A\Ctx)\times\A\Mem$ \\
    Abstract Judgements    & $\Abs{J}$ & $\subseteq$ & $\A\Judge\triangleq\A\Config\:+\A\Downarrow$
  \end{tabular}
  \caption{Definition of additional semantic domains for collecting semantics.}
  \label{fig:abscoldom}
\end{figure}
\begin{align*}
  \Abs\Eval(\Abs{J})          & \triangleq
  \left\{\A{c}\A\Downarrow\A{r}\middle|
  \begin{prooftree}[center=true]
    \hypo{\Abs{P}}\infer1{\A{c}\A\Downarrow\A{r}}
  \end{prooftree}\text{ and }
  \Abs{P}\subseteq\Abs{J}\text{ and }\A{c}\in\Abs{J}
  \right\}                                 \\
  \Abs\Step(\Abs{J})          & \triangleq
  \left\{\A{c'}\middle|
  \begin{prooftree}[center=true]
    \hypo{\Abs{P}}\infer1{\A{c}\A\semarrow\A{c'}}
  \end{prooftree}\text{ and }
  \Abs{P}\subseteq\Abs{J}\text{ and }\A{c}\in\Abs{J}
  \right\}                                 \\
  \Abs{\sembracket{e}}\Abs{S} & \triangleq
  \lfp(\lambda \Abs{X}.\Abs\Step(\Abs{X})\cup\Abs\Eval(\Abs{X})\cup\{(e,\A{s})|\A{s}\in\Abs{S}\})
\end{align*}
\subsection{Galois Connection}
\begin{align*}
  |\bullet|                 & \triangleq\bullet                                  &  &  & |(x,(\_,t))\cons\ctx|            & \triangleq(x,t)\cons|\ctx|                 \\
  |(\modid,\ctx)\cons\ctx'| & \triangleq(\modid,|\ctx|)\cons|\ctx'|              &  &  & |\langle\lambda x.e,\ctx\rangle| & \triangleq\langle\lambda x.e,|\ctx|\rangle \\
  |\mem|                    & \triangleq\lambda t.\{|\mem(n,t)||n\in\mathbb{N}\} &  &  & |(\_,\_)|                        & \triangleq(|\_|,|\_|)
\end{align*}
\begin{lem}[Galois Connection]
  \[\pset(\Judge)\galois{\alpha}{\gamma}\pset(\A\Judge)\]
  where
  \begin{align*}
    \gamma(\Abs{D}) & \triangleq\{c||c|\in\Abs{D}\}\cup\{c\Downarrow r||c|\A\Downarrow|r|\in\Abs{D}\} \\
    \alpha(D)       & \triangleq\{|c||c\in D\}\cup\{|c|\A\Downarrow|r||c\Downarrow r\in D\}
  \end{align*}
\end{lem}
\subsection{Soundness}
\begin{lem}[Operational Soundness]
  \[c\Downarrow r\Rightarrow|c|\A\Downarrow|r|\]
\end{lem}
\begin{lem}[Soundness of $\Step$, $\Eval$]
  \[\Step\circ\gamma\subseteq\gamma\circ\Abs\Step\qquad\Eval\circ\gamma\subseteq\gamma\circ\Abs\Eval\]
\end{lem}
\begin{thm}[Soundness]
  \[\sembracket{e}\circ\gamma\subseteq\gamma\circ\Abs{\sembracket{e}}\]
\end{thm}
\subsection{0CFA}
Instantiating
\[\tick(x)\triangleq x\]
leads to 0CFA.

%\section{Equivalence}
%The point of the instrumented semantics is that no matter what time domain $\Time$ you take, the concrete semantics stay \emph{equivalent}.
%To formalize this, we define 
%\begin{figure}[h!]
%  \centering
%  \begin{tabular}{rclr}
%    $p$                                   & $\rightarrow$ & $\epsilon$                              \\
%                                          & $|$           & $\xrightarrow{x}t\:p$                   \\
%                                          & $|$           & $\xrightarrow{\modid}p$                 \\
%                                          & $|$           & $\xrightarrow{\lambda x.e}p$            \\
%    $\varphi(\epsilon)$                   & $\triangleq$  & $\epsilon$                              \\
%    $\varphi(\xrightarrow{x}t\:p)$        & $\triangleq$  & $\xrightarrow{x}\varphi(t)\:\varphi(p)$ \\
%    $\varphi(\xrightarrow{M}p)$           & $\triangleq$  & $\xrightarrow{M}\varphi(p)$             \\
%    $\varphi(\xrightarrow{\lambda x.e}p)$ & $\triangleq$  & $\xrightarrow{\lambda x.e}\varphi(p)$
%  \end{tabular}
%  \begin{minipage}{0.3\linewidth}
%    \footnotesize
%    \[
%      \begin{prooftree}
%        \infer0{\valid(\_,\mem,\epsilon)}
%      \end{prooftree}
%    \]
%
%    \[
%      \begin{prooftree}
%        \hypo{t=\ctx(x)\qquad\valid(t,\mem,p)}
%        \infer1{\valid(\ctx,\mem,\xrightarrow{x}t\:p)}
%      \end{prooftree}
%    \]
%
%    \[
%      \begin{prooftree}
%        \hypo{\ctx'=\ctx(\modid)\qquad\valid(\ctx',\mem,p)}
%        \infer1{\valid(\ctx,\mem,\xrightarrow{\modid}p)}
%      \end{prooftree}
%    \]
%
%    \[
%      \begin{prooftree}
%        \hypo{\langle\lambda x.e,{\ctx}\rangle=\mem(t)\qquad\valid(\ctx,\mem,p)}
%        \infer1{\valid(t,\mem,\xrightarrow{\lambda x.e}p)}
%      \end{prooftree}
%    \]
%  \end{minipage}
%  \begin{minipage}{0.3\linewidth}
%    \footnotesize
%    \[
%      \begin{prooftree}
%        \infer0{\A\valid(\_,\A\mem,\epsilon)}
%      \end{prooftree}
%    \]
%
%    \[
%      \begin{prooftree}
%        \hypo{\A{t}=\A\ctx(x)\qquad\A\valid(\A{t},\A\mem,\A{p})}
%        \infer1{\A\valid(\A\ctx,\A\mem,\xrightarrow{x}\A{t}\:\A{p})}
%      \end{prooftree}
%    \]
%
%    \[
%      \begin{prooftree}
%        \hypo{\A\ctx'=\A\ctx(\modid)\qquad\A\valid(\A\ctx',\A\mem,\A{p})}
%        \infer1{\A\valid(\A\ctx,\A\mem,\xrightarrow{\modid}\A{p})}
%      \end{prooftree}
%    \]
%
%    \[
%      \begin{prooftree}
%        \hypo{\langle\lambda x.e,{\A\ctx}\rangle\in\A\mem(\A{t})\qquad\A\valid(\A\ctx,\A\mem,\A{p})}
%        \infer1{\A\valid(\A{t},\A\mem,\xrightarrow{\lambda x.e}\A{p})}
%      \end{prooftree}
%    \]
%  \end{minipage}
%  \caption{Definitions for access paths. Given from the left are: the syntax of access paths, the translation of timestamps, the $\valid$ predicate, and the $\A\valid$ predicate.}
%  \label{fig:valid}
%\end{figure}
%\begin{definition}[Equivalent Concrete States: $\equivalent$]
%  Let $s=(\ctx,\mem)\in\State$ and $s'=(\ctx',\mem')\in\State'$.
%  $s\equivalent s'$ ($s$ is equivalent to $s'$) iff $\exists\varphi\in\Addr\rightarrow\Addr',\varphi'\in\Addr'\rightarrow\Addr:$
%  \begin{enumerate}
%    \item $\forall p\in\Path:\valid(\ctx,\mem,p)\Rightarrow(\valid(\ctx',\mem',\varphi(p))\land p=\varphi'(\varphi(p)))$
%    \item $\forall p'\in\Path':\valid(\ctx',\mem',p')\Rightarrow(\valid(\ctx,\mem,\varphi'(p'))\land p'=\varphi(\varphi'(p')))$
%  \end{enumerate}
%\end{definition}
%\begin{definition}[Weakly Equivalent Abstract States]\label{def:weakequiv}
%  Let $\A{s}=(\A{\ctx},\A\mem)\in\AState$ and $\A{s'}=(\A{\ctx'},\A{\mem'})\in\A{\State'}$.
%  $\A{s}$ is weakly equivalent to $\A{s'}$ iff $\exists\A\varphi\in\Time\rightarrow\Time',\A{\varphi'}\in\Time'\rightarrow\Time:$
%  \begin{enumerate}
%    \item $\forall\A{p}\in\A\Path:\A\valid(\A{\ctx},\A\mem,\A{p})\Rightarrow(\A\valid(\A{\ctx'},\A{\mem'},\A\varphi(\A{p}))\land\A{p}=\A{\varphi'}(\A\varphi(\A{p})))$
%    \item $\forall\A{p'}\in\A{\Path'}:\A\valid(\A{\ctx'},\A{\mem'},\A{p'})\Rightarrow(\A\valid(\A{\ctx},\A{\mem},\A{\varphi'}(\A{p'}))\land\A{p'}=\A\varphi(\A{\varphi'}(\A{p'})))$
%  \end{enumerate}
%\end{definition}
%The reason that the above definition is called ``weak equivalence'' is because it is not sufficient to guarantee equivalence after concretization.
%Consider
%\[
%  \ctx=[(x,0)],\mem=\{0\mapsto\{\langle\lambda z.z,[(x,1)]\rangle,\langle\lambda z.z,[(y,2)]\rangle\},1\mapsto\{\langle\lambda z.z,[]\rangle\}\}
%\]
%and
%\[
%  \ctx'=[(x,0)],\mem'=\{0\mapsto\{\langle\lambda z.z,[(x,1);(y,2)]\rangle\},1\mapsto\{\langle\lambda z.z,[]\rangle\}\}
%\]
%They are weakly equivalent, yet their concretizations are not equivalent.
%Thus, we need to strengthen the definition for abstract equivalence.
%
%Before going into the definition, we introduce some terminology.
%First, we say that two states are \emph{weakly equivalent by} $\A\varphi,\A{\varphi'}$ when $\A\varphi,\A{\varphi'}$ are the functions that translate between abstract timestamps in Definition \ref{def:weakequiv}.
%Second, we say that $\A{t}$ is \emph{reachable from} $\A{s}$ when there is some valid access path $\A{p}$ from $\A{s}$ containing $\A{t}$.
%Now we actually give the definition:
%\begin{definition}[Equivalent Abstract States: $\A\equivalent$]
%  Let $\A{s}=(\_,\A\mem)\in\AState$ and $\A{s'}=(\_,\A{\mem'})\in\A{\State'}$.
%  $\A{s}\A\equivalent\A{s'}$ ($\A{s}$ is equivalent to $\A{s'}$) iff $\exists\A\varphi\in\ATime\rightarrow\A{\Time'},\A{\varphi'}\in\A{\Time'}\rightarrow\ATime:$
%  \begin{enumerate}
%    \item $\A{s}$ and $\A{s'}$ are weakly equivalent by $\A\varphi,\A{\varphi'}$.
%    \item For each $\A{t}$ reachable from $\A{s}$ and for each $\langle\lambda x.e,\A{\ctx}\rangle\in\A\mem(\A{t})$, $\langle\lambda x.e,\exists{\A{\ctx'}}\rangle\in\A{\mem'}(\A\varphi(\A{t}))$ such that $\A{\ctx}$, $\A{\ctx'}$ are weakly equivalent by $\A\varphi,\A{\varphi'}$ under the empty memory.
%    \item The same holds for each $\A{t'}$ reachable from $\A{s'}$.
%  \end{enumerate}
%\end{definition}
%
%
\section{Punching Holes}
Punch holes in the abstract semantics and define injection/linking operators such that:
\[\Abs{\sembracket{e}}(\Abs{S}\rhd S_H)\sqsubseteq\Abs{S}\semlink\sembracket{e}_H S_H\]
The goal is to define:
\begin{center}
  \begin{tabular}{lr}
    $\sembracket{\cdot}_H$ & semantics with holes \\
    $S_H$                  & states with holes    \\
    $\semlink$             & semantic linking     \\
    $\rhd$                 & semantic injection
  \end{tabular}
\end{center}
\subsection{States with Holes}
First define:
\[\ctx_H\in\Ctx_H\]
by:
\begin{center}
  \begin{tabular}{rclr}
    $\ctx_H$ & $\rightarrow$ & $[]$                          & hole           \\
             & $\vbar$       & $\bullet$                     & empty stack    \\
             & $\vbar$       & $(x,t)\cons \ctx_H$           & value binding  \\
             & $\vbar$       & $(\modid,\ctx_H)\cons \ctx_H$ & module binding
  \end{tabular}
\end{center}
and define:
\[\Value_H\qquad\Mem_H\qquad\State_H\qquad\Config_H\qquad\Result_H\qquad\Judge_H\]
accordingly.
\subsection{Semantics with Holes}
Define:
\[\Downarrow_H\subseteq\Config_H\times\Result_H\qquad\semarrow_H\subseteq\Config_H\times\Config_H\]
by using:
\[\tick_H\in\ExprVar\rightarrow\Time_H\]
where $\Time_H$ is some set chosen by the analysis designer.

Define:
\[\Step_H\in\pset(\Judge_H)\rightarrow\pset(\Judge_H)\qquad\Eval_H\in\pset(\Judge_H)\rightarrow\pset(\Judge_H)\]
Define:
\[\sembracket{e}_HS_H\triangleq\lfp(\lambda X_H.\Step_H(X)\cup\Eval_H(X)\cup\{(e,s_H)|s_H\in S_H\})\]
\subsection{Injection}
Define injection into a context with holes:
\[\cdot[\cdot]:\Ctx_H\rightarrow\A\Ctx\rightarrow\A\Ctx_+\]
where
\[\A\Ctx_+\triangleq\fin{\ExprVar}{\Time+\Time_H}\]
by:
\begin{align*}
  [][\A\ctx]                   & \triangleq\A\ctx                     &  &  & \bullet[\A\ctx]                           & \triangleq\bullet                                         \\
  ((x,t)\cons\A\ctx_H)[\A\ctx] & \triangleq(x,t)\cons\A\ctx_H[\A\ctx] &  &  & ((\modid,\A\ctx_H)\cons\A\ctx_H')[\A\ctx] & \triangleq(\modid,\A\ctx_H[\A\ctx])\cons\A\ctx_H'[\A\ctx]
\end{align*}
Define injection into other domains where
\[\qquad\A\Value_+\qquad\A\Mem_+\qquad\A\State_+\qquad\A\Config_+\qquad\A\Result_+\qquad\A\Judge_+\]
are defined accordingly as domains that use $\Time+\Time_H$ for timestamps and $\tick_H$ for $\tick$ by:
\begin{align*}
  \langle\lambda x.e,\ctx_H\rangle[\A\ctx] & \triangleq\langle\lambda x.e,\ctx_H[\A\ctx]\rangle  &  &  & \mem_H[\A\ctx]                & \triangleq\lambda t_H.\{v_H[\A\ctx]|v_H\in\mem_H(t_H)\} \\
  (\ctx_H,\mem_H)[(\A\ctx,\A\mem)]         & \triangleq(\ctx_H[\A\ctx],\A\mem\cup\mem_H[\A\ctx]) &  &  & (v_H,\mem_H)[(\A\ctx,\A\mem)] & \triangleq(v_H[\A\ctx],\A\mem\cup\mem_H[\A\ctx])        \\
  (e,s_H)[\A{s}]                           & \triangleq (e,s_H[\A{s}])                           &  &  & (m,s_H)[\A{s}]                & \triangleq (m,s_H[\A{s}])
\end{align*}

Then we have:
\begin{lem}[Preservation After Injection]
  \[c_H\Downarrow_H r_H\Rightarrow c_H[\A{s}]\A\Downarrow r_H[\A{s}]\qquad c_H\semarrow_H c_H'\Rightarrow c_H[\A{s}]\A\semarrow c_H'[\A{s}]\]
\end{lem}

Define injection into set of states:
\[\rhd:\pset(\A\State)\rightarrow\pset(\State_H)\rightarrow\pset(\A\State_+)\]
by:
\[\Abs{S}\rhd S_H\triangleq\{\ctx_H[\A\ctx]|\A\ctx\in\Abs{S},\ctx_H\in S_H\}\]

\subsection{Linking}
Define injection into set of judgements:
\[\rhd:\pset(\A\State)\rightarrow\pset(\Judge_H)\rightarrow\pset(\A\Judge_+)\]
by:
\[\Abs{S}\rhd J_H\triangleq\{c_H[\A{s}]|\A{s}\in\Abs{S},c_H\in J_H\}\cup\{c_H[\A{s}]\A\Downarrow r_H[\A{s}]|\A{s}\in\Abs{S},c_H\Downarrow_H r_H\in J_H\}\]

Define linking:
\[\semlink:\pset(\A\State)\rightarrow\pset(\Judge_H)\rightarrow\pset(\A\Judge_+)\]
by:
\[\Abs{S}\semlink J_H\triangleq\lfp(\lambda\Abs{X}_+.\Abs{\Step}_+(\Abs{X}_+)\cup\Abs{\Eval}_+(\Abs{X}_+)\cup(\Abs{S}\rhd J_H))\]
\begin{lem}[Advance]
  \[\Abs{\sembracket{e}}(\Abs{S}\rhd S_H)=\Abs{S}\semlink\sembracket{e}_H S_H\]
\end{lem}
\subsection{Separate Modular Analysis}
\begin{thm}[Separate Analysis]
  Let $\mt\triangleq\{(\bullet,\varnothing)\}\subseteq\A\State$. Then:
  \[(\Abs{\sembracket{\link{m}{e}}}\mt)|_{e}\Abs\cong(\Abs{S}\semlink\sembracket{e}_H S_H)|_{e}\]
  when
  \[(\Abs{\sembracket{m}}\mt)|_{m}\Abs\cong\Abs{S}\rhd S_H\]
\end{thm}
\end{document}
%%% Local Variables: 
%%% coding: utf-8
%%% mode: latex
%%% TeX-engine: xetex
%%% End:
