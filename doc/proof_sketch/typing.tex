%! TEX program = xelatex
\documentclass{article}
\usepackage{geometry}
\geometry{
  a4paper,
  total={170mm,257mm},
  left=20mm,
  top=20mm,
}
\usepackage{setspace}
\usepackage{graphicx}
\usepackage{xcolor}
\usepackage{kotex}
\usepackage{csquotes}

%%% Typesetting for listings
\usepackage{listings}

\definecolor{dkgreen}{rgb}{0,0.6,0}
\definecolor{ltblue}{rgb}{0,0.4,0.4}
\definecolor{dkviolet}{rgb}{0.3,0,0.5}

%%% lstlisting coq style (inspired from a file of Assia Mahboubi)
\lstdefinelanguage{Coq}{ 
    % Anything betweeen $ becomes LaTeX math mode
    mathescape=true,
    % Comments may or not include Latex commands
    texcl=false, 
    % Vernacular commands
    morekeywords=[1]{Section, Module, End, Require, Import, Export,
        Variable, Variables, Parameter, Parameters, Axiom, Hypothesis,
        Hypotheses, Notation, Local, Tactic, Reserved, Scope, Open, Close,
        Bind, Delimit, Definition, Let, Ltac, Fixpoint, CoFixpoint, Add,
        Morphism, Relation, Implicit, Arguments, Unset, Contextual,
        Strict, Prenex, Implicits, Inductive, CoInductive, Record,
        Structure, Canonical, Coercion, Context, Class, Global, Instance,
        Program, Infix, Theorem, Lemma, Corollary, Proposition, Fact,
        Remark, Example, Proof, Goal, Save, Qed, Defined, Hint, Resolve,
        Rewrite, View, Search, Show, Print, Printing, All, Eval, Check,
        Projections, inside, outside, Def},
    % Gallina
    morekeywords=[2]{forall, exists, exists2, fun, fix, cofix, struct,
        match, with, end, as, in, return, let, if, is, then, else, for, of,
        nosimpl, when},
    % Sorts
    morekeywords=[3]{Type, Prop, Set, true, false, option},
    % Various tactics, some are std Coq subsumed by ssr, for the manual purpose
    morekeywords=[4]{pose, set, move, case, elim, apply, clear, hnf,
        intro, intros, generalize, rename, pattern, after, destruct,
        induction, using, refine, inversion, injection, rewrite, congr,
        unlock, compute, ring, field, fourier, replace, fold, unfold,
        change, cutrewrite, simpl, have, suff, wlog, suffices, without,
        loss, nat_norm, assert, cut, trivial, revert, bool_congr, nat_congr,
        symmetry, transitivity, auto, split, left, right, autorewrite},
    % Terminators
    morekeywords=[5]{by, done, exact, reflexivity, tauto, romega, omega,
        assumption, solve, contradiction, discriminate},
    % Control
    morekeywords=[6]{do, last, first, try, idtac, repeat},
    % Comments delimiters, we do turn this off for the manual
    morecomment=[s]{(*}{*)},
    % Spaces are not displayed as a special character
    showstringspaces=false,
    % String delimiters
    morestring=[b]",
    morestring=[d],
    % Size of tabulations
    tabsize=3,
    % Enables ASCII chars 128 to 255
    extendedchars=false,
    % Case sensitivity
    sensitive=true,
    % Automatic breaking of long lines
    breaklines=false,
    % Default style fors listings
    basicstyle=\small,
    % Position of captions is bottom
    captionpos=b,
    % flexible columns
    columns=[l]flexible,
    % Style for (listings') identifiers
    identifierstyle={\ttfamily\color{black}},
    % Style for declaration keywords
    keywordstyle=[1]{\ttfamily\color{dkviolet}},
    % Style for gallina keywords
    keywordstyle=[2]{\ttfamily\color{dkgreen}},
    % Style for sorts keywords
    keywordstyle=[3]{\ttfamily\color{ltblue}},
    % Style for tactics keywords
    keywordstyle=[4]{\ttfamily\color{dkblue}},
    % Style for terminators keywords
    keywordstyle=[5]{\ttfamily\color{dkred}},
    %Style for iterators
    %keywordstyle=[6]{\ttfamily\color{dkpink}},
    % Style for strings
    stringstyle=\ttfamily,
    % Style for comments
    commentstyle={\ttfamily\color{dkgreen}},
    %moredelim=**[is][\ttfamily\color{red}]{/&}{&/},
    literate=
    {\\forall}{{\color{dkgreen}{$\forall\;$}}}1
    {\\exists}{{$\exists\;$}}1
    {\le -}{{$\leftarrow\;$}}1
    {=>}{{$\Rightarrow\;$}}1
    {==}{{\code{==}\;}}1
    {==>}{{\code{==>}\;}}1
    %    {:>}{{\code{:>}\;}}1
    {->}{{$\rightarrow\;$}}1
    {\le ->}{{$\leftrightarrow\;$}}1
    {\le ==}{{$\leq\;$}}1
    {\#}{{$^\star$}}1 
    {\\o}{{$\circ\;$}}1 
    {\@}{{$\cdot$}}1 
    {\/\\}{{$\text{ and }\;$}}1
    {\\\/}{{$\vee\;$}}1
    {++}{{\code{++}}}1
    {~}{{\ }}1
    {\@\@}{{$@$}}1
    {\\mapsto}{{$\mapsto\;$}}1
    {\\hline}{{\rule{\linewidth}{0.5pt}}}1
    %
}[keywords,comments,strings]

%%% Math settings
\usepackage{amssymb,amsmath,amsthm,mathtools}
\usepackage[math-style=TeX,bold-style=TeX]{unicode-math}
\theoremstyle{definition}
\newtheorem{definition}{Definition}[section]
\newtheorem{example}{Example}[section]
\newtheorem{lem}{Lemma}[section]
\newtheorem{thm}{Theorem}[section]
\newtheorem{cor}{Corollary}[section]
\newtheorem{clm}{Claim}[section]

%%% Font settings
%\setmainfont{Libertinus Serif}
%\setsansfont{Libertinus Sans}[Scale=MatchUppercase]
%\setmonofont{JuliaMono}[Scale=MatchLowercase]
%\setmainhangulfont{Noto Serif CJK KR}
%\setmonohangulfont{D2Coding}

%%% PL constructs
\usepackage{galois}
\usepackage{ebproof}
\ebproofset{left label template=\textsc{[\inserttext]}}
\ebproofset{center=false}

%%% Custom commands
\newcommand*{\vbar}{|}
\newcommand*{\finto}{\xrightarrow{\text{\textrm{fin}}}}
\newcommand*{\istype}{\mathrel{⩴}}
\newcommand*{\ortype}{\mathrel{|}}
\newcommand*{\cons}{::}
\newcommand*{\pset}{\mathscr{P}}

\def\ovbarw{1.2mu}
\def\ovbarh{1}
\newcommand*{\ovbar}[1]{\mkern \ovbarw\overline{\mkern-\ovbarw{\smash{#1}\scalebox{1}[\ovbarh]{\vphantom{i}}}\mkern-\ovbarw}\mkern \ovbarw}
\newcommand*{\A}[1]{\overset{\,_{\mbox{\Large .}}}{#1}}
\newcommand*{\Abs}[1]{{#1}^{\#}}
\newcommand*{\Expr}{\text{Expr}}
\newcommand*{\ExprVar}{\text{Var}}
\newcommand*{\Module}{\text{Module}}
\newcommand*{\ModVar}{\text{ModVar}}
\newcommand*{\modid}{d}
\newcommand*{\Time}{\mathbb{T}}
\newcommand*{\ATime}{\A{\Time}}
\newcommand*{\Ctx}{\text{Ctx}}
\newcommand*{\ctx}{\sigma}
\newcommand*{\Value}{\text{Val}}
\newcommand*{\Mem}{\text{Mem}}
\newcommand*{\Left}{\text{Left}}
\newcommand*{\Right}{\text{Right}}
\newcommand*{\Sig}{\text{Sig}}
\newcommand*{\mem}{m}
\newcommand*{\AMem}{\A{\text{Mem}}}
\newcommand*{\State}{\text{State}}
\newcommand*{\AState}{\A{\text{State}}}
\newcommand*{\Result}{\text{Result}}
\newcommand*{\AResult}{\A{\text{Result}}}
\newcommand*{\Tick}{\text{Tick}}
\newcommand*{\lfp}{\mathsf{lfp}}
\newcommand*{\Step}{\mathsf{Step}}
\newcommand*{\semarrow}{\Downarrow}
\newcommand*{\asemarrow}{\A{\rightsquigarrow}}
\newcommand*{\synlink}{\rtimes}
\newcommand*{\semlink}{\mathbin{\rotatebox[origin=c]{180}{$\propto$}}}
\newcommand*{\link}[2]{{#1}\rtimes{#2}}
\newcommand*{\mt}{\mathsf{empty}}
\newcommand*{\valid}{\mathsf{valid}}
\newcommand*{\Path}{\text{Path}}
\newcommand*{\equivalent}{\sim}

\newcommand*{\doubleplus}{\ensuremath{\mathbin{+\mkern-3mu+}}}
\newcommand*{\project}{\text{\texttt{:>} }}
\newcommand*{\Exp}{\mathsf{Exp}}
\newcommand*{\Imp}{\mathsf{Imp}}
\newcommand*{\Fin}{\mathsf{Fin}}
\newcommand*{\Link}{\mathsf{Link}}
\newcommand*{\sembracket}[1]{\lBrack{#1}\rBrack}
\newcommand*{\fin}[2]{{#1}\xrightarrow{\text{fin}}{#2}}
\newcommand*{\addr}{\mathsf{addr}}
\newcommand*{\tick}{\mathsf{tick}}
\newcommand*{\modctx}{\mathsf{ctx}}
\newcommand*{\mapinject}[2]{{#2}[{#1}]}
\newcommand*{\inject}[2]{{#2}\langle{#1}\rangle}
\newcommand*{\deletepre}[2]{{#2}\overline{\doubleplus}{#1}}
\newcommand*{\deletemap}[2]{{#1}\overline{[{#2}]}}
\newcommand*{\delete}[2]{{#2}{\langle{#1}\rangle}^{-1}}
\newcommand*{\filter}{\mathsf{filter}}
\newcommand*{\Lete}{\mathtt{val}}
\newcommand*{\Letm}{\mathtt{mod}}

\newcommand*{\ValRel}[1]{\mathcal{V}\sembracket{#1}}
\newcommand*{\ExprRel}[1]{\mathcal{E}\sembracket{#1}}
\newcommand*{\CtxRel}[1]{\mathcal{C}\sembracket{#1}}
\newcommand*{\TyEnv}{\text{TyEnv}}
\newcommand*{\TyVar}{\text{TyVar}}
\newcommand*{\Type}{\text{Type}}
\newcommand*{\Subst}{\text{Subst}}
\newcommand*{\external}{\Gamma_{\text{ext}}}

\title{Type Inference as an Instance for Modular Analysis}
\author{Joonhyup Lee}
\begin{document}
\maketitle
\section{For the Simple Module Language}
\begin{figure}[htb]
  \centering
  \begin{tabular}{rrcll}
    Identifiers & $x,\modid$ & $\in$         & $\ExprVar$                                                                            \\
    Expression  & $e$        & $\rightarrow$ & $()$ $\vbar$ $x$ $\vbar$ $\lambda x.e$ $\vbar$ $e$ $e$ & $\lambda$-calculus with unit \\
                &            & $\vbar$       & $\link{m}{e}$                                          & linked expression            \\
    Module      & $m$        & $\rightarrow$ & $\varepsilon$                                          & empty module                 \\
                &            & $\vbar$       & $\modid$                                               & module identifier            \\
                &            & $\vbar$       & $\Lete$ $x$ $e$ $m$                                    & value binding                \\
                &            & $\vbar$       & $\Letm$ $\modid$ $m$ $m$                               & module binding               \\
  \end{tabular}
  \caption{Abstract syntax of the simple module language.}
  \label{fig:syntax}
\end{figure}
\subsection{Operational Semantics}
\begin{figure}[h!]
  \centering
  \begin{tabular}{rrcll}
    Environment/Context  & $\ctx$ & $\in$         & $\Ctx$                                                                   \\
    Value of expressions & $v$    & $\in$         & $\Value \triangleq\{()\}+\ExprVar\times\Expr\times\Ctx$                  \\
    Context              & $\ctx$ & $\rightarrow$ & $\bullet$                                               & empty stack    \\
                         &        & $\vbar$       & $(x,v)\cons \ctx$                                       & value binding  \\
                         &        & $\vbar$       & $(\modid,\ctx)\cons \ctx$                               & module binding \\
    Value of expressions & $v$    & $\rightarrow$ & $()$                                                    & unit           \\
                         &        & $\vbar$       & $\langle \lambda x.e, \ctx \rangle$                     & closure
  \end{tabular}
  \caption{Definition of the semantic domains.}
  \label{fig:simpdom}
\end{figure}

\begin{figure}[h!]
  \footnotesize
  \begin{flushright}
    \fbox{$(e,\ctx)\semarrow v$ and $(m,\ctx)\semarrow\ctx$}
  \end{flushright}
  \centering
  \vspace{0pt} % -0.75em}
  \[
    \begin{prooftree}
      \infer[left label=Unit]0{
      ((),\ctx)
      \semarrow
      ()
      }
    \end{prooftree}\qquad
    \begin{prooftree}
      \hypo{v=\ctx(x)}
      \infer[left label=ExprID]1{
      (x, \ctx)
      \semarrow
      v
      }
    \end{prooftree}\qquad
    \begin{prooftree}
      \infer[left label=Fn]0{
      (\lambda x.e, \ctx)
      \semarrow
      \langle\lambda x.e, \ctx\rangle
      }
    \end{prooftree}\qquad
    \begin{prooftree}
      \hypo{
        \begin{matrix}
          (e_{1}, \ctx)
          \semarrow
          \langle\lambda x.e_{\lambda}, \ctx_{\lambda}\rangle \\
          (e_{2}, \ctx)
          \semarrow
          v                                                   \\
          (e_{\lambda}, (x, v)\cons \ctx_{\lambda})
          \semarrow
          v'
        \end{matrix}
      }
      \infer[left label={App}]1{
      (e_{1}\:e_{2}, \ctx)
      \semarrow
      v'
      }
    \end{prooftree}\qquad
    \begin{prooftree}
      \hypo{
        \begin{matrix}
          (m_{1}, \ctx)
          \semarrow
          \ctx' \\
          (e_{2}, \ctx')
          \semarrow
          v
        \end{matrix}
      }
      \infer[left label=Link]1{
      (\link{m_{1}}{e_{2}}, \ctx)
      \semarrow
      v
      }
    \end{prooftree}
  \]

  \[
    \begin{prooftree}
      \infer[left label=Empty]0{
      (\varepsilon, \ctx)
      \semarrow
      \bullet
      }
    \end{prooftree}\qquad
    \begin{prooftree}
      \hypo{\ctx'=\ctx(\modid)}
      \infer[left label=ModID]1{
      (\modid, \ctx)
      \semarrow
      \ctx'
      }
    \end{prooftree}\qquad
    \begin{prooftree}
      \hypo{
        \begin{matrix}
          (e_{1}, \ctx)
          \semarrow
          v \\
          (m_{2}, (x, v)\cons \ctx)
          \semarrow
          \ctx'
        \end{matrix}
      }
      \infer[left label=LetE]1{
      (\Lete\:x\:e_1\:m_2, \ctx)
      \semarrow
      (x,v)\cons\ctx'
      }
    \end{prooftree}\qquad
    \begin{prooftree}
      \hypo{
        \begin{matrix}
          (m_{1}, \ctx)
          \semarrow
          \ctx' \\
          (m_{2}, (\modid, \ctx')\cons \ctx)
          \semarrow
          \ctx''
        \end{matrix}
      }
      \infer[left label=LetM]1{
      (\Letm\:\modid\:m_{1}\:m_{2}, \ctx)
      \semarrow
      (\modid,\ctx')\cons\ctx''
      }
    \end{prooftree}
  \]
  \caption{The big-step operational semantics.}
  \label{fig:simpreach}
\end{figure}
\subsection{Typing}
The definitions for types are in Figure \ref{fig:typdom} and the typing rules are in Figure \ref{fig:typjudge}.

\begin{figure}[h!]
  \centering
  \begin{tabular}{rrcll}
    Types            & $\tau$   & $\rightarrow$ & $\iota$                       & unit type         \\
                     &          & $\vbar$       & $\tau\rightarrow\tau$         & function type     \\
    Type Environment & $\Gamma$ & $\rightarrow$ & $\bullet$                     & empty environment \\
                     &          & $\vbar$       & $(x,\tau)\cons \Gamma$        & value binding     \\
                     &          & $\vbar$       & $(\modid,\Gamma)\cons \Gamma$ & module binding
  \end{tabular}
  \caption{Definition of types.}
  \label{fig:typdom}
\end{figure}

\begin{figure}[h!]
  \footnotesize
  \begin{flushright}
    \fbox{$\Gamma\vdash e:\tau$ and $\Gamma\vdash m:\Gamma$}
  \end{flushright}
  \centering
  \vspace{0pt} % -0.75em}
  \[
    \begin{prooftree}
      \infer[left label=Unit]0{
      \Gamma\vdash():\iota
      }
    \end{prooftree}\qquad
    \begin{prooftree}
      \hypo{\tau=\Gamma(x)}
      \infer[left label=ExprID]1{
      \Gamma\vdash x:\tau
      }
    \end{prooftree}\qquad
    \begin{prooftree}
      \hypo{(x,\tau_1)\cons\Gamma\vdash e:\tau_2}
      \infer[left label=Fn]1{
      \Gamma\vdash\lambda x.e:\tau_1\rightarrow\tau_2
      }
    \end{prooftree}\qquad
    \begin{prooftree}
      \hypo{
        \begin{matrix}
          \Gamma\vdash e_1:\tau'\rightarrow\tau \\
          \Gamma\vdash e_2:\tau'
        \end{matrix}
      }
      \infer[left label={App}]1{
      \Gamma\vdash e_{1}\:e_{2}:\tau
      }
    \end{prooftree}\qquad
    \begin{prooftree}
      \hypo{
        \begin{matrix}
          \Gamma\vdash m_1:\Gamma_1 \\
          \Gamma_1\vdash e_2:\tau_2
        \end{matrix}
      }
      \infer[left label=Link]1{
      \Gamma\vdash\link{m_{1}}{e_{2}}:\tau_2
      }
    \end{prooftree}
  \]

  \[
    \begin{prooftree}
      \infer[left label=Empty]0{
      \Gamma\vdash\varepsilon:\bullet
      }
    \end{prooftree}\qquad
    \begin{prooftree}
      \hypo{\Gamma'=\Gamma(\modid)}
      \infer[left label=ModID]1{
      \Gamma\vdash\modid:\Gamma'
      }
    \end{prooftree}\qquad
    \begin{prooftree}
      \hypo{
        \begin{matrix}
          \Gamma\vdash e_1:\tau_1 \\
          (x, \tau_1)\cons\Gamma\vdash m_2:\Gamma_2
        \end{matrix}
      }
      \infer[left label=LetE]1{
      \Gamma\vdash\Lete\:x\:e_1\:m_2:(x,\tau_1)\cons\Gamma_2
      }
    \end{prooftree}\qquad
    \begin{prooftree}
      \hypo{
        \begin{matrix}
          \Gamma\vdash m_1:\Gamma_1 \\
          (\modid, \Gamma_1)\cons\Gamma\vdash m_2:\Gamma_2
        \end{matrix}
      }
      \infer[left label=LetM]1{
      \Gamma\vdash\Letm\:\modid\:m_{1}\:m_{2}:(\modid,\Gamma_1)\cons\Gamma_2
      }
    \end{prooftree}
  \]
  \caption{The typing judgment.}
  \label{fig:typjudge}
\end{figure}

\subsection{Type Safety}
\begin{clm}[Type Safety]
  For all $e\in\Expr$, if $\bullet\vdash e:\tau$ for some $\tau$, then there exists some $v\in\Value$ such that $(e,\bullet)\semarrow v$.
  Likewise, if $\bullet\vdash m:\Gamma$ for some $\Gamma$, then there exists some $\ctx\in\Ctx$ such that $(m,\bullet)\Downarrow\ctx$.
\end{clm}
\begin{proof}[Proof sketch]
  We prove this through unary logical relations and induction on the typing judgment.

  \begin{tabular}{rclr}
    \textbf{Value Relation}                   &              &                                                                                                        & \fbox{$\ValRel{\tau}$}                                   \\
    $\ValRel{\iota}$                          & $\triangleq$ & $\{()\}$                                                                                                                                                          \\
    $\ValRel{\tau_1\rightarrow\tau_2}$        & $\triangleq$ & $\{\langle\lambda x.e,\ctx\rangle|\forall v\in\ValRel{\tau_1}:(e,(x,v)\cons\ctx)\in\ExprRel{\tau_2}\}$                                                            \\
    \\
    \textbf{Expression Relation}              &              &                                                                                                        & \fbox{$\ExprRel{\tau}$ and $\ExprRel{\Gamma}$}           \\
    $\ExprRel{\tau}$                          & $\triangleq$ & $\{(e,\ctx)|\exists v\in\ValRel{\tau}:(e,\ctx)\semarrow v\}$                                                                                                      \\
    $\ExprRel{\Gamma}$                        & $\triangleq$ & $\{(m,\ctx)|\exists \ctx'\in\CtxRel{\Gamma}:(m,\ctx)\semarrow \ctx'\}$                                                                                            \\
    \\
    \textbf{Context Relation}                 &              &                                                                                                        & \fbox{$\CtxRel{\Gamma}$}                                 \\
    $\CtxRel{\bullet}$                        & $\triangleq$ & $\{\bullet\}$                                                                                                                                                     \\
    $\CtxRel{(x,\tau_1)\cons\Gamma_2}$        & $\triangleq$ & $\{(x,v_1)\cons\ctx_2|v_1\in\ValRel{\tau_1}\land\ctx_2\in\CtxRel{\Gamma_2}\}$                                                                                     \\
    $\CtxRel{(\modid,\Gamma_1)\cons\Gamma_2}$ & $\triangleq$ & $\{(\modid,\ctx_1)\cons\ctx_2|\ctx_1\in\CtxRel{\Gamma_1}\land\ctx_2\in\CtxRel{\Gamma_2}\}$                                                                        \\
    \\
    \textbf{Semantic Typing}                  &              &                                                                                                        & \fbox{$\Gamma\vDash e:\tau$ and $\Gamma\vDash m:\Gamma$} \\
    $\Gamma\vDash e:\tau$                     & $\triangleq$ & $\forall\ctx\in\CtxRel{\Gamma}:(e,\ctx)\in\ExprRel{\tau}$                                                                                                         \\
    $\Gamma\vDash m:\Gamma'$                  & $\triangleq$ & $\forall\ctx\in\CtxRel{\Gamma}:(m,\ctx)\in\ExprRel{\Gamma'}$
  \end{tabular}

  \vphantom{}

  We want to prove that:
  \[\Gamma\vdash e:\tau\Rightarrow\Gamma\vDash e:\tau\]
  \[\Gamma\vdash m:\Gamma'\Rightarrow\Gamma\vDash m:\Gamma'\]
  by induction on $\vdash$.

  For the base cases of $\iota$ and $\bullet$, the proof is trivial.
  For inductive cases, we need to show \emph{compatibility} lemmas.
  That is, we must show that the typing rules for syntactic typing hold for semantic typing as well.
  Then by the inductive hypothesis and compatibility, the result follows.
\end{proof}

\subsection{Type Inference}

For the simple module language, the operational semantics constrains which expressions \emph{must} evaluate to contexts and which expressions \emph{must} evaluate to closures.
Therefore, type inference in this language is simple, as the shape of the typing environment can be accurately inferred from the syntax of the program.

First we define the syntax for type constraints.
\begin{figure}[h!]
  \centering
  \begin{tabular}{rrcll}
    Type Variable      & $\alpha$ & $\in$         & $\TyVar$                                                                 \\
    Module Path        & $p$      & $\rightarrow$ & $\epsilon$                       & empty string                          \\
                       &          & $\vbar$       & $p\modid$                        & concatenation with module identifier  \\
    Types              & $\tau$   & $\rightarrow$ & $\iota$ | $\tau\rightarrow\tau$  & unit/function types                   \\
                       &          & $\vbar$       & $\alpha$                         & type variables                        \\
                       &          & $\vbar$       & $[].p.x$                         & types from the external environment   \\
    Type Environment   & $\Gamma$ & $\rightarrow$ & $\bullet$                        & empty environment                     \\
                       &          & $\vbar$       & $(x,\tau)\cons \Gamma$           & value binding                         \\
                       &          & $\vbar$       & $(\modid,\Gamma)\cons \Gamma$    & module binding                        \\
                       &          & $\vbar$       & $[].p$                           & modules from the external environment \\
    Type Constraint    & $u$      & $\rightarrow$ & $\tau\A{=}\tau$                  & equality constraint                   \\
    Set of Constraints & $U$      & $\subseteq$   & $\{u|u\text{ type constraint}\}$
  \end{tabular}
  \caption{Definition of type constraints.}
  \label{fig:typeqdom}
\end{figure}

Next we define the module access operation $\Gamma(\modid)$ and the type access operation $\Gamma(x)$:
\begin{align*}
  \bullet(\modid)                   & \triangleq\bot           &            &  & \bullet(x)                  & \triangleq\bot                   \\
  ((x,\_)\cons\Gamma)(\modid)       & \triangleq\Gamma(\modid) &            &  & ((\modid,\_)\cons\Gamma)(x) & \triangleq\Gamma(x)              \\
  ((\modid,\Gamma)\cons\_)(\modid)  & \triangleq\Gamma         &            &  & ((x,\tau)\cons\_)(x)        & \triangleq\tau                   \\
  ((\modid',\_)\cons\Gamma)(\modid) & \triangleq\Gamma(\modid) & (d'\neq d) &  & ((x',\_)\cons\Gamma)(x)     & \triangleq\Gamma(x) & (x'\neq x) \\
  ([].p)(\modid)                    & \triangleq[].pd          &            &  & ([].p)(x)                   & \triangleq[].p.x
\end{align*}

Now we can define the constraint generation algorithms $V_1(\Gamma,e,\alpha)$ and $V_2(\Gamma,m)$.
Note that the \textbf{let} $U$ \texttt{=} $\_$ \textbf{in} $\_$ notation returns $\bot$ if the right hand side is $\bot$.
Likewise, the \textbf{let} $(\Gamma,U)$ \texttt{=} $\_$ \textbf{in} $\_$ notation returns $\bot$ if the right hand side is $\bot$.

\begin{flushright}\fbox{$V_1(\Gamma,e,\alpha)=U$ and $V_2(\Gamma,m)=(\Gamma,U)$}\end{flushright}
\hspace{-2.5em}
\begin{tabular}{rclrcl}
  $V_1(\Gamma,(),\alpha)$              & $\triangleq$ & $\{\alpha\A{=}\iota\}$                                                            & $V_2(\Gamma,\varepsilon)$             & $\triangleq$ & $(\bullet,\varnothing)$                                                                      \\
  $V_1(\Gamma,x,\alpha)$               & $\triangleq$ & \textbf{let} $\tau$ \texttt{=} $\Gamma(x)$ \textbf{in}                            & $V_2(\Gamma,\modid)$                  & $\triangleq$ & \textbf{let} $\Gamma'$ \texttt{=} $\Gamma(d)$ \textbf{in}                                    \\
                                       &              & $\{\alpha\A{=}\tau\}$                                                             &                                       &              & $(\Gamma',\varnothing)$                                                                      \\
  $V_1(\Gamma,\lambda x.e,\alpha)$     & $\triangleq$ & \textbf{let} $\alpha_1,\alpha_2$ \texttt{=} \textit{fresh} \textbf{in}            & $V_2(\Gamma,\Lete\:x\:e_1\:m_2)$      & $\triangleq$ & \textbf{let} $\alpha_1$ \texttt{=} \textit{fresh} \textbf{in}                                \\
                                       &              & \textbf{let} $U$ \texttt{=} $V_1((x,\alpha_1)\cons\Gamma,e,\alpha_2)$ \textbf{in} &                                       &              & \textbf{let} $U_1$ \texttt{=} $V_1(\Gamma,e_1,\alpha_1)$ \textbf{in}                         \\
                                       &              & $\{\alpha\A{=}\alpha_1\rightarrow\alpha_2\}\cup U$                                &                                       &              & \textbf{let} $(\Gamma_2,U_2)$ \texttt{=} $V_2((x,\alpha_1)\cons\Gamma,m_2)$ \textbf{in}      \\
  $V_1(\Gamma,e_1\:e_2,\alpha)$        & $\triangleq$ & \textbf{let} $\alpha_1,\alpha_2$ \texttt{=} \textit{fresh} \textbf{in}            &                                       &              & $((x,\alpha_1)\cons\Gamma_2,U_1\cup U_2)$                                                    \\
                                       &              & \textbf{let} $U_1$ \texttt{=} $V_1(\Gamma,e_1,\alpha_1)$ \textbf{in}              & $V_2(\Gamma,\Letm\:\modid\:m_1\:m_2)$ & $\triangleq$ & \textbf{let} $(\Gamma_1,U_1)$ \texttt{=} $V_2(\Gamma,m_1)$ \textbf{in}                       \\
                                       &              & \textbf{let} $U_2$ \texttt{=} $V_1(\Gamma,e_2,\alpha_2)$ \textbf{in}              &                                       &              & \textbf{let} $(\Gamma_2,U_2)$ \texttt{=} $V_2((\modid,\Gamma_1)\cons\Gamma,m_2)$ \textbf{in} \\
                                       &              & $\{\alpha_1\A{=}\alpha_2\rightarrow\alpha\}\cup U_1\cup U_2$                      &                                       &              & $((\modid,\Gamma_1)\cons\Gamma_2,U_1\cup U_2)$                                               \\
  $V_1(\Gamma,\link{m_1}{e_2},\alpha)$ & $\triangleq$ & \textbf{let} $(\Gamma_1,U_1)$ \texttt{=} $V_2(\Gamma,m_1)$ \textbf{in}                                                                                                                                                                  \\
                                       &              & \textbf{let} $U_2$ \texttt{=} $V_1(\Gamma_1,e_2,\alpha)$ \textbf{in}                                                                                                                                                                    \\
                                       &              & $U_1\cup U_2$                                                                                                                                                                                                                           \\
\end{tabular}

\vphantom{}

We want to prove that the constraint generation algorithm is correct.

First, for $\external\in\TyEnv$, define the access operations $\external.p$ and $\external.p.x$ (which may fail):
\begin{align*}
  \external.\epsilon & \triangleq\external &  &  & \external.p\modid & \triangleq(\external.p)(\modid) &  &  & \external.p.x & \triangleq(\external.p)(x)
\end{align*}
and define the injection operations $\Gamma[\external]$ and $\tau[\external]$:
\begin{align*}
  (\bullet)[\external]                     & \triangleq\bullet                                           &  &  & ((x,\tau)\cons\Gamma)[\external]     & \triangleq(x,\tau[\external])\cons\Gamma[\external]     \\
  ((\modid,\Gamma)\cons\Gamma')[\external] & \triangleq(\modid,\Gamma[\external])\cons\Gamma'[\external] &  &  & ([].p)[\external]                    & \triangleq\external.p                                   \\
  (\iota)[\external]                       & \triangleq\iota                                             &  &  & (\tau_1\rightarrow\tau_2)[\external] & \triangleq\tau_1[\external]\rightarrow\tau_2[\external] \\
  (\alpha)[\external]                      & \triangleq\alpha                                            &  &  & ([].p.x)[\external]                  & \triangleq\external.p.x
\end{align*}

Let $\Subst\triangleq\fin{\TyVar}{\Type}$ be the set of substitutions.
For $S\in\Subst$, define:
\begin{align*}
  S\iota               & \triangleq\iota                  &                                  &  & S(\tau_1\rightarrow\tau_2)   & \triangleq S\tau_1\rightarrow S\tau_2                                         \\
  S\alpha              & \triangleq\alpha                 & \text{when }\alpha\not\in dom(S) &  & S\alpha                      & \triangleq \tau                          & \text{when }\alpha\mapsto\tau\in S \\
  S[].\_               & \triangleq [].\_                 &                                  &  & S\bullet                     & \triangleq\bullet                                                             \\
  S(x,\tau)\cons\Gamma & \triangleq(x,S\tau)\cons S\Gamma &                                  &  & S(\modid,\Gamma)\cons\Gamma' & \triangleq(\modid,S\Gamma)\cons S\Gamma'
\end{align*}
Define:
\[(S,\external)\vDash U\triangleq\forall(\tau_1\A{=}\tau_2)\in U:(S\tau_1)[\external]= (S\tau_2)[\external]\]
Then we can show that:
\begin{clm}[Correnctness of $V$]
  For $e\in\Expr$, $m\in\Module$, $\Gamma,\external\in\TyEnv$, $\alpha\in\TyVar$, $S\in\Subst$:
  \begin{align*}
    (S,\external)\vDash U \Leftrightarrow & (S\Gamma)[\external]\vdash e:(S\alpha)[\external]  & \text{when }V_1(\Gamma,e,\alpha)=U    \\
    (S,\external)\vDash U \Leftrightarrow & (S\Gamma)[\external]\vdash m:(S\Gamma')[\external] & \text{when }V_2(\Gamma,m)=(\Gamma',U)
  \end{align*}
\end{clm}
\begin{proof}[Proof sketch]
  Mutual induction on $e,m$.
\end{proof}

Note that by including $[].p$ in type environments, we can naturally generate constraints about the external environment $[]$.
Also, by injection, we can utilize constraints generated \emph{in advance} to obtain constraints generated from a more informed environment.
We extend injection to the output of the constraint-generating algorithm:
\begin{align*}
  \bot[\external]       & \triangleq\bot                                                                                                      \\
  U[\external]          & \triangleq\{\tau_1[\external]\A{=}\tau_2[\external]|(\tau_1\A{=}\tau_2)\in U\} & \text{when all injections succeed} \\
  U[\external]          & \triangleq\bot                                                                 & \text{when injection fails}        \\
  (\Gamma,U)[\external] & \triangleq(\Gamma[\external],U[\external])
\end{align*}
Then we can prove:
\begin{clm}[Advance]
  For $e\in\Expr$, $m\in\Module$, $\Gamma,\external\in\TyEnv$, $\alpha\in\TyVar$:
  \begin{align*}
    V_1(\Gamma[\external],e,\alpha) & =V_1(\Gamma,e,\alpha)[\external] \\
    V_2(\Gamma[\external],m)        & =V_2(\Gamma,m)[\external]
  \end{align*}
\end{clm}
\begin{proof}[Proof sketch]
  Structural induction on $\Gamma$.
\end{proof}

\section{For the Language with First-Class Modules}
\begin{figure}[htb]
  \centering
  \begin{tabular}{rrcll}
    Identifiers & $x$ & $\in$         & $\ExprVar$                                                     \\
    Expression  & $e$ & $\rightarrow$ & $x$ $\vbar$ $\lambda x.e$ $\vbar$ $e$ $e$ & $\lambda$-calculus \\
                &     & $\vbar$       & $\link{e}{e}$                             & linked expression  \\
                &     & $\vbar$       & $\varepsilon$                             & empty module       \\
                &     & $\vbar$       & $\Lete$ $x$ $e$ $e$                       & value binding
  \end{tabular}
  \caption{Abstract syntax of the language where modules are first-class.}
  \label{fig:extsyntax}
\end{figure}
\subsection{Operational Semantics}
\begin{figure}[h!]
  \centering
  \begin{tabular}{rrcll}
    Environment/Context & $\ctx$ & $\in$         & $\Ctx$                                                                   \\
    Value               & $v$    & $\in$         & $\Value \triangleq\Ctx+\ExprVar\times\Expr\times\Ctx$                    \\
    Context             & $\ctx$ & $\rightarrow$ & $\bullet$                                             & empty stack      \\
                        &        & $\vbar$       & $(x,v)\cons \ctx$                                     & value binding    \\
    Value               & $v$    & $\rightarrow$ & $\ctx$                                                & exported context \\
                        &        & $\vbar$       & $\langle \lambda x.e, \ctx \rangle$                   & closure
  \end{tabular}
  \caption{Definition of the semantic domains.}
  \label{fig:extsimpdom}
\end{figure}

\begin{figure}[h!]
  \footnotesize
  \begin{flushright}
    \fbox{$(e,\ctx)\semarrow v$}
  \end{flushright}
  \centering
  \vspace{0pt} % -0.75em}
  \[
    \begin{prooftree}
      \hypo{v=\ctx(x)}
      \infer[left label=Id]1{
      (x, \ctx)
      \semarrow
      v
      }
    \end{prooftree}\qquad
    \begin{prooftree}
      \infer[left label=Fn]0{
      (\lambda x.e, \ctx)
      \semarrow
      \langle\lambda x.e, \ctx\rangle
      }
    \end{prooftree}\qquad
    \begin{prooftree}
      \hypo{
        \begin{matrix}
          (e_{1}, \ctx)
          \semarrow
          \langle\lambda x.e_{\lambda}, \ctx_{\lambda}\rangle \\
          (e_{2}, \ctx)
          \semarrow
          v                                                   \\
          (e_{\lambda}, (x, v)\cons \ctx_{\lambda})
          \semarrow
          v'
        \end{matrix}
      }
      \infer[left label={App}]1{
      (e_{1}\:e_{2}, \ctx)
      \semarrow
      v'
      }
    \end{prooftree}\qquad
    \begin{prooftree}
      \hypo{
        \begin{matrix}
          (e_{1}, \ctx)
          \semarrow
          \ctx' \\
          (e_{2}, \ctx')
          \semarrow
          v
        \end{matrix}
      }
      \infer[left label=Link]1{
      (\link{e_{1}}{e_{2}}, \ctx)
      \semarrow
      v
      }
    \end{prooftree}
  \]

  \[
    \begin{prooftree}
      \infer[left label=Empty]0{
      (\varepsilon, \ctx)
      \semarrow
      \bullet
      }
    \end{prooftree}\qquad
    \begin{prooftree}
      \hypo{
        \begin{matrix}
          (e_{1}, \ctx)
          \semarrow
          v \\
          (e_{2}, (x, v)\cons \ctx)
          \semarrow
          \ctx'
        \end{matrix}
      }
      \infer[left label=Bind]1{
      (\Lete\:x\:e_1\:e_2, \ctx)
      \semarrow
      (x,v)\cons\ctx'
      }
    \end{prooftree}
  \]
  \caption{The big-step operational semantics.}
  \label{fig:extsimpreach}
\end{figure}
\subsection{Typing}
The definitions for types are in Figure \ref{fig:exttypdom} and the typing rules are in Figure \ref{fig:exttypjudge}.
The definitions for subtyping are in Figure \ref{fig:subtyping}.

\begin{figure}[h!]
  \centering
  \begin{tabular}{rrcll}
    Types              & $\tau$   & $\rightarrow$ & $\Gamma$               & module type       \\
                       &          & $\vbar$       & $\tau\rightarrow\tau$  & function type     \\
    Typing Environment & $\Gamma$ & $\rightarrow$ & $\bullet$              & empty environment \\
                       &          & $\vbar$       & $(x,\tau)\cons \Gamma$ & type binding
  \end{tabular}
  \caption{Definition of types.}
  \label{fig:exttypdom}
\end{figure}

\begin{figure}[h!]
  \footnotesize
  \begin{flushright}
    \fbox{$\Gamma\vdash e:\tau$}
  \end{flushright}
  \centering
  \vspace{0pt} % -0.75em}
  \[
    \begin{prooftree}
      \hypo{\tau=\Gamma(x)}
      \infer[left label=Id]1{
      \Gamma\vdash x:\tau
      }
    \end{prooftree}\qquad
    \begin{prooftree}
      \hypo{(x,\tau_1)\cons\Gamma\vdash e:\tau_2}
      \infer[left label=Fn]1{
      \Gamma\vdash\lambda x.e:\tau_1\rightarrow\tau_2
      }
    \end{prooftree}\qquad
    \begin{prooftree}
      \hypo{
        \begin{matrix}
          \Gamma\vdash e_1:\tau_1\rightarrow\tau \\
          \Gamma\vdash e_2:\tau_2                \\
          \tau_1\le\tau_2
        \end{matrix}
      }
      \infer[left label={App}]1{
      \Gamma\vdash e_{1}\:e_{2}:\tau
      }
    \end{prooftree}\qquad
    \begin{prooftree}
      \hypo{
        \begin{matrix}
          \Gamma\vdash e_1:\Gamma_1 \\
          \Gamma_1\vdash e_2:\tau_2
        \end{matrix}
      }
      \infer[left label=Link]1{
      \Gamma\vdash\link{e_{1}}{e_{2}}:\tau_2
      }
    \end{prooftree}
  \]

  \[
    \begin{prooftree}
      \infer[left label=Empty]0{
      \Gamma\vdash\varepsilon:\bullet
      }
    \end{prooftree}\qquad
    \begin{prooftree}
      \hypo{
        \begin{matrix}
          \Gamma\vdash e_1:\tau_1 \\
          (x, \tau_1)\cons\Gamma\vdash e_2:\Gamma_2
        \end{matrix}
      }
      \infer[left label=Bind]1{
      \Gamma\vdash\Lete\:x\:e_1\:e_2:(x,\tau_1)\cons\Gamma_2
      }
    \end{prooftree}
  \]
  \caption{The typing judgment.}
  \label{fig:exttypjudge}
\end{figure}
\begin{figure}[h!]
  \footnotesize
  \begin{flushright}
    \fbox{$\tau\le\tau$}
  \end{flushright}
  \centering
  \vspace{0pt} % -0.75em}
  \[
    \begin{prooftree}
      \infer[left label=Empty]0{
      \bullet\le\bullet
      }
    \end{prooftree}\qquad
    \begin{prooftree}
      \hypo{
        \begin{matrix}
          \Gamma(x)\le\tau \\
          \Gamma-x\le\Gamma'
        \end{matrix}
      }
      \infer[left label=Bind]1{
      \Gamma\le(x,\tau)\cons\Gamma'
      }
    \end{prooftree}\qquad
    \begin{prooftree}
      \hypo{
        \begin{matrix}
          \tau_2\le\tau_1 \\
          \tau_1'\le\tau_2'
        \end{matrix}
      }
      \infer[left label=Fn]1{
      \tau_1\rightarrow\tau_1'\le
      \tau_2\rightarrow\tau_2'
      }
    \end{prooftree}
  \]
  \caption{The subtype relation.}
  \label{fig:subtyping}
\end{figure}

\subsection{Type Safety}
\begin{clm}[Type Safety]
  For all $e\in\Expr$, if $\bullet\vdash e:\tau$ for some $\tau$, then there exists some $v\in\Value$ such that $(e,\bullet)\semarrow v$.
\end{clm}
\begin{proof}[Proof sketch]
  We prove this through unary logical relations and induction on the typing judgment.

  \begin{tabular}{rclr}
    \textbf{Value Relation}            &              &                                                                                                        & \fbox{$\ValRel{\tau}$}       \\
    $\ValRel{\bullet}$                 & $\triangleq$ & $\Ctx$                                                                                                                                \\
    $\ValRel{(x,\tau)\cons\Gamma}$     & $\triangleq$ & $\{\ctx|\ctx(x)\in\ValRel{\tau}\land(\ctx-x)\in\ValRel{\Gamma-x}\}$                                                                   \\
    $\ValRel{\tau_1\rightarrow\tau_2}$ & $\triangleq$ & $\{\langle\lambda x.e,\ctx\rangle|\forall v\in\ValRel{\tau_1}:(e,(x,v)\cons\ctx)\in\ExprRel{\tau_2}\}$                                \\
    \\
    \textbf{Expression Relation}       &              &                                                                                                        & \fbox{$\ExprRel{\tau}$}      \\
    $\ExprRel{\tau}$                   & $\triangleq$ & $\{(e,\ctx)|\exists v\in\ValRel{\tau}:(e,\ctx)\semarrow v\}$                                                                          \\
    \\
    \textbf{Semantic Typing}           &              &                                                                                                        & \fbox{$\Gamma\vDash e:\tau$} \\
    $\Gamma\vDash e:\tau$              & $\triangleq$ & $\forall\ctx\in\ValRel{\Gamma}:(e,\ctx)\in\ExprRel{\tau}$
  \end{tabular}

  \vphantom{}

  We want to prove that:
  \[\Gamma\vdash e:\tau\Rightarrow\Gamma\vDash e:\tau\]
  by induction on $\vdash$.

  For the base cases of $\bullet$, the proof is trivial.
  For inductive cases, we need to show \emph{compatibility} lemmas.
  That is, we must show that the typing rules for syntactic typing hold for semantic typing as well.
  For this, we need the \emph{subtyping} lemma:
  \[\tau_1\le\tau_2\Rightarrow\ValRel{\tau_2}\subseteq\ValRel{\tau_1}\]
  Then by the inductive hypothesis and compatibility, the result follows.
\end{proof}

\subsection{Type Inference}

When modules are first-class, type variables can go in the place of type environments.

First we define the syntax for type constraints.
\begin{figure}[h!]
  \centering
  \begin{tabular}{rrcll}
    Type Variable      & $\alpha$ & $\in$         & $\TyVar$                                                              \\
    Path               & $p$      & $\rightarrow$ & $\epsilon$                       & empty string                       \\
                       &          & $\vbar$       & $p x$                            & concatenation with identifier      \\
    Types              & $\tau$   & $\rightarrow$ & $\Gamma$ | $\tau\rightarrow\tau$ & module/function types              \\
    Type Environment   & $\Gamma$ & $\rightarrow$ & $\bullet$                        & empty environment                  \\
                       &          & $\vbar$       & $(x,\tau)\cons \Gamma$           & binding                            \\
                       &          & $\vbar$       & $\alpha.p$                       & type variable                      \\
                       &          & $\vbar$       & $[].p$                           & type from the external environment \\
    Type Constraint    & $u$      & $\rightarrow$ & $\tau\A{=}\tau$                  & equality constraint                \\
                       &          & $\vbar$       & $\tau\A{\le}\tau$                & subtyping constraint               \\
    Set of Constraints & $U$      & $\subseteq$   & $\{u|u\text{ type constraint}\}$
  \end{tabular}
  \caption{Definition of type constraints.}
  \label{fig:typeqdom}
\end{figure}

Next we define the type access operation $\tau(x)$:
\begin{align*}
  \bullet(x)              & \triangleq\bot      &                      &  & (\alpha.p)(x)        & \triangleq\alpha.px \\
  ((x,\tau)\cons\_)(x)    & \triangleq\tau      &                      &  & ([].p)(x)            & \triangleq[].px     \\
  ((x',\_)\cons\Gamma)(x) & \triangleq\Gamma(x) & \text{when }x'\neq x &  & (\_\rightarrow\_)(x) & \triangleq\bot
\end{align*}

Now we can define the constraint generation algorithm $V(\Gamma,e,\alpha)$.
Note that the \textbf{let} $U$ \texttt{=} $\_$ \textbf{in} $\_$ notation returns $\bot$ if the right hand side is not a set of type constraints.
Also note that we write $\alpha$ for $\alpha.\epsilon$ as well.

\begin{flushright}\fbox{$V(\Gamma,e,\alpha)=U\text{ or }\bot$}\end{flushright}
\hspace{-2.5em}
\begin{tabular}{rclrcl}
  $V(\Gamma,\varepsilon,\alpha)$ & $\triangleq$ & $\{\alpha\A{=}\bullet\}$                                                             & $V(\Gamma,\link{e_1}{e_2},\alpha)$         & $\triangleq$ & \textbf{let} $\alpha_1$ \texttt{=} \textit{fresh} \textbf{in}                       \\
  $V(\Gamma,x,\alpha)$           & $\triangleq$ & \textbf{let} $\tau$ \texttt{=} $\Gamma(x)$ \textbf{in}                               &                                            &              & \textbf{let} $U_1$ \texttt{=} $V(\Gamma,e_1,\alpha_1)$ \textbf{in}                  \\
                                 &              & $\{\alpha\A{=}\tau\}$                                                                &                                            &              & \textbf{let} $U_2$ \texttt{=} $V(\alpha_1,e_2,\alpha)$  \textbf{in}                 \\
  $V(\Gamma,\lambda x.e,\alpha)$ & $\triangleq$ & \textbf{let} $\alpha_1,\alpha_2$ \texttt{=} \textit{fresh} \textbf{in}               &                                            &              & $U_1\cup U_2$                                                                       \\
                                 &              & \textbf{let} $U$ \texttt{=} $V((x,\alpha_1)\cons\Gamma,e,\alpha_2)$ \textbf{in}      & $V(\Gamma,\Lete\:\modid\:e_1\:e_2,\alpha)$ & $\triangleq$ & \textbf{let} $\alpha_1,\alpha_2$ \texttt{=} \textit{fresh} \textbf{in}              \\
                                 &              & $\{\alpha\A{=}\alpha_1\rightarrow\alpha_2\}\cup U$                                   &                                            &              & \textbf{let} $U_1$ \texttt{=} $V(\Gamma,e_1,\alpha_1)$ \textbf{in}                  \\
  $V(\Gamma,e_1\:e_2,\alpha)$    & $\triangleq$ & \textbf{let} $\alpha_1,\alpha_2,\alpha_3$ \texttt{=} \textit{fresh} \textbf{in}      &                                            &              & \textbf{let} $U_2$ \texttt{=} $V((x,\alpha_1)\cons\Gamma,e_2,\alpha_2)$ \textbf{in} \\
                                 &              & \textbf{let} $U_1$ \texttt{=} $V(\Gamma,e_1,\alpha_1)$ \textbf{in}                   &                                            &              & $\{\alpha\A{=}(x,\alpha_1)\cons\alpha_2\}\cup U_1\cup U_2$                          \\
                                 &              & \textbf{let} $U_2$ \texttt{=} $V(\Gamma,e_2,\alpha_2)$ \textbf{in}                                                                                                                                                                     \\
                                 &              & $\{\alpha_1\A{=}\alpha_3\rightarrow\alpha,\alpha_3\A{\le}\alpha_2\}\cup U_1\cup U_2$
\end{tabular}

\vphantom{}

We want to prove that the constraint generation algorithm is correct.

First, for $\tau\in\Type$, define the access operation $\tau.p$ (which may fail):
\begin{align*}
  \tau.\epsilon & \triangleq\tau &  &  & \tau.px & \triangleq(\tau.p)(x)\quad\text{when }\tau.p\neq\bot &  &  & \tau.px & \triangleq\bot\quad\text{when }\tau.p=\bot
\end{align*}
and define the injection operation $\tau[\external]$:
\begin{align*}
  (\bullet)[\external]                 & \triangleq\bullet                                       &  &  & ((x,\tau)\cons\Gamma)[\external] & \triangleq(x,\tau[\external])\cons\Gamma[\external] \\
  (\alpha.p)[\external]                & \triangleq\alpha.p                                      &  &  & ([].p)[\external]                & \triangleq\external.p                               \\
  (\tau_1\rightarrow\tau_2)[\external] & \triangleq\tau_1[\external]\rightarrow\tau_2[\external]
\end{align*}

Let $\Subst\triangleq\fin{\TyVar}{\Type}$ be the set of substitutions.
For $S\in\Subst$, define:
\begin{align*}
  S\bullet    & \triangleq\bullet  &                                  &  & S(\tau_1\rightarrow\tau_2) & \triangleq S\tau_1\rightarrow S\tau_2                                      \\
  S(\alpha.p) & \triangleq\alpha.p & \text{when }\alpha\not\in dom(S) &  & S(\alpha.p)                & \triangleq \tau.p                     & \text{when }\alpha\mapsto\tau\in S \\
  S([].p)     & \triangleq [].p
\end{align*}
Define:
\begin{align*}
  (S,\external)\vDash U\triangleq & \forall(\tau_1\A{=}\tau_2)\in U:(S\tau_1)[\external]= (S\tau_2)[\external]\text{ and} \\
                                  & \forall(\tau_1\A{\le}\tau_2)\in U:(S\tau_1)[\external]\le(S\tau_2)[\external]
\end{align*}
where subtyping rules are the same as Figure \ref{fig:subtyping} and subtyping between type variables are not defined.

Then we can show that:
\begin{clm}[Correnctness of $V$]
  For $e\in\Expr$, $\Gamma,\external\in\TyEnv$, $\alpha\in\TyVar$, $S\in\Subst$:
  \begin{align*}
    (S,\external)\vDash V(\Gamma,e,\alpha) \Leftrightarrow & (S\Gamma)[\external]\vdash e:(S\alpha)[\external]
  \end{align*}
\end{clm}
\begin{proof}[Proof sketch]
  Structural induction on $e$.
\end{proof}

Note that by including $[].p$ in type environments, we can naturally generate constraints about the external environment $[]$.
Also, by injection, we can utilize constraints generated \emph{in advance} to obtain constraints generated from a more informed environment.
We extend injection to the output of the constraint-generating algorithm:
\begin{align*}
  \bot[\external] \triangleq & \bot                                                                                                          \\
  U[\external]    \triangleq & \{\tau_1[\external]\A{=}\tau_2[\external]|(\tau_1\A{=}\tau_2)\in U\}\cup                                      \\
                             & \{\tau_1[\external]\A{\le}\tau_2[\external]|(\tau_1\A{\le}\tau_2)\in U\} & \text{when all injections succeed} \\
  U[\external]    \triangleq & \bot                                                                     & \text{when injection fails}
\end{align*}
Then we can prove:
\begin{clm}[Advance]
  For $e\in\Expr$, $\Gamma,\external\in\TyEnv$, $\alpha\in\TyVar$:
  \[V(\Gamma[\external],e,\alpha)=V(\Gamma,e,\alpha)[\external]\]
\end{clm}
\begin{proof}[Proof sketch]
  Structural induction on $\Gamma$.
\end{proof}

\end{document}
%%% Local Variables: 
%%% coding: utf-8
%%% mode: latex
%%% TeX-engine: xetex
%%% End:
