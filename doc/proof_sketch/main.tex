%! TEX program = xelatex
\documentclass{article}
\usepackage{geometry}
\geometry{
  a4paper,
  total={170mm,257mm},
  left=20mm,
  top=20mm,
}
\usepackage{setspace}
\usepackage{graphicx}
\usepackage{xcolor}
\usepackage{kotex}
\usepackage{csquotes}

%%% Typesetting for listings
\usepackage{listings}

\definecolor{dkgreen}{rgb}{0,0.6,0}
\definecolor{ltblue}{rgb}{0,0.4,0.4}
\definecolor{dkviolet}{rgb}{0.3,0,0.5}

%%% lstlisting coq style (inspired from a file of Assia Mahboubi)
\lstdefinelanguage{Coq}{ 
    % Anything betweeen $ becomes LaTeX math mode
    mathescape=true,
    % Comments may or not include Latex commands
    texcl=false, 
    % Vernacular commands
    morekeywords=[1]{Section, Module, End, Require, Import, Export,
        Variable, Variables, Parameter, Parameters, Axiom, Hypothesis,
        Hypotheses, Notation, Local, Tactic, Reserved, Scope, Open, Close,
        Bind, Delimit, Definition, Let, Ltac, Fixpoint, CoFixpoint, Add,
        Morphism, Relation, Implicit, Arguments, Unset, Contextual,
        Strict, Prenex, Implicits, Inductive, CoInductive, Record,
        Structure, Canonical, Coercion, Context, Class, Global, Instance,
        Program, Infix, Theorem, Lemma, Corollary, Proposition, Fact,
        Remark, Example, Proof, Goal, Save, Qed, Defined, Hint, Resolve,
        Rewrite, View, Search, Show, Print, Printing, All, Eval, Check,
        Projections, inside, outside, Def},
    % Gallina
    morekeywords=[2]{forall, exists, exists2, fun, fix, cofix, struct,
        match, with, end, as, in, return, let, if, is, then, else, for, of,
        nosimpl, when},
    % Sorts
    morekeywords=[3]{Type, Prop, Set, true, false, option},
    % Various tactics, some are std Coq subsumed by ssr, for the manual purpose
    morekeywords=[4]{pose, set, move, case, elim, apply, clear, hnf,
        intro, intros, generalize, rename, pattern, after, destruct,
        induction, using, refine, inversion, injection, rewrite, congr,
        unlock, compute, ring, field, fourier, replace, fold, unfold,
        change, cutrewrite, simpl, have, suff, wlog, suffices, without,
        loss, nat_norm, assert, cut, trivial, revert, bool_congr, nat_congr,
        symmetry, transitivity, auto, split, left, right, autorewrite},
    % Terminators
    morekeywords=[5]{by, done, exact, reflexivity, tauto, romega, omega,
        assumption, solve, contradiction, discriminate},
    % Control
    morekeywords=[6]{do, last, first, try, idtac, repeat},
    % Comments delimiters, we do turn this off for the manual
    morecomment=[s]{(*}{*)},
    % Spaces are not displayed as a special character
    showstringspaces=false,
    % String delimiters
    morestring=[b]",
    morestring=[d],
    % Size of tabulations
    tabsize=3,
    % Enables ASCII chars 128 to 255
    extendedchars=false,
    % Case sensitivity
    sensitive=true,
    % Automatic breaking of long lines
    breaklines=false,
    % Default style fors listings
    basicstyle=\small,
    % Position of captions is bottom
    captionpos=b,
    % flexible columns
    columns=[l]flexible,
    % Style for (listings') identifiers
    identifierstyle={\ttfamily\color{black}},
    % Style for declaration keywords
    keywordstyle=[1]{\ttfamily\color{dkviolet}},
    % Style for gallina keywords
    keywordstyle=[2]{\ttfamily\color{dkgreen}},
    % Style for sorts keywords
    keywordstyle=[3]{\ttfamily\color{ltblue}},
    % Style for tactics keywords
    keywordstyle=[4]{\ttfamily\color{dkblue}},
    % Style for terminators keywords
    keywordstyle=[5]{\ttfamily\color{dkred}},
    %Style for iterators
    %keywordstyle=[6]{\ttfamily\color{dkpink}},
    % Style for strings
    stringstyle=\ttfamily,
    % Style for comments
    commentstyle={\ttfamily\color{dkgreen}},
    %moredelim=**[is][\ttfamily\color{red}]{/&}{&/},
    literate=
    {\\forall}{{\color{dkgreen}{$\forall\;$}}}1
    {\\exists}{{$\exists\;$}}1
    {<-}{{$\leftarrow\;$}}1
    {=>}{{$\Rightarrow\;$}}1
    {==}{{\code{==}\;}}1
    {==>}{{\code{==>}\;}}1
    %    {:>}{{\code{:>}\;}}1
    {->}{{$\rightarrow\;$}}1
    {<->}{{$\leftrightarrow\;$}}1
    {<==}{{$\leq\;$}}1
    {\#}{{$^\star$}}1 
    {\\o}{{$\circ\;$}}1 
    {\@}{{$\cdot$}}1 
    {\/\\}{{$\wedge\;$}}1
    {\\\/}{{$\vee\;$}}1
    {++}{{\code{++}}}1
    {~}{{\ }}1
    {\@\@}{{$@$}}1
    {\\mapsto}{{$\mapsto\;$}}1
    {\\hline}{{\rule{\linewidth}{0.5pt}}}1
    %
}[keywords,comments,strings]

%%% Math settings
\usepackage{amssymb,amsmath,amsthm,mathtools}
\usepackage[math-style=TeX,bold-style=TeX]{unicode-math}
\theoremstyle{definition}
\newtheorem{definition}{Definition}[section]
\newtheorem{example}{Example}[section]
\newtheorem{lem}{Lemma}[section]
\newtheorem{thm}{Theorem}[section]
\newtheorem{cor}{Corollary}[section]
\newtheorem{clm}{Claim}[section]

%%% Font settings
\setmainfont{Libertinus Serif}
\setsansfont{Libertinus Sans}[Scale=MatchUppercase]
\setmonofont{JuliaMono}[Scale=MatchLowercase]
\setmathfont{Libertinus Math} % Before set*hangulfont
\setmathfont{TeX Gyre Pagella Math}[range={\lbrace,\rbrace},Scale=1.1]
\setmainhangulfont{Noto Serif CJK KR}
\setmonohangulfont{D2Coding}

%%% PL constructs
\usepackage{galois}
\usepackage{ebproof}
\ebproofset{left label template=\textsc{[\inserttext]}}
\ebproofset{center=false}

%%% Custom commands
\newcommand*{\vbar}{|}
\newcommand*{\finto}{\xrightarrow{\text{\textrm{fin}}}}
\newcommand*{\istype}{\mathrel{⩴}}
\newcommand*{\ortype}{\mathrel{|}}
\newcommand*{\cons}{::}

\def\ovbarw{1.2mu}
\def\ovbarh{1}
\newcommand*{\ovbar}[1]{\mkern \ovbarw\overline{\mkern-\ovbarw{\smash{#1}\scalebox{1}[\ovbarh]{\vphantom{i}}}\mkern-\ovbarw}\mkern \ovbarw}
\newcommand*{\A}[1]{\ovbar{#1}}
\newcommand*{\Abs}[1]{{#1}^{\#}}
\newcommand*{\Expr}{\text{Expr}}
\newcommand*{\ExprVar}{\text{Var}}
\newcommand*{\Module}{\text{Module}}
\newcommand*{\ModVar}{\text{ModVar}}
\newcommand*{\Time}{\mathbb{T}}
\newcommand*{\ATime}{\A{\Time}}
\newcommand*{\Ctx}{\text{Ctx}}
\newcommand*{\Value}{\text{Val}}
\newcommand*{\Mem}{\text{Mem}}
\newcommand*{\Left}{\text{Left}}
\newcommand*{\Right}{\text{Right}}
\newcommand*{\Sig}{\text{Sig}}
\newcommand*{\mem}{m}
\newcommand*{\AMem}{\A{\text{Mem}}}
\newcommand*{\State}{\text{State}}
\newcommand*{\AState}{\A{\text{State}}}
\newcommand*{\Result}{\text{Result}}
\newcommand*{\AResult}{\A{\text{Result}}}
\newcommand*{\Tick}{\text{Tick}}
\newcommand*{\lfp}{\mathsf{lfp}}
\newcommand*{\Step}{\mathsf{Step}}
\newcommand*{\semarrow}{\rightsquigarrow}
\newcommand*{\asemarrow}{\A{\rightsquigarrow}}
\newcommand*{\synlink}{\rtimes}
\newcommand*{\semlink}{\mathbin{\rotatebox[origin=c]{180}{$\propto$}}}
\newcommand*{\link}[2]{{#1}\rtimes{#2}}
\newcommand*{\mt}{\mathsf{empty}}

\newcommand*{\doubleplus}{\ensuremath{\mathbin{+\mkern-3mu+}}}
\newcommand*{\project}{\text{\texttt{:>} }}
\newcommand*{\Exp}{\mathsf{Exp}}
\newcommand*{\Imp}{\mathsf{Imp}}
\newcommand*{\Fin}{\mathsf{Fin}}
\newcommand*{\Link}{\mathsf{Link}}
\newcommand*{\sembracket}[1]{\lBrack{#1}\rBrack}
\newcommand*{\fin}[2]{{#1}\xrightarrow{\text{fin}}{#2}}
\newcommand*{\addr}{\mathsf{addr}}
\newcommand*{\tick}{\mathsf{tick}}
\newcommand*{\modctx}{\mathsf{ctx}}
\newcommand*{\mapinject}[2]{{#2}[{#1}]}
\newcommand*{\inject}[2]{{#2}\langle{#1}\rangle}
\newcommand*{\deletepre}[2]{{#2}\overline{\doubleplus}{#1}}
\newcommand*{\deletemap}[2]{{#1}\overline{[{#2}]}}
\newcommand*{\delete}[2]{{#2}{\langle{#1}\rangle}^{-1}}
\newcommand*{\filter}{\mathsf{filter}}
\newcommand*{\Let}{\mathtt{let}}

\title{(Sketches of) Proofs for Modular Analysis}
\author{Joonhyup Lee}
\begin{document}
\maketitle

\section{Part I: Semantics that Allow Open Code to be Closed Fractionally}

We make the following observations:
\begin{itemize}
  \item Most code that static analyzers deal with is \emph{open code} that uses external values.
  \item Those external values are defined in a different \emph{scope} from the code of interest.
  \item The different scopes are organized in term of \emph{modules}.
  \item The modules are usually bound to \emph{module names}.
\end{itemize}
Therefore, experts who write realistic analyzers are immediately faced with the problem of \emph{closing} open code.
Especially, in the case when external values are not defined in the same language, the semantics of such values must be \emph{modelled}, either by the analysis expert or by the user of the analyzer.
Since we cannot possibly model all such cases in one try, attempts to close open code must be a never-ending race of fractional advances.

If we force the analyzers to output results only in the fortunate case that all external values has already been modelled, we end up unnecessarily recomputing each time we fail to close completely.
We claim that this is undesirable: the analyzer, upon meeting an open term, may just ``cache'' what has been computed already and ``pick up'' from there when that open term is resolved.
No doubt, there may already be program analyzers that perform such caching, but how can we model such a computation \emph{mathematically}?
Therefore, we aim to define semantics for terms that have been fractionally closed, and prove that closing the \emph{fractionally closed semantics} is equal to the \emph{closed semantics}.

To illustrate what will be proven, say that $e$ is an open term, $S_2$ is the set of contexts that close $e$ fractionally, and $S_1$ is another set of contexts that aim to fill in the blanks.
What we aim to show is:
\[\sembracket{e}(S_1\rhd S_2)=S_1\semlink\sembracket{e}S_2\]
when $S_1\semlink\sembracket{e}S_2$ closes the fractionally closed $\sembracket{e}S_2$ and $\sembracket{e}(S_1\rhd S_2)$ starts from the filled-in states.

In this section, we will define:
\begin{enumerate}
  \item The \emph{abstract syntax} of a model language with modules that can be bound to names.
  \item The \emph{semantics} of the language that is defined regardless of openness.
\end{enumerate}
and sketch how to design an analysis that allows fractional specification.
\subsection{Abstract Syntax}
The language is basically an extension of untyped lambda calculus with modules and the linking construct.
$\link{e_1}{e_2}$ means that $e_1$ is a module that is evaluated first to a \emph{context}, and that $e_2$ is evaluated under the exported context.

\begin{figure}[htb]
  \centering
  \begin{tabular}{rrcll}
    Identifiers & $x,M$ & $\in$         & $\ExprVar$                              \\
    Expression  & $e$   & $\rightarrow$ & $x$                & value identifier   \\
                &       & $\vbar$       & $\lambda x.e$      & function           \\
                &       & $\vbar$       & $e$ $e$            & application        \\
                &       & $\vbar$       & $\link{e}{e}$      & linked expression  \\
                &       & $\vbar$       & $\varepsilon$      & empty module       \\
                &       & $\vbar$       & $M$                & module identifier  \\
                &       & $\vbar$       & $\Let$ $x$ $e$ $e$ & binding expression \\
                &       & $\vbar$       & $\Let$ $M$ $e$ $e$ & binding module     \\
  \end{tabular}
  \caption{Abstract syntax of the simple module language.}
  \label{fig:syntax}
\end{figure}
\subsection{Operational Semantics}
\begin{figure}[h!]
  \centering
  \begin{tabular}{rrcll}
    Environment/Context          & $C$ & $\in$         & $\Ctx$                                                      \\
    Value of expressions         & $v$ & $\in$         & $\Value \subseteq \Expr\times\Ctx$                          \\
    Value of expressions/modules & $V$ & $\in$         & $\Value\Ctx\triangleq\Value\uplus\Ctx$                      \\
    Context                      & $C$ & $\rightarrow$ & []                                     & empty stack        \\
                                 &     & $\vbar$       & $(x,v)\cons C$                         & expression binding \\
                                 &     & $\vbar$       & $(M,C)\cons C$                         & module binding     \\
    Value of expressions         & $v$ & $\rightarrow$ & $\langle \lambda x.e, C \rangle$       & closure
  \end{tabular}
  \caption{Definition of the semantic domains.}
  \label{fig:simpdom}
\end{figure}

\begin{figure}[h!]
  \begin{flushright}
    \fbox{$(e,C)\semarrow V\text{ or }(e',C')$}
  \end{flushright}
  \centering
  \vspace{0pt} % -0.75em}
  \[
    \begin{prooftree}
      \hypo{v=C(x)}
      \infer[left label=ExprID]1{
      (x, C)
      \semarrow
      v
      }
    \end{prooftree}\qquad
    \begin{prooftree}
      \infer[left label=Fn]0{
      (\lambda x.e, C)
      \semarrow
      \langle\lambda x.e, C\rangle
      }
    \end{prooftree}\qquad
    \begin{prooftree}
      \infer[left label={AppL}]0{
      (e_{1}\:e_{2}, C)
      \semarrow
      (e_{1},C)
      }
    \end{prooftree}\qquad
    \begin{prooftree}
      \hypo{
        \begin{matrix}
          (e_{1}, C)
          \semarrow
          \langle\lambda x.e_{\lambda}, C_{\lambda}\rangle
        \end{matrix}
      }
      \infer[left label={AppR}]1{
      (e_{1}\:e_{2}, C)
      \semarrow
      (e_{2}, C)
      }
    \end{prooftree}
  \]

  \[
    \begin{prooftree}
      \hypo{
        \begin{matrix}
          (e_{1}, C)
          \semarrow
          \langle\lambda x.e_{\lambda}, C_{\lambda}\rangle \\
          (e_{2}, C)
          \semarrow
          v
        \end{matrix}
      }
      \infer[left label={AppBody}]1{
      (e_{1}\:e_{2}, C)
      \semarrow
      (e_{\lambda}, (x, v)\cons C_{\lambda})
      }
    \end{prooftree}\qquad
    \begin{prooftree}
      \hypo{
        \begin{matrix}
          (e_{1}, C)
          \semarrow
          \langle\lambda x.e_{\lambda}, C_{\lambda}\rangle \\
          (e_{2}, C)
          \semarrow
          v                                                \\
          (e_{\lambda}, (x, v)\cons C_{\lambda})
          \semarrow
          v'
        \end{matrix}
      }
      \infer[left label={App}]1{
      (e_{1}\:e_{2}, C)
      \semarrow
      v'
      }
    \end{prooftree}
  \]

  \[
    \begin{prooftree}
      \infer[left label=LinkL]0{
      (\link{e_{1}}{e_{2}}, C)
      \semarrow
      (e_{1}, C)
      }
    \end{prooftree}\qquad
    \begin{prooftree}
      \hypo{
        \begin{matrix}
          (e_{1}, C)
          \semarrow
          C'
        \end{matrix}
      }
      \infer[left label=LinkR]1{
      (\link{e_{1}}{e_{2}}, C)
      \semarrow
      (e_{2}, C')
      }
    \end{prooftree}\qquad
    \begin{prooftree}
      \hypo{
        \begin{matrix}
          (e_{1}, C)
          \semarrow
          C' \\
          (e_{2}, C')
          \semarrow
          V
        \end{matrix}
      }
      \infer[left label=Link]1{
      (\link{e_{1}}{e_{2}}, C)
      \semarrow
      V
      }
    \end{prooftree}
  \]

  \[
    \begin{prooftree}
      \infer[left label=Empty]0{
      (\varepsilon, C)
      \semarrow
      C
      }
    \end{prooftree}\qquad
    \begin{prooftree}
      \hypo{C'=C(M)}
      \infer[left label=ModID]1{
      (M, C)
      \semarrow
      C'
      }
    \end{prooftree}
  \]

  \[
    \begin{prooftree}
      \infer[left label=LetEL]0{
      (\mathtt{let}\:x\:e_1\:e_2, C)
      \semarrow
      (e_{1}, C)
      }
    \end{prooftree}\quad
    \begin{prooftree}
      \hypo{
        \begin{matrix}
          (e_{1}, C)
          \semarrow
          v
        \end{matrix}
      }
      \infer[left label=LetER]1{
      (\mathtt{let}\:x\:e_1\:e_2, C)
      \semarrow
      (e_{2}, (x, v)\cons C)
      }
    \end{prooftree}\quad
    \begin{prooftree}
      \hypo{
        \begin{matrix}
          (e_{1}, C)
          \semarrow
          v \\
          (e_{2}, (x, v)\cons C)
          \semarrow
          C'
        \end{matrix}
      }
      \infer[left label=LetE]1{
      (\mathtt{let}\:x\:e_1\:e_2, C)
      \semarrow
      C'
      }
    \end{prooftree}
  \]

  \[
    \begin{prooftree}
      \infer[left label=LetML]0{
      (\mathtt{let}\:M\:e_{1}\:e_{2}, C)
      \semarrow
      (e_{1}, C)
      }
    \end{prooftree}\quad
    \begin{prooftree}
      \hypo{
        \begin{matrix}
          (e_{1}, C)
          \semarrow
          C'
        \end{matrix}
      }
      \infer[left label=LetMR]1{
      (\mathtt{let}\:M\:e_{1}\:e_{2}, C)
      \semarrow
      (e_{2}, (M, C')\cons C)
      }
    \end{prooftree}\quad
    \begin{prooftree}
      \hypo{
        \begin{matrix}
          (e_{1}, C)
          \semarrow
          C' \\
          (e_{2}, (M, C')\cons C)
          \semarrow
          C''
        \end{matrix}
      }
      \infer[left label=LetM]1{
      (\mathtt{let}\:M\:e_{1}\:e_{2}, C)
      \semarrow
      C''
      }
    \end{prooftree}
  \]
  \caption{The concrete one-step transition relation.}
  \label{fig:simpreach}
\end{figure}
We present the operational semantics $\semarrow$ for our language.
The semantic domains are given in Figure \ref{fig:simpdom} and the operational semantics is defined in Figure \ref{fig:simpreach}.

Our semantics relate an element $\ell$ of $\Left\triangleq\Expr\times\Ctx$ with an element $\rho$ of $\Right\triangleq\Left\uplus\Value\Ctx$.
Note that $C(x)$ pops the highest value that is associated with $x$ from the stack $C$ and $C(M)$ pops the highest context associated with $M$ from $C$.
The relation $\semarrow$ is unorthodox in that unlike normal big-step operaional semantics, the relation $\semarrow$ relates a configuration not only to its final result but also to intermediate configurations of which its values are required to compute the final result.
Why it is defined as such is because defining a \emph{collecting semantics} becomes much simpler.

\subsection{Collecting Semantics}
To define a semantics that is computable, we must formulate the collecting semantics as a least fixed point of a monotonic function that maps an element of some CPO $D$ to $D$.
In our case, $D\triangleq\wp(\Sigma)$ when $\Sigma\triangleq\semarrow\uplus\Right$.
The semantics of an expression $e$ starting from initial states in $S\subseteq\Ctx$ is the collection of $\ell\semarrow\rho$ and $\rho$ derivable from initial configurations $(e,C)$ with $C\in S$.
Defining the transfer function is straightforward from the definition of the transition relation.

\begin{definition}[Transfer function]
  Given $A\in D$, define
  \[
    \mathsf{Step}(A)\triangleq
    \left\{\ell\semarrow\rho, \rho\middle|
    \begin{prooftree}[center=true]
      \hypo{A'}\infer1{\ell\semarrow\rho}
    \end{prooftree}\wedge
    A'\subseteq A\wedge\ell\in A
    \right\}
  \]
\end{definition}

The $\mathsf{Step}$ function is naturally monotonic, as a ``cache'' $A$ that remembers more about the intermediate proof tree will derive more results than a cache that remembers less.
Now, because of Tarski's fixpoint theorem, we can formulate the collecting semantics in fixpoint form.
\begin{definition}[Collecting semantics]
  Given $e\in\Expr$ and $S\subseteq\Ctx$, define:
  \[
    \sembracket{e}S\triangleq\lfp(\lambda X.\mathsf{Step}(X)\cup\{(e,C)|C\in S\})
  \]
\end{definition}
Note that the above definition can be defined without qualms for situations when the $C$ in $(e,C)$ does not close $e$.
Then the collecting semantics will store the proof tree only up to the point the first free variable is evaluated.

\subsection{Injection and Linking}
We first define what it means to \emph{fill in the blanks} of an individual $V_2\in\Value\Ctx$ with a $C_1\in\Ctx$:
\[
  \inject{C_{1}}{V_{2}}\triangleq
  \begin{cases}
    C_1                                           & V_{2}=[]                        \\
    (x, \inject{C_1}{v})\cons\inject{C_{1}}{C}    & V_{2}=(x,v)\cons C              \\
    (M, \inject{C_{1}}{C})\cons\inject{C_{1}}{C'} & V_{2}=(M,C)\cons C'             \\
    \langle\lambda x.e,\inject{C_1}{C}\rangle     & V_2=\langle\lambda x.e,C\rangle
  \end{cases}
\]
This does indeed ``fill in the blanks'', since:
\begin{clm}[Fill in the Blanks]
  For all $C_1,C_2\in\Ctx$, for each expression variable $x$,
  \[
    C_2(x)=v\Rightarrow\inject{C_1}{C_2}(x)=\inject{C_1}{v}\text{ and }C_2(x)=\bot\Rightarrow\inject{C_1}{C_2}(x)=C_1(x)
  \]
  and for each module variable $M$,
  \[
    C_2(M)=C\Rightarrow\inject{C_1}{C_2}(x)=\inject{C_1}{C}\text{ and }C_2(M)=\bot\Rightarrow\inject{C_1}{C_2}(M)=C_1(M)
  \]
\end{clm}
\begin{proof}[Sketch]
  Induction on $C_2$.
\end{proof}

Moreover, filling in the blanks preserves the evaluation relation $\semarrow$:
\begin{clm}[Injection Preserves Evaluation]
  For all $\ell\in\Left$, $\rho\in\Right$, $\ell\semarrow\rho\Rightarrow\inject{C}{\ell}\semarrow\inject{C}{\rho}$. More explicity,
  \[(\ell,\rho)\in\Sigma\Rightarrow(\inject{C}{\ell},\inject{C}{\rho})\in\Sigma\]
\end{clm}
\begin{proof}[Sketch]
  Induction on $\semarrow$.
\end{proof}

Thus, we can define $\rhd$ that injects a \emph{set} of contexts $S$ into a subset $A$ of $D$ and a semantic linking operation $\semlink$ that does the rest of the computation:
\begin{definition}[Injection]
  For $C_1\in\Ctx,\ell=(e,C_2)\in\Left$, define $\inject{C_1}{\ell}\triangleq(e,\inject{C_1}{C_2})$.

  Then for $S\subseteq\Ctx$ and $A\in D$, define:
  \[S\rhd A\triangleq\{\inject{C}{\rho}|C\in S\wedge\rho\in A\}\cup\{\inject{C}{\ell}\semarrow\inject{C}{\rho}|C\in S\wedge\ell\semarrow\rho\in A\}\]
\end{definition}
\begin{definition}[Semantic Linking]
  For $S\subseteq\Ctx$ and $A\in D$, define:
  \[S\semlink A\triangleq\lfp(\lambda X.\Step(X)\cup(S\rhd A))\]
\end{definition}

Thus we reach the main theorem that allows ``fractional closures'' to be soundly defined:
\begin{clm}[Advance]
  For all $e\in\Expr$ and $S_1,S_2\subseteq\Ctx$,
  \[\sembracket{e}(S_1\rhd S_2)=S_1\semlink\sembracket{e}S_2\]
\end{clm}
\begin{proof}
  Let $A$ be $\{(e,C)|C\in S_1\rhd S_2\}$, and let $B$ be $S_1\rhd\sembracket{e}S_2$.
  Note that $A\subseteq B$ by the definition of $\sembracket{e}S_2$.

  Also, let $X_A$ be $\lfp(\lambda X.\Step(X)\cup A)=\sembracket{e}(S_1\rhd S_2)$ and let $X_B$ be $\lfp(\lambda X.\Step(X)\cup B)=S_1\semlink\sembracket{e}S_2$.
  By the previous lemma, we have that $B\subseteq X_A$.

  Then first, $X_A$ is a fixed point of $\lambda X.\Step(X)\cup B$, since
  \[X_A=X_A\cup B=(\Step(X_A)\cup A)\cup B=\Step(X_A)\cup(A\cup B)=\Step(X_A)\cup B\]
  . Then since $X_B$ is the least fixed point, $X_B\subseteq X_A$.

  Also, note that $X_B$ is a pre-fixed point of $\lambda X.\Step(X)\cup A$, since
  \[\Step(X_B)\cup A\subseteq\Step(X_B)\cup B=X_B\]
  . $D$ is a complete lattice, so by Tarski's fixpoint theorem, $X_A$ is the least of all pre-fixed points.
  Thus, $X_A\subseteq X_B$.

  Since $X_B\subseteq X_A$ and $X_A\subseteq X_B$, we have that $X_A=X_B$.
\end{proof}

\subsection{Skeleton of a Static Analysis}
Since we have defined a semantics that fully embrace \emph{incomplete computations}, we only have to abstract our semantic operators to obtain a sound static analysis.

We require a CPO $\Abs{D}$ that is Galois connected with $D$ by abstraction $\alpha$ and concretization $\gamma$:
\[D\galois{\alpha}{\gamma}\Abs{D}\]
and semantic operators $\Abs\Step$ and $\Abs\rhd$ that satisfies:
\[\Step\circ\gamma\subseteq\gamma\circ\Abs\Step\qquad\rhd\circ(\gamma,\gamma)\subseteq\gamma\circ\Abs\rhd\]
. Then we define $\Abs{\sembracket{e}}$ and $\Abs\semlink$ as:
\[
  \Abs{\sembracket{e}}\Abs{S}\triangleq\lfp(\lambda\Abs{X}.\Abs\Step(\Abs{X})\Abs\cup\alpha\{(e,C)|C\in\gamma\Abs{S}\})\qquad
  \Abs{S}\Abs\semlink\Abs{A}\triangleq\lfp(\lambda\Abs{X}.\Abs\Step(\Abs{X})\Abs\cup(\Abs{S}\Abs\rhd\Abs{A}))
\]
which, by definition and Tarski's fixpoint theorem satisfies:
\[\sembracket{e}\circ\gamma\subseteq\gamma\circ\Abs{\sembracket{e}}\qquad\semlink\circ(\gamma,\gamma)\subseteq\gamma\circ\Abs\semlink\]
. Then we can soundly approximate fractional specifications by:
\begin{align*}
  S_1\semlink\sembracket{e}S_2 & \subseteq S_1\semlink\gamma(\Abs{\sembracket{e}}\alpha(S_2))                 & (\because\sembracket{e}\subseteq\gamma\circ\Abs{\sembracket{e}}\circ\alpha\text{ and monotonicity of }\semlink) \\
                               & \subseteq \gamma(\alpha(S_1))\semlink\gamma(\Abs{\sembracket{e}}\alpha(S_2)) & (\because\text{id}\subseteq\gamma\circ\alpha\text{ and monotonicity of }\semlink)                               \\
                               & \subseteq\gamma(\alpha(S_1)\Abs\semlink\Abs{\sembracket{e}}\alpha(S_2))      & (\because\semlink\circ(\gamma,\gamma)\subseteq\gamma\circ\Abs\semlink)
\end{align*}

\section{Part II: A Simple Method to Derive Sound Analyzers with Arbitrary Precision}

It is not obvious how to abstract the semantics in the previous section.
For easy abstraction, we employ the common technique of introducing addresses and a memory.
Thus, we extend the operational semantics of the previous section to a sematics that involve choosing a \emph{time} domain $\Time$ to use as addresses,
and an \emph{abstract time domain} $\A\Time$ that allow easy abstraction of the semantics via a \emph{single} function $\A\alpha:\Time\rightarrow\A\Time$.

\subsection{Semantic Domains}
\begin{figure}[h!]
  \centering
  \begin{tabular}{rrcll}
    Time                         & $t$     & $\in$         & $\Time$                                                                               \\
    Environment/Context          & $C$     & $\in$         & $\Ctx$                                                                                \\
    Value of expressions         & $v$     & $\in$         & $\Value \subseteq \Expr\times\Ctx$                                                    \\
    Value of expressions/modules & $V$     & $\in$         & $\Value\Ctx\triangleq\Value\uplus\Ctx$                                                \\
    Memory                       & $\mem$  & $\in$         & $\Mem \triangleq \fin{\Time}{\Value}$                                                 \\
    State                        & $s$     & $\in$         & $\State \subseteq \Ctx\times\Mem\times\Time$                                          \\
    Result                       & $r$     & $\in$         & $\Result \subseteq \Value\Ctx\times\Mem\times\Time$                                   \\
    Context                      & $C$     & $\rightarrow$ & []                                                               & empty stack        \\
                                 &         & $\vbar$       & $(x,t)\cons C$                                                   & expression binding \\
                                 &         & $\vbar$       & $(M,C)\cons C$                                                   & module binding     \\
    Value of expressions         & $v$     & $\rightarrow$ & $\langle \lambda x.e, C \rangle$                                 & closure
  \end{tabular}
  \caption{Definition of the semantic domains in the concrete case.}
  \label{fig:concdom}
\end{figure}

The domains for defining the operational semantics is extended to include the \emph{concrete time} and \emph{memory}.
Compared with Figure \ref{fig:simpdom}, Figure \ref{fig:concdom} defines four more sets, $\Time$, $\Mem$, $\State$, and $\Result$ to streamline the presentation.
A $s=(C,\mem,t)\in\State$ corresponds to a $C$ in Figure \ref{fig:simpdom}, as the pair $C,\mem$ cooperates to represent the recursively defined $C$ in the original representation without memory.
The $t$ component represents both the execution history up to that point and the address that will be written to next.
Similarly, a $r=(V,\mem,t)\in\Result$ corresponds to a $V$ in Figure \ref{fig:simpdom}, as the pair $V,\mem$ cooperates to represent the recursively defined $V$.

\begin{figure}[h!]
  \centering
  \begin{tabular}{rrcll}
    Abstract Time                & $\A{t}$     & $\in$         & $\A{\Time}$                                                                                            \\
    Environment/Context          & $\A{C}$     & $\in$         & $\A{\Ctx}$                                                                                             \\
    Value of expressions         & $\A{v}$     & $\in$         & $\A{\Value} \subseteq \Expr\times\A{\Ctx}$                                                             \\
    Value of expressions/modules & $\A{V}$     & $\in$         & $\A{\Value\Ctx}\triangleq\A{\Value}\uplus\A{\Ctx}$                                                     \\
    Abstract Memory              & $\A{\mem}$  & $\in$         & $\A{\Mem} \triangleq \fin{\A{\Time}}{\wp(\A{\Value})}$                                                 \\
    Abstract State               & $\A{s}$     & $\in$         & $\A{\State} \triangleq \A{\Ctx}\times\A{\Mem}\times\A{\Time}$                                          \\
    Abstract Result              & $\A{r}$     & $\in$         & $\A{\Result} \triangleq \A{\Value\Ctx}\times\A{\Mem}\times\A{\Time}$                                   \\
    Abstract Tick                & $\A{\tick}$ & $\in$         & $\A{\Tick}\triangleq\A{\State}\times\ExprVar\times\A{\Value}\rightarrow\A{\Time}$                      \\
    Context                      & $\A{C}$     & $\rightarrow$ & []                                                                                & empty stack        \\
                                 &             & $\vbar$       & $(x,\A{t})\cons \A{C}$                                                            & expression binding \\
                                 &             & $\vbar$       & $(M,\A{C})\cons \A{C}$                                                            & module binding     \\
    Value of expressions         & $\A{v}$     & $\rightarrow$ & $\langle \lambda x.e, \A{C} \rangle$                                              & closure
  \end{tabular}
  \caption{Definition of the semantic domains in the abstract case. By $\wp(S)$ we mean the powerset of $S$.}
\end{figure}
We first introduce sets that are used to describe the abstract version of the operational semantics.
Among these sets, the most important component for the analysis designer is the $\A{\tick}$ component.
The $\A{\tick}$ function is what determines the granularity of the analysis.
Since it takes in $(\A{s},x,\A{v})$ and returns a new $\A{t}$, the timestamps can encode a variety of context information.
Increasing the variety of information encoded in $\A{t}$ increases the accuracy of the analysis.

Naturally, all $\A{\tick}$ functions must have a concrete $\tick$ that mirrors its action.
Since the abstract time domain $\A{\Time}$ and the translation function $\A{\alpha}$ meets the basic list of requirements, for each $\A{\tick}$ there exists a $\tick$ that \emph{respects} $\A{\tick}$ and produces \emph{fresh} timestamps.
To elaborate, we must take a look at the definitions for the concrete semantic domains.

To enforce that the $\tick$ function always produces fresh timestamps, the definition of the concrete $\State$ and $\Result$ ensure that the next timestamp that is recorded in memory is \emph{larger} than any timestamp that is already used.
Define:
\[
  \begin{array}{cc}
    C < t\triangleq
    \begin{cases}
      \mathsf{True}        & C=[]              \\
      t' < t\wedge C' < t  & C=(x,t')\cons C'  \\
      C' < t\wedge C'' < t & C=(M,C')\cons C''
    \end{cases}
     &
    \begin{array}{l}
      V < t\triangleq
      \begin{cases}
        C < t & V=\langle\_,C\rangle \\
        C < t & V=C
      \end{cases} \\
      \mem < t\triangleq\forall t'\in\mathsf{dom}(\mem):t'<t\wedge\mem(t') < t
    \end{array}
  \end{array}
\]
and define:
\[
  \State\triangleq\{(C,m,t)|C<t\wedge m<t\}\qquad
  \Result\triangleq\{(V,m,t)|V<t\wedge m<t\}
\]
Note that $\State\subseteq\Result$, thus any statement about $r\in\Result$ also holds for any $s\in\State$.

The fact that the $t$-component in any $r$ bounds the $C$ and $m$ component means that the next location to be updated is fresh.
After updating the memory, a new timestamp that is larger than the original $t$ must be given by $\tick$.
This implies that: $\forall t:t<\tick((\_,\_,t),\_,\_)$ should be required of $\tick$.
Moreover, for every concrete $\tick$ to have a corresponding abstract $\A{\tick}$ connected through $\A{\alpha}$, we want to require that:
$\exists\A{\tick}:\A{\alpha}\circ\tick=\A{\tick}\circ\A{\alpha}$, when:
\[
  \begin{array}{cc}
    \A{\alpha}(C)\triangleq
    \begin{cases}
      []                                  & C=[]              \\
      (x,\A{\alpha}(t))::\A{\alpha}(C')   & C=(x,t)\cons C'   \\
      (M,\A{\alpha}(C'))::\A{\alpha}(C'') & C=(M,C')\cons C''
    \end{cases}
     &
    \begin{array}{l}
      \A{\alpha}(V)\triangleq
      \begin{cases}
        \langle e,\A{\alpha}(C)\rangle & V=\langle e,C\rangle \\
        \A{\alpha}(C)                  & V=C
      \end{cases} \\
      \A{\alpha}(\mem)\triangleq\bigcup_{t\in\mathsf{dom}(\mem)}[\A{\alpha}(t)\mapsto\{\A{\alpha}(\mem(t))\}]
    \end{array}
  \end{array}
\]

Thus we can define:
\[
  \Tick\triangleq\{\tick|\forall t:t<\tick((\_,\_,t),\_,\_)\wedge\exists\A{\tick}:\A{\alpha}\circ\tick=\A{\tick}\circ\A{\alpha}\}
\]
when the $\A{\alpha}$ in $\A{\tick}\circ\A{\alpha}$ maps over tuples and define:
\[
  \A{\alpha}(\tick)\triangleq\A{\tick}\qquad\text{where }\A{\alpha}\circ\tick=\A{\tick}\circ\A{\alpha}
\]
since if such a $\A{\tick}$ exists, it must be unique.

\subsubsection{Operational Semantics}
Now we give the operational semantics.
The relation $\semarrow_{\tick}$ relates two machine states $\ell\in\Left\triangleq\Expr\times\State$ and $\rho\in\Right\triangleq\Left\uplus\Result$, using $\tick$ as the policy for memory allocation.
Likewise, $\asemarrow_{\A{\tick}}$ relates two machine states $\A{\ell}\in\A{\Left}\triangleq\Expr\times\AState$ and $\A{\rho}\in\A{\Right}\triangleq\A{\Left}\uplus\AResult$, using $\A{\tick}$.

There are auxiliary functions that are used to describe the relation.
The $\addr$ and $\modctx$ operators each extract the address bound to a variable $x$ and the context bound to a variable $M$ from a context $C$.
\[
  {C}(x)\triangleq
  \begin{cases}
    \bot         & C=[]                              \\
    t            & C=(x, t)\cons C'                  \\
    \addr(C',x)  & C=(x', t)\cons C' \wedge x'\neq x \\
    \addr(C'',x) & C=(M, C')\cons C''
  \end{cases}
  \quad
  {C}(M)\triangleq
  \begin{cases}
    \bot           & C=[]                               \\
    C'             & C=(M, C')\cons C''                 \\
    \modctx(C'',M) & C=(M', C')\cons C''\wedge M'\neq M \\
    \modctx(C',M)  & C=(x, t)\cons C'
  \end{cases}
\]

Also, the update of the memory for the abstract operational semantics is not a strong update, which overwrites the previous value stored at the address that is written to.
\[
  \A{\mem}[\A{t}\A{\mapsto}\A{v}](\A{t'})\triangleq
  \begin{cases}
    \A{\mem}(\A{t})\cup\{\A{v}\} & (\A{t'}=\A{t})     \\
    \A{\mem}(\A{t'})             & (\text{otherwise})
  \end{cases}
\]
We call this a ``weak update'' of the abstract memory.

The definitions for the operational semantics are given in Figure \ref{fig:concreach} and Figure \ref{fig:absreach}.

\begin{figure}[h!]
  \begin{flushright}
    \fbox{$(e,C,\mem,t)\semarrow(V,\mem',t')\text{ or }(e',C',\mem',t')$}
  \end{flushright}
  \centering
  \vspace{0pt} % -0.75em}
  \[
    \begin{prooftree}
      \hypo{t_{x}={C}(x)}
      \hypo{v=\mem(t_{x})}
      \infer[left label=ExprID]2{
      (x, C, \mem, t)
      \semarrow
      (v, \mem, t)
      }
    \end{prooftree}\qquad
    \begin{prooftree}
      \infer[left label=Fn]0{
      (\lambda x.e, C, \mem, t)
      \semarrow
      (\langle\lambda x.e, C\rangle, \mem, t)
      }
    \end{prooftree}
  \]

  \[
    \begin{prooftree}
      \infer[left label={AppL}]0{
      (e_{1}\:e_{2}, C, \mem, t)
      \semarrow
      (e_{1},C, \mem,t)
      }
    \end{prooftree}\qquad
    \begin{prooftree}
      \hypo{
        \begin{matrix}
          (e_{1}, C, \mem, t)
          \semarrow
          (\langle\lambda x.e_{\lambda}, C_{\lambda}\rangle, \mem_{\lambda}, t_{\lambda})
        \end{matrix}
      }
      \infer[left label={AppR}]1{
      (e_{1}\:e_{2}, C, \mem, t)
      \semarrow
      (e_{2}, C, \mem_{\lambda}, t_{\lambda})
      }
    \end{prooftree}
  \]

  \[
    \begin{prooftree}
      \hypo{
        \begin{matrix}
          (e_{1}, C, \mem, t)
          \semarrow
          (\langle\lambda x.e_{\lambda}, C_{\lambda}\rangle, \mem_{\lambda}, t_{\lambda}) \\
          (e_{2}, C, \mem_{\lambda}, t_{\lambda})
          \semarrow
          (v, \mem_{a}, t_{a})
        \end{matrix}
      }
      \infer[left label={AppBody}]1{
      (e_{1}\:e_{2}, C, \mem, t)
      \semarrow
      (e_{\lambda}, (x, t_{a})\cons C_{\lambda}, \mem_{a}[t_{a}\mapsto v], \tick(t_a))
      }
    \end{prooftree}
  \]

  \[
    \begin{prooftree}
      \hypo{
        \begin{matrix}
          (e_{1}, C, \mem, t)
          \semarrow
          (\langle\lambda x.e_{\lambda}, C_{\lambda}\rangle, \mem_{\lambda}, t_{\lambda}) \\
          (e_{2}, C, \mem_{\lambda}, t_{\lambda})
          \semarrow
          (v, \mem_{a}, t_{a})                                                            \\
          (e_{\lambda}, (x, t_{a})\cons C_{\lambda}, \mem_{a}[t_{a}\mapsto v], \tick(t_a))
          \semarrow
          (v', \mem',t')
        \end{matrix}
      }
      \infer[left label={App}]1{
      (e_{1}\:e_{2}, C, \mem, t)
      \semarrow
      (v', \mem',t')
      }
    \end{prooftree}
  \]

  \[
    \begin{prooftree}
      \infer[left label=LinkL]0{
      (\link{e_{1}}{e_{2}}, C, \mem, t)
      \semarrow
      (e_{1}, C, \mem, t)
      }
    \end{prooftree}\qquad
    \begin{prooftree}
      \hypo{
        \begin{matrix}
          (e_{1}, C, \mem, t)
          \semarrow
          (C', \mem', t')
        \end{matrix}
      }
      \infer[left label=LinkR]1{
      (\link{e_{1}}{e_{2}}, C, \mem, t)
      \semarrow
      (e_{2}, C', \mem', t')
      }
    \end{prooftree}
  \]

  \[
    \begin{prooftree}
      \hypo{
        \begin{matrix}
          (e_{1}, C, \mem, t)
          \semarrow
          (C', \mem', t') \\
          (e_{2}, C', \mem', t')
          \semarrow
          (V, \mem'', t'')
        \end{matrix}
      }
      \infer[left label=Link]1{
      (\link{e_{1}}{e_{2}}, C, \mem, t)
      \semarrow
      (V, \mem'', t'')
      }
    \end{prooftree}\qquad
    \begin{prooftree}
      \infer[left label=Empty]0{
      (\varepsilon, C, \mem, t)
      \semarrow
      (C, \mem, t)
      }
    \end{prooftree}\qquad
    \begin{prooftree}
      \hypo{C'={C}(M)}
      \infer[left label=ModID]1{
      (M, C, \mem, t)
      \semarrow
      (C', \mem, t)
      }
    \end{prooftree}
  \]

  \[
    \begin{prooftree}
      \infer[left label=LetEL]0{
      (\mathtt{let}\:x\:e_1\:e_2, C, \mem, t)
      \semarrow
      (e_{1}, C, \mem, t)
      }
    \end{prooftree}
  \]

  \[
    \begin{prooftree}
      \hypo{
        \begin{matrix}
          (e_{1}, C, \mem, t)
          \semarrow
          (v, \mem', t')
        \end{matrix}
      }
      \infer[left label=LetER]1{
      (\mathtt{let}\:x\:e_1\:e_2, C, \mem, t)
      \semarrow
      (e_{2}, (x, t')\cons C, \mem'[t'\mapsto v], \tick(t'))
      }
    \end{prooftree}
  \]

  \[
    \begin{prooftree}
      \infer[left label=LetML]0{
      (\mathtt{let}\:M\:e_{1}\:e_{2}, C, \mem, t)
      \semarrow
      (e_{1}, C, \mem, t)
      }
    \end{prooftree}\qquad
    \begin{prooftree}
      \hypo{
        \begin{matrix}
          (e_{1}, C, \mem, t)
          \semarrow
          (C', \mem', t')
        \end{matrix}
      }
      \infer[left label=LetMR]1{
      (\mathtt{let}\:M\:e_{1}\:e_{2}, C, \mem, t)
      \semarrow
      (e_{2}, (M, C')\cons C, \mem', t')
      }
    \end{prooftree}
  \]

  \[
    \begin{prooftree}
      \hypo{
        \begin{matrix}
          (e_{1}, C, \mem, t)
          \semarrow
          (v, \mem', t') \\
          (e_{2}, (x, t')\cons C, \mem'[t'\mapsto v], \tick(t'))
          \semarrow
          (C', \mem'', t'')
        \end{matrix}
      }
      \infer[left label=LetE]1{
      (\mathtt{let}\:x\:e_1\:e_2, C, \mem, t)
      \semarrow
      (C', \mem'', t'')
      }
    \end{prooftree}\quad
    \begin{prooftree}
      \hypo{
        \begin{matrix}
          (e_{1}, C, \mem, t)
          \semarrow
          (C', \mem', t') \\
          (e_{2}, (M, C')\cons C, \mem', t')
          \semarrow
          (C'', \mem'', t'')
        \end{matrix}
      }
      \infer[left label=LetM]1{
      (\mathtt{let}\:M\:e_{1}\:e_{2}, C, \mem, t)
      \semarrow
      (C'', \mem'', t'')
      }
    \end{prooftree}
  \]
  \caption{The concrete one-step transition relation.}
  \label{fig:concreach}
\end{figure}

\begin{figure}[h!]
  \begin{flushright}
    \fbox{$(e,\A{C},\A\mem,\A{t})\A\semarrow(\A{V},\A{\mem'},\A{t'})\text{ or }(e',\A{C'},\A{\mem'},\A{t'})$}
  \end{flushright}
  \vspace{0pt} % -0.75em}
  \[
    \begin{prooftree}
      \hypo{\A{t_x}=\addr(\A{C},x)}
      \hypo{\A{v}\in\A\mem(\A{t_x})}
      \infer[left label=ExprID]2{
      (x, \A{C}, \A\mem, \A{t})
      \A\semarrow
      (\A{v}, \A\mem, \A{t})
      }
    \end{prooftree}\qquad
    \begin{prooftree}
      \infer[left label=Fn]0{
      (\lambda x.e, \A{C}, \A\mem, \A{t})
      \A\semarrow
      (\langle\lambda x.e, \A{C}\rangle, \A\mem, \A{t})
      }
    \end{prooftree}
  \]

  \[
    \begin{prooftree}
      \infer[left label={AppL}]0{
      (e_{1}\:e_{2}, \A{C}, \A\mem, \A{t})
      \A\semarrow
      (e_{1},\A{C}, \A\mem,\A{t})
      }
    \end{prooftree}\qquad
    \begin{prooftree}
      \hypo{
        \begin{matrix}
          (e_{1}, \A{C}, \A\mem, \A{t})
          \A\semarrow
          (\langle\lambda x.e_{\lambda}, \A{C}_{\lambda}\rangle, \A\mem_{\lambda}, \A{t_\lambda})
        \end{matrix}
      }
      \infer[left label={AppR}]1{
      (e_{1}\:e_{2}, \A{C}, \A\mem, \A{t})
      \A\semarrow
      (e_{2}, \A{C}, \A\mem_{\lambda}, \A{t_\lambda})
      }
    \end{prooftree}
  \]

  \[
    \begin{prooftree}
      \hypo{
        \begin{matrix}
          (e_{1}, \A{C}, \A\mem, \A{t})
          \A\semarrow
          (\langle\lambda x.e_{\lambda}, \A{C}_{\lambda}\rangle, \A\mem_{\lambda}, \A{t_\lambda}) \\
          (e_{2}, \A{C}, \A\mem_{\lambda}, \A{t_\lambda})
          \A\semarrow
          (v, \A\mem_{a}, \A{t_a})
        \end{matrix}
      }
      \infer[left label={AppBody}]1{
      (e_{1}\:e_{2}, \A{C,} \A\mem, \A{t})
      \A\semarrow
      (e_{\lambda}, (x, \A{t_a})\cons \A{C}_{\lambda}, \A\mem_{a}[\A{t_a}\A\mapsto \A{v}], \A\tick(\A{t_a}))
      }
    \end{prooftree}
  \]

  \[
    \begin{prooftree}
      \hypo{
        \begin{matrix}
          (e_{1}, \A{C}, \A\mem, \A{t})
          \A\semarrow
          (\langle\lambda x.e_{\lambda}, \A{C}_{\lambda}\rangle, \A\mem_{\lambda}, \A{t_\lambda}) \\
          (e_{2}, \A{C}, \A\mem_{\lambda}, \A{t_\lambda})
          \A\semarrow
          (\A{v}, \A\mem_{a}, \A{t_a})                                                            \\
          (e_{\lambda}, (x, \A{t_a})\cons \A{C}_{\lambda}, \A\mem_{a}[\A{t_a}\A\mapsto \A{v}], \A\tick(\A{t_a}))
          \A\semarrow
          (\A{v'}, \A{\mem'},\A{t'})
        \end{matrix}
      }
      \infer[left label={App}]1{
      (e_{1}\:e_{2}, \A{C}, \A\mem, \A{t})
      \A\semarrow
      (\A{v'}, \A{\mem'},\A{t'})
      }
    \end{prooftree}
  \]

  \[
    \begin{prooftree}
      \infer[left label=LinkL]0{
      (\link{e_{1}}{e_{2}}, \A{C}, \A\mem, \A{t})
      \A\semarrow
      (e_{1}, \A{C}, \A\mem, \A{t})
      }
    \end{prooftree}\qquad
    \begin{prooftree}
      \hypo{
        \begin{matrix}
          (e_{1}, \A{C}, \A\mem, \A{t})
          \A\semarrow
          (\A{C'}, \A{\mem'}, \A{t'})
        \end{matrix}
      }
      \infer[left label=LinkR]1{
      (\link{e_{1}}{e_{2}}, \A{C}, \A\mem, \A{t})
      \A\semarrow
      (e_{2}, \A{C'}, \A{\mem'}, \A{t'})
      }
    \end{prooftree}
  \]

  \[
    \begin{prooftree}
      \hypo{
        \begin{matrix}
          (e_{1}, \A{C}, \A\mem, \A{t})
          \A\semarrow
          (\A{C'}, \A{\mem'}, \A{t'}) \\
          (e_{2}, \A{C'}, \A{\mem'}, \A{t'})
          \A\semarrow
          (\A{V}, \A{\mem''}, \A{t''})
        \end{matrix}
      }
      \infer[left label=Link]1{
      (\link{e_{1}}{e_{2}}, \A{C}, \A\mem, \A{t})
      \A\semarrow
      (\A{V}, \A{\mem''}, \A{t''})
      }
    \end{prooftree}\qquad
    \begin{prooftree}
      \infer[left label=Empty]0{
      (\varepsilon, \A{C}, \A\mem, \A{t})
      \A\semarrow
      (\A{C}, \A\mem, \A{t})
      }
    \end{prooftree}
  \]

  \[
    \begin{prooftree}
      \hypo{\A{C'}=\modctx(\A{C},M)}
      \infer[left label=ModID]1{
      (M, \A{C}, \A\mem, \A{t})
      \A\semarrow
      (\A{C'}, \A\mem, \A{t})
      }
    \end{prooftree}\qquad
    \begin{prooftree}
      \infer[left label=LetEL]0{
      (\mathtt{let}\:x\:e_1\:e_2, \A{C}, \A\mem, \A{t})
      \A\semarrow
      (e_{1}, \A{C}, \A\mem, \A{t})
      }
    \end{prooftree}
  \]

  \[
    \begin{prooftree}
      \hypo{
        \begin{matrix}
          (e_{1}, \A{C}, \A\mem, \A{t})
          \A\semarrow
          (\A{v}, \A{\mem'}, \A{t'})
        \end{matrix}
      }
      \infer[left label=LetER]1{
      (\mathtt{let}\:x\:e_1\:e_2, \A{C}, \A\mem, \A{t})
      \A\semarrow
      (e_{2}, (x, \A{t'})\cons \A{C}, \A{\mem'}[\A{t'}\A\mapsto \A{v}], \A\tick(\A{t'}))
      }
    \end{prooftree}
  \]

  \[
    \begin{prooftree}
      \infer[left label=LetML]0{
      (\mathtt{let}\:M\:e_{1}\:e_{2}, \A{C}, \A\mem, \A{t})
      \A\semarrow
      (e_{1}, \A{C}, \A\mem, \A{t})
      }
    \end{prooftree}\quad
    \begin{prooftree}
      \hypo{
        \begin{matrix}
          (e_{1}, \A{C}, \A\mem, \A{t})
          \A\semarrow
          (\A{C'}, \A{\mem'}, \A{t'})
        \end{matrix}
      }
      \infer[left label=LetMR]1{
      (\mathtt{let}\:M\:e_{1}\:e_{2}, \A{C}, \A\mem, \A{t})
      \A\semarrow
      (e_{2}, (M, \A{C'})\cons \A{C}, \A{\mem'}, \A{t'})
      }
    \end{prooftree}
  \]

  \[
    \begin{prooftree}
      \hypo{
        \begin{matrix}
          (e_{1}, \A{C}, \A\mem, \A{t})
          \A\semarrow
          (\A{v}, \A{\mem'}, \A{t'}) \\
          (e_{2}, (x, \A{t'})\cons \A{C}, \A{\mem'}[\A{t'}\A\mapsto \A{v}], \A\tick(\A{t'}))
          \A\semarrow
          (\A{C'}, \A{\mem''}, \A{t''})
        \end{matrix}
      }
      \infer[left label=LetE]1{
      (\mathtt{let}\:x\:e_1\:e_2, \A{C}, \A\mem, \A{t})
      \A\semarrow
      (\A{C'}, \A{\mem''}, \A{t''})
      }
    \end{prooftree}
  \]

  \[
    \begin{prooftree}
      \hypo{
        \begin{matrix}
          (e_{1}, \A{C}, \A\mem, \A{t})
          \A\semarrow
          (\A{C'}, \A{\mem'}, \A{t'}) \\
          (e_{2}, (M, \A{C'})\cons \A{C}, \A{\mem'}, \A{t'})
          \A\semarrow
          (\A{C''}, \A{\mem''}, \A{t''})
        \end{matrix}
      }
      \infer[left label=LetM]1{
      (\mathtt{let}\:M\:e_{1}\:e_{2}, \A{C}, \A\mem, \A{t})
      \A\semarrow
      (\A{C''}, \A{\mem''}, \A{t''})
      }
    \end{prooftree}
  \]
  \caption{The abstract one-step transition relation.}
  \label{fig:absreach}
\end{figure}

\section{Part III: Analysis that Allow Open Code to be Closed Fractionally}
Fixing a $\tick$ for each time domain is not enough!

\subsection{Definitions}
\subsubsection{Semantic Domains}
As it always must be, the domains used in the concrete semantics and the abstract semantics must be connected in a ``coherent'' way.
Our definition of semantics is parametrized by a quadruple $(\Time,\le,\A{\Time},\A{\alpha})$, where:
\begin{enumerate}
  \item $\le$ is a total order on $\Time$.
  \item $\A{\Time}$ is a (nonempty) finite set.
  \item $\A{\alpha}\in\Time\rightarrow\A{\Time}$, and there exists a $\A{\alpha}^{-1}\in(\Time\times\A{\Time})\rightarrow\Time$ that satisfies:
        \[\forall t\in\Time:t<\A{\alpha}^{-1}(t,\_)\qquad\text{and}\qquad\forall\A{t}\in\A{\Time}:\A{t}=\A{\alpha}(\A{\alpha}^{-1}(\_,\A{t}))\]
\end{enumerate}

As will be elaborated, these conditions ensure that the concrete and the abstract semantics are soundly connected.
Now we give the semantic domains that are parametrized by the choice of $(\Time,\le,\A{\Time},\A{\alpha})$.

\begin{figure}[h!]
  \centering
  \begin{tabular}{rrcll}
    Abstract Time                & $\A{t}$     & $\in$         & $\A{\Time}$                                                                                            \\
    Environment/Context          & $\A{C}$     & $\in$         & $\A{\Ctx}$                                                                                             \\
    Value of expressions         & $\A{v}$     & $\in$         & $\A{\Value} \subseteq \Expr\times\A{\Ctx}$                                                             \\
    Value of expressions/modules & $\A{V}$     & $\in$         & $\A{\Value\Ctx}\triangleq\A{\Value}\uplus\A{\Ctx}$                                                     \\
    Abstract Memory              & $\A{\mem}$  & $\in$         & $\A{\Mem} \triangleq \fin{\A{\Time}}{\wp(\A{\Value})}$                                                 \\
    Abstract State               & $\A{s}$     & $\in$         & $\A{\State} \triangleq \A{\Ctx}\times\A{\Mem}\times\A{\Time}$                                          \\
    Abstract Result              & $\A{r}$     & $\in$         & $\A{\Result} \triangleq \A{\Value\Ctx}\times\A{\Mem}\times\A{\Time}$                                   \\
    Abstract Tick                & $\A{\tick}$ & $\in$         & $\A{\Tick}\triangleq\A{\State}\times\ExprVar\times\A{\Value}\rightarrow\A{\Time}$                      \\
    Context                      & $\A{C}$     & $\rightarrow$ & []                                                                                & empty stack        \\
                                 &             & $\vbar$       & $(x,\A{t})\cons \A{C}$                                                            & expression binding \\
                                 &             & $\vbar$       & $(M,\A{C})\cons \A{C}$                                                            & module binding     \\
    Value of expressions         & $\A{v}$     & $\rightarrow$ & $\langle \lambda x.e, \A{C} \rangle$                                              & closure
  \end{tabular}
  \caption{Definition of the semantic domains in the abstract case. By $\wp(S)$ we mean the powerset of $S$.}
\end{figure}
We first introduce sets that are used to describe the abstract version of the operational semantics.
Among these sets, the most important component for the analysis designer is the $\A{\tick}$ component.
The $\A{\tick}$ function is what determines the granularity of the analysis.
Since it takes in $(\A{s},x,\A{v})$ and returns a new $\A{t}$, the timestamps can encode a variety of context information.
Increasing the variety of information encoded in $\A{t}$ increases the accuracy of the analysis.

Naturally, all $\A{\tick}$ functions must have a concrete $\tick$ that mirrors its action.
Since the abstract time domain $\A{\Time}$ and the translation function $\A{\alpha}$ meets the basic list of requirements, for each $\A{\tick}$ there exists a $\tick$ that \emph{respects} $\A{\tick}$ and produces \emph{fresh} timestamps.
To elaborate, we must take a look at the definitions for the concrete semantic domains.

\begin{figure}[h!]
  \centering
  \begin{tabular}{rrcll}
    Time                         & $t$     & $\in$         & $\Time$                                                                               \\
    Environment/Context          & $C$     & $\in$         & $\Ctx$                                                                                \\
    Value of expressions         & $v$     & $\in$         & $\Value \subseteq \Expr\times\Ctx$                                                    \\
    Value of expressions/modules & $V$     & $\in$         & $\Value\Ctx\triangleq\Value\uplus\Ctx$                                                \\
    Memory                       & $\mem$  & $\in$         & $\Mem \triangleq \fin{\Time}{\Value}$                                                 \\
    State                        & $s$     & $\in$         & $\State \subseteq \Ctx\times\Mem\times\Time$                                          \\
    Result                       & $r$     & $\in$         & $\Result \subseteq \Value\Ctx\times\Mem\times\Time$                                   \\
    Tick                         & $\tick$ & $\in$         & $\Tick\subseteq\State\times\ExprVar\times\Value\rightarrow\Time$                      \\
    Context                      & $C$     & $\rightarrow$ & []                                                               & empty stack        \\
                                 &         & $\vbar$       & $(x,t)\cons C$                                                   & expression binding \\
                                 &         & $\vbar$       & $(M,C)\cons C$                                                   & module binding     \\
    Value of expressions         & $v$     & $\rightarrow$ & $\langle \lambda x.e, C \rangle$                                 & closure
  \end{tabular}
  \caption{Definition of the semantic domains in the concrete case.}
\end{figure}

To enforce that the $\tick$ function always produces fresh timestamps, the definition of the concrete $\State$ and $\Result$ ensure that the next timestamp that is recorded in memory is \emph{larger} than any timestamp that is already used.
Define:
\[
  \begin{array}{cc}
    C < t\triangleq
    \begin{cases}
      \mathsf{True}        & C=[]              \\
      t' < t\wedge C' < t  & C=(x,t')\cons C'  \\
      C' < t\wedge C'' < t & C=(M,C')\cons C''
    \end{cases}
     &
    \begin{array}{l}
      V < t\triangleq
      \begin{cases}
        C < t & V=\langle\_,C\rangle \\
        C < t & V=C
      \end{cases} \\
      \mem < t\triangleq\forall t'\in\mathsf{dom}(\mem):t'<t\wedge\mem(t') < t
    \end{array}
  \end{array}
\]
and define:
\[
  \State\triangleq\{(C,m,t)|C<t\wedge m<t\}\qquad
  \Result\triangleq\{(V,m,t)|V<t\wedge m<t\}
\]
Note that $\State\subseteq\Result$, thus any statement about $r\in\Result$ also holds for any $s\in\State$.

The fact that the $t$-component in any $r$ bounds the $C$ and $m$ component means that the next location to be updated is fresh.
After updating the memory, a new timestamp that is larger than the original $t$ must be given by $\tick$.
This implies that: $\forall t:t<\tick((\_,\_,t),\_,\_)$ should be required of $\tick$.
Moreover, for every concrete $\tick$ to have a corresponding abstract $\A{\tick}$ connected through $\A{\alpha}$, we want to require that:
$\exists\A{\tick}:\A{\alpha}\circ\tick=\A{\tick}\circ\A{\alpha}$, when:
\[
  \begin{array}{cc}
    \A{\alpha}(C)\triangleq
    \begin{cases}
      []                                  & C=[]              \\
      (x,\A{\alpha}(t))::\A{\alpha}(C')   & C=(x,t)\cons C'   \\
      (M,\A{\alpha}(C'))::\A{\alpha}(C'') & C=(M,C')\cons C''
    \end{cases}
     &
    \begin{array}{l}
      \A{\alpha}(V)\triangleq
      \begin{cases}
        \langle e,\A{\alpha}(C)\rangle & V=\langle e,C\rangle \\
        \A{\alpha}(C)                  & V=C
      \end{cases} \\
      \A{\alpha}(\mem)\triangleq\bigcup_{t\in\mathsf{dom}(\mem)}[\A{\alpha}(t)\mapsto\{\A{\alpha}(\mem(t))\}]
    \end{array}
  \end{array}
\]

Thus we can define:
\[
  \Tick\triangleq\{\tick|\forall t:t<\tick((\_,\_,t),\_,\_)\wedge\exists\A{\tick}:\A{\alpha}\circ\tick=\A{\tick}\circ\A{\alpha}\}
\]
when the $\A{\alpha}$ in $\A{\tick}\circ\A{\alpha}$ maps over tuples and define:
\[
  \A{\alpha}(\tick)\triangleq\A{\tick}\qquad\text{where }\A{\alpha}\circ\tick=\A{\tick}\circ\A{\alpha}
\]
since if such a $\A{\tick}$ exists, it must be unique.

\subsubsection{Operational Semantics}
Now we give the operational semantics.
The relation $\semarrow_{\tick}$ relates two machine states $\ell\in\Left\triangleq\Expr\times\State$ and $\rho\in\Right\triangleq\Left\uplus\Result$, using $\tick$ as the policy for memory allocation.
Likewise, $\asemarrow_{\A{\tick}}$ relates two machine states $\A{\ell}\in\A{\Left}\triangleq\Expr\times\AState$ and $\A{\rho}\in\A{\Right}\triangleq\A{\Left}\uplus\AResult$, using $\A{\tick}$.

There are auxiliary functions that are used to describe the relation.
The $\addr$ and $\modctx$ operators each extract the address bound to a variable $x$ and the context bound to a variable $M$ from a context $C$.
\[
  {C}(x)\triangleq
  \begin{cases}
    \bot         & C=[]                              \\
    t            & C=(x, t)\cons C'                  \\
    \addr(C',x)  & C=(x', t)\cons C' \wedge x'\neq x \\
    \addr(C'',x) & C=(M, C')\cons C''
  \end{cases}
  \quad
  {C}(M)\triangleq
  \begin{cases}
    \bot           & C=[]                               \\
    C'             & C=(M, C')\cons C''                 \\
    \modctx(C'',M) & C=(M', C')\cons C''\wedge M'\neq M \\
    \modctx(C',M)  & C=(x, t)\cons C'
  \end{cases}
\]

Also, the update of the memory for the abstract operational semantics is not a strong update, which overwrites the previous value stored at the address that is written to.
\[
  \A{\mem}[\A{t}\A{\mapsto}\A{v}](\A{t'})\triangleq
  \begin{cases}
    \A{\mem}(\A{t})\cup\{\A{v}\} & (\A{t'}=\A{t})     \\
    \A{\mem}(\A{t'})             & (\text{otherwise})
  \end{cases}
\]
We call this a ``weak update'' of the abstract memory.

The definitions for the operational semantics are given in Figure \ref{fig:concreach} and Figure \ref{fig:absreach}.

\begin{figure}[h!]
  \begin{flushright}
    \fbox{$(e,C,\mem,t)\semarrow_\tick(V,\mem',t')\text{ or }(e',C',\mem',t')$}
  \end{flushright}
  \centering
  \vspace{0pt} % -0.75em}
  \[
    \begin{prooftree}
      \hypo{t_{x}={C}(x)}
      \hypo{v=\mem(t_{x})}
      \infer[left label=ExprID]2{
      (x, C, \mem, t)
      \semarrow
      (v, \mem, t)
      }
    \end{prooftree}\qquad
    \begin{prooftree}
      \infer[left label=Fn]0{
      (\lambda x.e, C, \mem, t)
      \semarrow
      (\langle\lambda x.e, C\rangle, \mem, t)
      }
    \end{prooftree}
  \]

  \[
    \begin{prooftree}
      \infer[left label={AppL}]0{
      (e_{1}\:e_{2}, C, \mem, t)
      \semarrow
      (e_{1},C, \mem,t)
      }
    \end{prooftree}\qquad
    \begin{prooftree}
      \hypo{
        \begin{matrix}
          (e_{1}, C, \mem, t)
          \semarrow
          (\langle\lambda x.e_{\lambda}, C_{\lambda}\rangle, \mem_{\lambda}, t_{\lambda})
        \end{matrix}
      }
      \infer[left label={AppR}]1{
      (e_{1}\:e_{2}, C, \mem, t)
      \semarrow
      (e_{2}, C, \mem_{\lambda}, t_{\lambda})
      }
    \end{prooftree}
  \]

  \[
    \begin{prooftree}
      \hypo{
        \begin{matrix}
          (e_{1}, C, \mem, t)
          \semarrow
          (\langle\lambda x.e_{\lambda}, C_{\lambda}\rangle, \mem_{\lambda}, t_{\lambda}) \\
          (e_{2}, C, \mem_{\lambda}, t_{\lambda})
          \semarrow
          (v, \mem_{a}, t_{a})
        \end{matrix}
      }
      \infer[left label={AppBody}]1{
      (e_{1}\:e_{2}, C, \mem, t)
      \semarrow
      (e_{\lambda}, (x, t_{a})\cons C_{\lambda}, \mem_{a}[t_{a}\mapsto v], \tick((C,\mem_{a},t_{a}),x,v))
      }
    \end{prooftree}
  \]

  \[
    \begin{prooftree}
      \hypo{
        \begin{matrix}
          (e_{1}, C, \mem, t)
          \semarrow
          (\langle\lambda x.e_{\lambda}, C_{\lambda}\rangle, \mem_{\lambda}, t_{\lambda}) \\
          (e_{2}, C, \mem_{\lambda}, t_{\lambda})
          \semarrow
          (v, \mem_{a}, t_{a})                                                            \\
          (e_{\lambda}, (x, t_{a})\cons C_{\lambda}, \mem_{a}[t_{a}\mapsto v], \tick((C,\mem_{a},t_{a}),x,v))
          \semarrow
          (v', \mem',t')
        \end{matrix}
      }
      \infer[left label={App}]1{
      (e_{1}\:e_{2}, C, \mem, t)
      \semarrow
      (v', \mem',t')
      }
    \end{prooftree}
  \]

  \[
    \begin{prooftree}
      \infer[left label=LinkL]0{
      (\link{e_{1}}{e_{2}}, C, \mem, t)
      \semarrow
      (e_{1}, C, \mem, t)
      }
    \end{prooftree}\qquad
    \begin{prooftree}
      \hypo{
        \begin{matrix}
          (e_{1}, C, \mem, t)
          \semarrow
          (C', \mem', t')
        \end{matrix}
      }
      \infer[left label=LinkR]1{
      (\link{e_{1}}{e_{2}}, C, \mem, t)
      \semarrow
      (e_{2}, C', \mem', t')
      }
    \end{prooftree}
  \]

  \[
    \begin{prooftree}
      \hypo{
        \begin{matrix}
          (e_{1}, C, \mem, t)
          \semarrow
          (C', \mem', t') \\
          (e_{2}, C', \mem', t')
          \semarrow
          (V, \mem'', t'')
        \end{matrix}
      }
      \infer[left label=Link]1{
      (\link{e_{1}}{e_{2}}, C, \mem, t)
      \semarrow
      (V, \mem'', t'')
      }
    \end{prooftree}\qquad
    \begin{prooftree}
      \infer[left label=Empty]0{
      (\varepsilon, C, \mem, t)
      \semarrow
      (C, \mem, t)
      }
    \end{prooftree}\qquad
    \begin{prooftree}
      \hypo{C'={C}(M)}
      \infer[left label=ModID]1{
      (M, C, \mem, t)
      \semarrow
      (C', \mem, t)
      }
    \end{prooftree}
  \]

  \[
    \begin{prooftree}
      \infer[left label=LetEL]0{
      (\mathtt{let}\:x\:e_1\:e_2, C, \mem, t)
      \semarrow
      (e_{1}, C, \mem, t)
      }
    \end{prooftree}
  \]

  \[
    \begin{prooftree}
      \hypo{
        \begin{matrix}
          (e_{1}, C, \mem, t)
          \semarrow
          (v, \mem', t')
        \end{matrix}
      }
      \infer[left label=LetER]1{
      (\mathtt{let}\:x\:e_1\:e_2, C, \mem, t)
      \semarrow
      (e_{2}, (x, t')\cons C, \mem'[t'\mapsto v], \tick((C,\mem',t'),x,v))
      }
    \end{prooftree}
  \]

  \[
    \begin{prooftree}
      \infer[left label=LetML]0{
      (\mathtt{let}\:M\:e_{1}\:e_{2}, C, \mem, t)
      \semarrow
      (e_{1}, C, \mem, t)
      }
    \end{prooftree}\qquad
    \begin{prooftree}
      \hypo{
        \begin{matrix}
          (e_{1}, C, \mem, t)
          \semarrow
          (C', \mem', t')
        \end{matrix}
      }
      \infer[left label=LetMR]1{
      (\mathtt{let}\:M\:e_{1}\:e_{2}, C, \mem, t)
      \semarrow
      (e_{2}, (M, C')\cons C, \mem', t')
      }
    \end{prooftree}
  \]

  \[
    \begin{prooftree}
      \hypo{
        \begin{matrix}
          (e_{1}, C, \mem, t)
          \semarrow
          (v, \mem', t') \\
          (e_{2}, (x, t')\cons C, \mem'[t'\mapsto v], \tick((C,\mem',t'),x,v))
          \semarrow
          (C', \mem'', t'')
        \end{matrix}
      }
      \infer[left label=LetE]1{
      (\mathtt{let}\:x\:e_1\:e_2, C, \mem, t)
      \semarrow
      (C', \mem'', t'')
      }
    \end{prooftree}\quad
    \begin{prooftree}
      \hypo{
        \begin{matrix}
          (e_{1}, C, \mem, t)
          \semarrow
          (C', \mem', t') \\
          (e_{2}, (M, C')\cons C, \mem', t')
          \semarrow
          (C'', \mem'', t'')
        \end{matrix}
      }
      \infer[left label=LetM]1{
      (\mathtt{let}\:M\:e_{1}\:e_{2}, C, \mem, t)
      \semarrow
      (C'', \mem'', t'')
      }
    \end{prooftree}
  \]
  \caption{The concrete one-step transition relation. The subscript $\tick$ is omitted for brevity.}
  \label{fig:concreach}
\end{figure}

\begin{figure}[h!]
  \begin{flushright}
    \fbox{$(e,\A{C},\A\mem,\A{t})\A\semarrow_{\A\tick}(\A{V},\A{\mem'},\A{t'})\text{ or }(e',\A{C'},\A{\mem'},\A{t'})$}
  \end{flushright}
  \vspace{0pt} % -0.75em}
  \[
    \begin{prooftree}
      \hypo{\A{t_x}=\addr(\A{C},x)}
      \hypo{\A{v}\in\A\mem(\A{t_x})}
      \infer[left label=ExprID]2{
      (x, \A{C}, \A\mem, \A{t})
      \A\semarrow
      (\A{v}, \A\mem, \A{t})
      }
    \end{prooftree}\qquad
    \begin{prooftree}
      \infer[left label=Fn]0{
      (\lambda x.e, \A{C}, \A\mem, \A{t})
      \A\semarrow
      (\langle\lambda x.e, \A{C}\rangle, \A\mem, \A{t})
      }
    \end{prooftree}
  \]

  \[
    \begin{prooftree}
      \infer[left label={AppL}]0{
      (e_{1}\:e_{2}, \A{C}, \A\mem, \A{t})
      \A\semarrow
      (e_{1},\A{C}, \A\mem,\A{t})
      }
    \end{prooftree}\qquad
    \begin{prooftree}
      \hypo{
        \begin{matrix}
          (e_{1}, \A{C}, \A\mem, \A{t})
          \A\semarrow
          (\langle\lambda x.e_{\lambda}, \A{C}_{\lambda}\rangle, \A\mem_{\lambda}, \A{t_\lambda})
        \end{matrix}
      }
      \infer[left label={AppR}]1{
      (e_{1}\:e_{2}, \A{C}, \A\mem, \A{t})
      \A\semarrow
      (e_{2}, \A{C}, \A\mem_{\lambda}, \A{t_\lambda})
      }
    \end{prooftree}
  \]

  \[
    \begin{prooftree}
      \hypo{
        \begin{matrix}
          (e_{1}, \A{C}, \A\mem, \A{t})
          \A\semarrow
          (\langle\lambda x.e_{\lambda}, \A{C}_{\lambda}\rangle, \A\mem_{\lambda}, \A{t_\lambda}) \\
          (e_{2}, \A{C}, \A\mem_{\lambda}, \A{t_\lambda})
          \A\semarrow
          (v, \A\mem_{a}, \A{t_a})
        \end{matrix}
      }
      \infer[left label={AppBody}]1{
      (e_{1}\:e_{2}, \A{C,} \A\mem, \A{t})
      \A\semarrow
      (e_{\lambda}, (x, \A{t_a})\cons \A{C}_{\lambda}, \A\mem_{a}[\A{t_a}\A\mapsto \A{v}], \A\tick((\A{C},\A\mem_{a},\A{t_a}),x,\A{v}))
      }
    \end{prooftree}
  \]

  \[
    \begin{prooftree}
      \hypo{
        \begin{matrix}
          (e_{1}, \A{C}, \A\mem, \A{t})
          \A\semarrow
          (\langle\lambda x.e_{\lambda}, \A{C}_{\lambda}\rangle, \A\mem_{\lambda}, \A{t_\lambda}) \\
          (e_{2}, \A{C}, \A\mem_{\lambda}, \A{t_\lambda})
          \A\semarrow
          (\A{v}, \A\mem_{a}, \A{t_a})                                                            \\
          (e_{\lambda}, (x, \A{t_a})\cons \A{C}_{\lambda}, \A\mem_{a}[\A{t_a}\A\mapsto \A{v}], \A\tick((\A{C},\A\mem_{a},\A{t_a}),x,\A{v}))
          \A\semarrow
          (\A{v'}, \A{\mem'},\A{t'})
        \end{matrix}
      }
      \infer[left label={App}]1{
      (e_{1}\:e_{2}, \A{C}, \A\mem, \A{t})
      \A\semarrow
      (\A{v'}, \A{\mem'},\A{t'})
      }
    \end{prooftree}
  \]

  \[
    \begin{prooftree}
      \infer[left label=LinkL]0{
      (\link{e_{1}}{e_{2}}, \A{C}, \A\mem, \A{t})
      \A\semarrow
      (e_{1}, \A{C}, \A\mem, \A{t})
      }
    \end{prooftree}\qquad
    \begin{prooftree}
      \hypo{
        \begin{matrix}
          (e_{1}, \A{C}, \A\mem, \A{t})
          \A\semarrow
          (\A{C'}, \A{\mem'}, \A{t'})
        \end{matrix}
      }
      \infer[left label=LinkR]1{
      (\link{e_{1}}{e_{2}}, \A{C}, \A\mem, \A{t})
      \A\semarrow
      (e_{2}, \A{C'}, \A{\mem'}, \A{t'})
      }
    \end{prooftree}
  \]

  \[
    \begin{prooftree}
      \hypo{
        \begin{matrix}
          (e_{1}, \A{C}, \A\mem, \A{t})
          \A\semarrow
          (\A{C'}, \A{\mem'}, \A{t'}) \\
          (e_{2}, \A{C'}, \A{\mem'}, \A{t'})
          \A\semarrow
          (\A{V}, \A{\mem''}, \A{t''})
        \end{matrix}
      }
      \infer[left label=Link]1{
      (\link{e_{1}}{e_{2}}, \A{C}, \A\mem, \A{t})
      \A\semarrow
      (\A{V}, \A{\mem''}, \A{t''})
      }
    \end{prooftree}\qquad
    \begin{prooftree}
      \infer[left label=Empty]0{
      (\varepsilon, \A{C}, \A\mem, \A{t})
      \A\semarrow
      (\A{C}, \A\mem, \A{t})
      }
    \end{prooftree}
  \]

  \[
    \begin{prooftree}
      \hypo{\A{C'}=\modctx(\A{C},M)}
      \infer[left label=ModID]1{
      (M, \A{C}, \A\mem, \A{t})
      \A\semarrow
      (\A{C'}, \A\mem, \A{t})
      }
    \end{prooftree}\qquad
    \begin{prooftree}
      \infer[left label=LetEL]0{
      (\mathtt{let}\:x\:e_1\:e_2, \A{C}, \A\mem, \A{t})
      \A\semarrow
      (e_{1}, \A{C}, \A\mem, \A{t})
      }
    \end{prooftree}
  \]

  \[
    \begin{prooftree}
      \hypo{
        \begin{matrix}
          (e_{1}, \A{C}, \A\mem, \A{t})
          \A\semarrow
          (\A{v}, \A{\mem'}, \A{t'})
        \end{matrix}
      }
      \infer[left label=LetER]1{
      (\mathtt{let}\:x\:e_1\:e_2, \A{C}, \A\mem, \A{t})
      \A\semarrow
      (e_{2}, (x, \A{t'})\cons \A{C}, \A{\mem'}[\A{t'}\A\mapsto \A{v}], \A\tick((\A{C},\A{\mem'},\A{t'}),x,\A{v}))
      }
    \end{prooftree}
  \]

  \[
    \begin{prooftree}
      \infer[left label=LetML]0{
      (\mathtt{let}\:M\:e_{1}\:e_{2}, \A{C}, \A\mem, \A{t})
      \A\semarrow
      (e_{1}, \A{C}, \A\mem, \A{t})
      }
    \end{prooftree}\quad
    \begin{prooftree}
      \hypo{
        \begin{matrix}
          (e_{1}, \A{C}, \A\mem, \A{t})
          \A\semarrow
          (\A{C'}, \A{\mem'}, \A{t'})
        \end{matrix}
      }
      \infer[left label=LetMR]1{
      (\mathtt{let}\:M\:e_{1}\:e_{2}, \A{C}, \A\mem, \A{t})
      \A\semarrow
      (e_{2}, (M, \A{C'})\cons \A{C}, \A{\mem'}, \A{t'})
      }
    \end{prooftree}
  \]

  \[
    \begin{prooftree}
      \hypo{
        \begin{matrix}
          (e_{1}, \A{C}, \A\mem, \A{t})
          \A\semarrow
          (\A{v}, \A{\mem'}, \A{t'}) \\
          (e_{2}, (x, \A{t'})\cons \A{C}, \A{\mem'}[\A{t'}\A\mapsto \A{v}], \A\tick((\A{C},\A{\mem'},\A{t'}),x,\A{v}))
          \A\semarrow
          (\A{C'}, \A{\mem''}, \A{t''})
        \end{matrix}
      }
      \infer[left label=LetE]1{
      (\mathtt{let}\:x\:e_1\:e_2, \A{C}, \A\mem, \A{t})
      \A\semarrow
      (\A{C'}, \A{\mem''}, \A{t''})
      }
    \end{prooftree}
  \]

  \[
    \begin{prooftree}
      \hypo{
        \begin{matrix}
          (e_{1}, \A{C}, \A\mem, \A{t})
          \A\semarrow
          (\A{C'}, \A{\mem'}, \A{t'}) \\
          (e_{2}, (M, \A{C'})\cons \A{C}, \A{\mem'}, \A{t'})
          \A\semarrow
          (\A{C''}, \A{\mem''}, \A{t''})
        \end{matrix}
      }
      \infer[left label=LetM]1{
      (\mathtt{let}\:M\:e_{1}\:e_{2}, \A{C}, \A\mem, \A{t})
      \A\semarrow
      (\A{C''}, \A{\mem''}, \A{t''})
      }
    \end{prooftree}
  \]
  \caption{The abstract one-step transition relation. The subscript $\A\tick$ is omitted for brevity.}
  \label{fig:absreach}
\end{figure}

\subsubsection{Collecting Semantics}
To define a semantics that is computable, we must formulate the collecting semantics as a least fixed point of a monotonic function that maps an element of some CPO $D$ to $D$.
In our case, $D\triangleq\wp((\Left\times\Tick\times\Right )\uplus(\Right\times\Tick))$.
The semantics of an expression $e$ starting from initial states in $S\subseteq\State\times\Tick$ is the collection of $\ell\semarrow_{\tick}\rho$ and $(\rho,\tick)$ derivable from initial configurations $((e,s),\tick)$ with $(s,\tick)\in S$.
Defining the transfer function is straightforward from the definition of the transition relation.

\begin{definition}[Transfer function]
  Given $A\subseteq D$, define
  \[
    \mathsf{Step}(A)\triangleq
    \left\{\ell\semarrow_\tick\rho, (\rho,\tick)\middle|
    \begin{prooftree}[center=true]
      \hypo{A'}\infer1{\ell\semarrow_\tick\rho}
    \end{prooftree}\wedge
    A'\subseteq A\wedge(\ell,\tick)\in A
    \right\}
  \]
\end{definition}

The $\mathsf{Step}$ function is naturally monotonic, as a ``cache'' $A$ that remembers more about the intermediate proof tree will derive more results than a cache that remembers less.
Now, because of Tarski's fixpoint theorem, we can formulate the collecting semantics in fixpoint form.
\begin{definition}[Collecting semantics]
  Given $e\in\Expr$ and $S\subseteq\State\times\Tick$, define:
  \[
    \sembracket{e}S\triangleq\lfp(\lambda X.\mathsf{Step}(X)\cup\{((e,s),\tick)|(s,\tick)\in S\})
  \]
\end{definition}

The abstract semantics is defined analogously.
Let $\A{D}\triangleq\wp((\A\Left\times\A\Tick\times\A\Right )\uplus(\A\Right\times\A\Tick))$, then the transfer function and the semantics are defined so that they respect the Galois connection between $\wp(D)$ and $\wp(\A{D})$.
How the CPOs are connected will be elaborated in the proof sketches.

\begin{definition}[Abstract transfer function]
  Given $\Abs{A}\subseteq\A{D}$, define
  \[
    \Abs{\mathsf{Step}}(\Abs{A})\triangleq
    \left\{\A\ell\A\semarrow_{\A\tick}\A\rho, (\A\rho,\A\tick)\middle|
    \begin{prooftree}[center=true]
      \hypo{\Abs{A'}}\infer1{\A\ell\A\semarrow_{\A\tick}\A\rho}
    \end{prooftree}\wedge
    \Abs{A'}\subseteq \Abs{A}\wedge(\A\ell,\A\tick)\in \Abs{A}
    \right\}
  \]
\end{definition}
\begin{definition}[Abstract semantics]
  Given $e\in\Expr$ and $\Abs{S}\subseteq\A\State\times\A\Tick$, define:
  \[
    \Abs{\sembracket{e}}\Abs{S}\triangleq\lfp(\lambda \Abs{X}.\Abs{\mathsf{Step}}(\Abs{X})\cup\{((e,\A{s}),\A\tick)|(\A{s},\A\tick)\in \Abs{S}\})
  \]
\end{definition}

\clearpage
\subsection{Proof Sketches}
\subsubsection{Well-Definedness of the Operational Semantics}

The $\Left$ and $\Right$ sets that are connected by the concrete transition relation $\semarrow_{\tick}$ are restricted by the property that the timestamps of each element are bound by its time component.
Therefore, for the concrete operational semantics to be truly well-defined, we have to check whether our inductive definition preserves the property of time-boundedness.
This is a simple proof by induction on the transition relation, since $\tick$ always produces increasing timestamps.

\begin{clm}[Well-definedness of $\semarrow_{\tick}$]
  For all $e,C,\mem,t$ and $\rho\in\Expr\times\Ctx\times\Mem\times\Time\uplus\Value\Ctx\times\Mem\times\Time$, if $s\triangleq(C,\mem,t)\in\State$ and
  $(e,s)\semarrow_{\tick}\rho$ according to the inference rules in Figure \ref{fig:concreach}, we have that $\rho\in\Right$.
\end{clm}
\begin{proof}[Sketch]
  Induction on the definition of $\semarrow_\tick$.
\end{proof}

We also have to check that $\A{\alpha}(\tick)$ is well-defined.

\begin{clm}[Well-definedness of $\A{\alpha}(\tick)$]
  For all $\tick\in\State\times\ExprVar\times\Value\rightarrow\Time$, if $\A{\tick}_1\circ\A{\alpha}=\A{\alpha}\circ\tick$ and $\A{\tick}_2\circ\A{\alpha}=\A{\alpha}\circ\tick$, we have that $\A\tick_1=\A\tick_2$.
\end{clm}
\begin{proof}[Sketch]
  We have to prove that for all $\A{s}\in\A\State$, $x\in\ExprVar$, $\A{v}\in\A\Value$, $\A\tick_1(\A{s},x,\A{v})=\A\tick_2(\A{s},x,\A{v})$.
  This is true if for all $\A{s}\in\A\State$, there exists a $s\in\State$ such that $\A\alpha(s)=\A{s}$.
  Then $\A\tick_1(\A{s},x,\A{v})=\tick(\A\alpha(s),x,\A\alpha(v))=\A\tick_2(\A{s},x,\A{v})$ by assumption, thus the desired conclusion is proved.
  The statement $\forall\A{s}\in\A\State:\exists s\in\State:\A\alpha(s)=\A{s}$ can be proved by using $\A\alpha^{-1}$ to generate fresh timestamps for all abstract times in $\A{s}$, and induction on the number of timestamps in $\A{s}$.
  Note that the time-boundedness of $s\in\State$ is ensured if we use the maximum of all generated timestamps as the first argument to $\A\alpha^{-1}$ and the time component of $\A{s}$ as the second argument.
\end{proof}

We finally check that for all $\A\tick$, there exists a $\tick$ such that $\A\alpha(\tick)=\A\tick$.

\begin{clm}[Well-definedness of $\Tick$]
  For all $\A\tick\in\A\Tick$, there exists a $\tick\in\Tick$ such that $\A\alpha(\tick)=\A\tick$.
\end{clm}
\begin{proof}[Sketch]
  Let $\tick((C,\mem,t),x,v)\triangleq\A\alpha^{-1}(t,\A\tick((\A\alpha(C),\A\alpha(m),\A\alpha(t)),x,\A\alpha(v)))$. Then $\tick\in\Tick$ by the properties of $\A\alpha^{-1}$.
\end{proof}

\subsubsection{Galois Connection between $\wp(D)$ and $\wp(\A{D})$}
We understand by $\A\alpha(\ell)=(e,\A\alpha(s))$, when $\ell=(e,s)\in\Left$ and $\A\alpha(s)$ maps over all coordinates.
Likewise, we understand $\A\alpha(\rho)$ to be $\A\alpha(\ell)$ when $\rho=\ell$ and $\A\alpha(r)$ when $\rho=r$ and $\A\alpha(r)$ maps over all coordinates.

\begin{definition}[Abstraction and Concretization]
  Define $\alpha:\wp(D)\rightarrow\wp(\A{D})$ and $\gamma:\wp(\A{D})\rightarrow\wp(D)$ by:
  \[
    \alpha(A)\triangleq\{\A\alpha(\ell)\asemarrow_{\A\alpha(\tick)}\A\alpha(\rho)|\ell\semarrow_\tick\rho\in A\}\cup\{(\A\alpha(\rho),\A\alpha(\tick))|(\rho,\tick)\in A\}
  \]
  \[
    \gamma(\Abs{A})\triangleq\{\ell\semarrow_\tick\rho|\A\alpha(\ell)\asemarrow_{\A\alpha(\tick)}\A\alpha(\rho)\in\Abs{A}\}\cup\{(\rho,\tick)|(\A\alpha(\rho),\A\alpha(\tick))\in\Abs{A}\}
  \]
\end{definition}

Then it is straightforward to see that:
\begin{clm}[Galois Connection]
  $\wp(D)\galois{\alpha}{\gamma}\wp(\A{D})$. That is, $\forall A\subseteq D,\Abs{A}\subseteq\A{D}:\alpha(A)\subseteq\Abs{A}\Leftrightarrow A\subseteq\gamma(\Abs{A})$.
\end{clm}
\begin{proof}[Sketch]
  Straightforward from the definitions of $\alpha$ and $\gamma$.
\end{proof}

Also, we can show that the concrete and abstract semantic functions are soundly connected:
\begin{clm}[Sound Abstraction]
  For all $\ell$, $\tick$, and $\rho$, $\ell\semarrow_{\tick}\rho$ implies $\A\alpha(\ell)\asemarrow_{\A\alpha(\tick)}\A\alpha(\rho)$.
  That is, $\alpha\circ\Step\subseteq\Abs\Step\circ\alpha$ and thus $\alpha(\sembracket{e}S)\subseteq\Abs{\sembracket{e}}\alpha(S)$ for all $e$ and $S\subseteq\State\times\Tick$.
\end{clm}
\begin{proof}[Sketch]
  Induction on the definition of $\semarrow_{\tick}$.
\end{proof}

\subsubsection{Computability of the Abstract Semantics}
Now we can say that $\Abs{\sembracket{e}}\alpha(S)$ is a sound abstraction of $\sembracket{e}S$.
However, is it true that $\Abs{\sembracket{e}}\alpha(S)$ is finitely computable?
Note that in practice, all reachable configurations are derived from some expression $e$ evaluated from the empty context $[]$ and empty memory $\varnothing$.
We claim that in such situations, when the abstract semantics is computed from a finite set $\Abs{S}$ of initial states, the resulting computation $\Abs{\sembracket{e}}\Abs{S}$ has finite cardinality.

Since $\ATime$ is finite, all we have to prove is that all reachable \emph{signatures} are finite.
What we mean by a \emph{signature} is a context that is stripped of all timestamps.
Explicitly, we mean an element of an inductively defined set $\Sig$ given by $X\rightarrow []\:|\:x::X\:|\:(M,X)::X$.
Then we may inductively define $\lfloor C\rfloor$ and $\lfloor \A{C}\rfloor$ to be the signatures that are obtained by stripping all timestamps from $C$ and $\A{C}$.
Moreover, we may define $\lfloor m\rfloor\triangleq\{\lfloor C\rfloor|\exists t:\langle\_,C\rangle=m(t)\}$ and $\lfloor\A{m}\rfloor\triangleq\{\lfloor\A{C}\rfloor|\exists\A{t}:\langle\_,\A{C}\rangle\in\A{m}(\A{t})\}$ to be all signatures in a memory.
Finally, we may define $\lfloor\rho\rfloor$ as the union of $\lfloor C\rfloor$ and $\lfloor m\rfloor$, when $C$ is the context component of $\rho$ and $m$ is the memory component of $\rho$.
$\lfloor\A\rho\rfloor$ can be analogously defined.

If we can prove that for all $e$ and $\A{s}$, $\bigcup_{(e,\A{s})\A\semarrow^*\A\rho}\lfloor\A\rho\rfloor\subseteq X$, when $X$ is a finite set containing all reachable signatures from $(e,\A{s})$, we can show that all reachable $\A\rho$s are finite, since $\ATime$ is finite.
It turns out, since the signature of the modules that are pushed into the stack $C$ can be accurately inferred from the definition of the operational semantics, we can \emph{compute} such an $X$.
Thus we have:
\begin{clm}[Computability of the Abstract Semantics]
  If $\Abs{S}$ is finite, $\Abs{\sembracket{e}}\Abs{S}$ is finite.
\end{clm}
\begin{proof}[Sketch]
  By existence of a function that computes all reachable signatures and by induction on $\asemarrow$.

  To elaborate, let this function be called $f$.
  We prove two statements, the first being $\forall\A\rho:\lfloor\A\rho\rfloor\subseteq f(\A\rho)$ and the second being $\forall\A\ell,\A\rho:\A\ell\A\semarrow\A\rho\Rightarrow f(\A\rho)\subseteq f(\A\ell)$.
  Then for all $\A\rho$ such that $(e,\A{s})\A\semarrow^*\A\rho$, $\lfloor\A\rho\rfloor\subseteq f(\rho)\subseteq f(\ell)$, thus all signatures in $\Abs{\sembracket{e}}\Abs{S}$ can be bound by $\{f(e,\A{s})|(\A{s},\_)\in S\}$.
\end{proof}

We stress that this is a nontrivial result, since our definition of $C$ allows contexts to be pushed into the stack.
In a language where functors that take modules that are not annotated by signatures as arguments, this claim does not hold.

\section{Results on Equivalence}
In this section, we define what it means for semantics that use different timestamps to be \emph{equivalent}.
Our framework hinges heavily on the definition of equivalence, since when we link semantics that use two different timestamps, we end up with a semantics that uses a totally different $\Time$ and $\tick$.
Even in the case without linking, we need to justify why no matter our choice of $(\Time,\le,\ATime,\A\alpha)$, the analysis overapproximates a \emph{compatible} notion of execution.

In this section, we assume a pair of semantics, each parametrized with $(\Time,\le,\ATime,\A\alpha)$ and $(\Time',\le',\ATime',\A\alpha')$.

\subsection{Definitions}
We first define what it means for two states $s\in\State$ and $s'\in\State'$ to be equivalent.
Note that $s=(C,\mem,t)$ and $s'=(C',\mem',t')$ for some contexts $C,C'$, some memories $\mem,\mem'$, and some times $t,t'$.
The choice of $t$ and $t'$ is ``not special'' in the sense that as long as they bound the context and memory, $\tick$ will continue producing fresh addresses.
Thus, the notion of equivalence is defined by how the ``information extractable'' from the $C$ and $\mem$ components are ``same''.

Note that the information that is extractable are only accessed through a sequence of names $x$ and $M$.
Thus, one may imagine access ``paths'' with names on the edges and information sources($C$ and $t$) on the nodes.
Then the definition of equivalence may be given as equivalence on all access paths with same labels on edges.
Taking this a step further, one may imagine an ``access graph'' that collects all reachable $C$ and $t$ as the nodes of the graph with directed edges connecting the nodes corresponding to accesses by names or the expression part of closures in memory.
\[
  \begin{array}{rclrcl}
    \mathsf{Step}_m(G) & \triangleq & \{C\xrightarrow{x}t,t|C\in G\wedge t={C}(x)\}             & \A{\mathsf{Step}}_{\A{m}}(\A{G}) & \triangleq & \{\A{C}\xrightarrow{x}\A{t},\A{t}|\A{C}\in \A{G}\wedge \A{t}=\addr(\A{C},x)\}                   \\
                       & \cup       & \{C\xrightarrow{M}C',C'|C\in G\wedge C'={C}(M)\}        &                                  & \cup       & \{\A{C}\xrightarrow{M}\A{C}',\A{C}'|\A{C}\in \A{G}\wedge \A{C}'=\modctx(\A{C},M)\}              \\
                       & \cup       & \{t\xrightarrow{e}C,C|t\in G\wedge \langle e,C \rangle=m(t)\} &                                  & \cup       & \{\A{t}\xrightarrow{e}\A{C},\A{C}|\A{t}\in \A{G}\wedge \langle e,\A{C} \rangle\in\A{m}(\A{t})\} \\
    \underline{C,m}    & \triangleq & \lfp(\lambda X.\mathsf{Step}_m(X)\cup\{C\})                   & \underline{\A{C},\A{m}}          & \triangleq & \lfp(\lambda \A{X}.\A{\mathsf{Step}}_{\A{m}}(\A{X})\cup\{\A{C}\})
  \end{array}
\]

The graphs $\underline{C,m}$ and $\underline{\A{C},\A{m}}$ are \emph{rooted labelled directed graph}s, with the initial context as the root.
Then the definition of equivalence between states simply means that the access graphs are isomorphic.
\begin{definition}[Equivalent Concrete States]
  Let $s=(C,\mem,t)\in\State$ and $s'=(C',\mem',t')\in\State'$.
  We say $s$ is \emph{equivalent} to $s'$ and write $s\cong s'$ when there exists a $\varphi:\Ctx\uplus\Time\rightarrow\Ctx'\uplus\Time'$ and $\varphi^{-1}:\Ctx'\uplus\Time'\rightarrow\Ctx\uplus\Time$ such that:
  \begin{enumerate}
    \item $\varphi(C)=C'$ and $\varphi^{-1}(C')=C$
    \item $\forall\text{node}_1,\text{node}_2:\text{node}_1\xrightarrow{\text{lbl}}\text{node}_2\in\underline{C,\mem}\Rightarrow\varphi(\text{node}_1)\xrightarrow{\text{lbl}}\varphi(\text{node}_2)\in\underline{C',\mem'}$
    \item $\forall\text{node}_1',\text{node}_2':\text{node}_1'\xrightarrow{\text{lbl}}\text{node}_2'\in\underline{C',\mem'}\Rightarrow\varphi^{-1}(\text{node}_1')\xrightarrow{\text{lbl}}\varphi^{-1}(\text{node}_2')\in\underline{C,\mem}$
    \item $\forall\text{node},\text{node}':\text{node}\in\underline{C,\mem}\Rightarrow\varphi^{-1}(\varphi(\text{node}))=\text{node}$ and $\text{node}'\in\underline{C',\mem'}\Rightarrow\varphi(\varphi^{-1}(\text{node}'))=\text{node}'$
  \end{enumerate}
\end{definition}
\begin{definition}[Equivalent Abstract States]
  Let $\A{s}=(\A{C},\A\mem,\A{t})\in\AState$ and $\A{s}'=(\A{C}',\A\mem',\A{t}')\in\AState'$.
  We say $\A{s}$ is \emph{equivalent} to $\A{s}'$ and write $\A{s}\A\cong\A{s}'$ when there exists a $\A\varphi:\A\Ctx\uplus\ATime\rightarrow\A\Ctx'\uplus\ATime'$ and $\A\varphi^{-1}:\A\Ctx'\uplus\ATime'\rightarrow\A\Ctx\uplus\ATime$ such that:
  \begin{enumerate}
    \item $\A\varphi(\A{C})=\A{C}'$ and $\A\varphi^{-1}(\A{C}')=\A{C}$
    \item $\forall\text{node}_1,\text{node}_2:\text{node}_1\xrightarrow{\text{lbl}}\text{node}_2\in\underline{\A{C},\A\mem}\Rightarrow\A\varphi(\text{node}_1)\xrightarrow{\text{lbl}}\A\varphi(\text{node}_2)\in\underline{\A{C}',\A\mem'}$
    \item $\forall\text{node}_1'.\text{node}_2':\text{node}_1'\xrightarrow{\text{lbl}}\text{node}_2'\in\underline{\A{C}',\A\mem'}\Rightarrow\A\varphi^{-1}(\text{node}_1')\xrightarrow{\text{lbl}}\A\varphi^{-1}(\text{node}_2')\in\underline{\A{C},\A\mem}$
    \item $\forall\text{node},\text{node}':\text{node}\in\underline{\A{C},\A\mem}\Rightarrow\A\varphi^{-1}(\A\varphi(\text{node}))=\text{node}$ and $\text{node}'\in\underline{\A{C}',\A\mem'}\Rightarrow\A\varphi(\A\varphi^{-1}(\text{node}'))=\text{node}'$
  \end{enumerate}
\end{definition}

We also define equivalence between results $r=(V,\mem.t)\in\Result$ and $r'=(V',\mem',t')\in\Result'$ by the conjunction of the equality between the expression part of $V$ and the isomorphism between access graphs.
Also, equivalence between $\ell=(e,s)\in\Left$ and $\ell'=(e',s')\in\Left'$ can be similarly defined as $e=e'\wedge s\cong s'$.
Thus equivalence between $\rho\in\Right$ and $\rho'\in\Right'$ is defined, as either $\rho\in\Left$ or $\rho\in\Result$.

Then the equivalence between elements of $\wp(D)$ and $\wp(D')$, the CPOs, can be defined as:
\begin{definition}[Equivalence between Elements of $\wp(D)$ and $\wp(D')$]
  Let $A\subseteq D$ and $A'\subseteq D'$. We say that $A$ and $A'$ are equivalent and write $A\cong A'$ iff:
  \begin{enumerate}
    \item $\forall\ell,\tick,\rho:\ell\semarrow_{\tick}\rho\in A\Rightarrow\exists\ell',\tick',\rho':\ell'\semarrow_{\tick'}\rho'\in A'\wedge\ell\cong\ell'\wedge\rho\cong\rho'$
    \item $\forall\ell',\tick',\rho':\ell'\semarrow_{\tick'}\rho'\in A'\Rightarrow\exists\ell,\tick,\rho:\ell\semarrow_{\tick}\rho\in A\wedge\ell\cong\ell'\wedge\rho\cong\rho'$
    \item $\forall\rho,\tick:(\rho,{\tick})\in A\Rightarrow\exists\rho',\tick':(\rho',{\tick'})\in A'\wedge\rho\cong\rho'$
    \item $\forall\rho',\tick':(\rho',{\tick'})\in A'\Rightarrow\exists\rho,\tick:(\rho,{\tick})\in A\wedge\rho\cong\rho'$
  \end{enumerate}
\end{definition}

Likewise, we can define equivalence between elements of $\wp(\A{D})$ and $\wp(\A{D}')$.
\begin{definition}[Equivalence between Elements of $\wp(\A{D})$ and $\wp(\A{D}')$]
  Let $\Abs{A}\subseteq \A{D}$ and $\Abs{A'}\subseteq \A{D}'$. We say that $\Abs{A}$ and $\Abs{A'}$ are equivalent and write $\Abs{A}\Abs\cong \Abs{A'}$ iff:
  \begin{enumerate}
    \item $\forall\A\ell,\A\tick,\A\rho:\A\ell\A\semarrow_{\A\tick}\A\rho\in \Abs{A}\Rightarrow\exists\A\ell',\A\tick',\A\rho':\A\ell'\semarrow_{\A\tick'}\A\rho'\in \Abs{A'}\wedge\A\ell\A\cong\A\ell'\wedge\A\rho\A\cong\A\rho'$
    \item $\forall\A\ell',\A\tick',\A\rho':\A\ell'\A\semarrow_{\A\tick'}\A\rho'\in \Abs{A'}\Rightarrow\exists\A\ell,\A\tick,\A\rho:\A\ell\semarrow_{\A\tick}\A\rho\in \Abs{A}\wedge\A\ell\A\cong\A\ell'\wedge\A\rho\A\cong\A\rho'$
    \item $\forall\A\rho,\A\tick:(\A\rho,\A{\tick})\in \Abs{A}\Rightarrow\exists\A\rho',\A\tick':(\A\rho',\A{\tick}')\in A'\wedge\A\rho\A\cong\A\rho'$
    \item $\forall\A\rho',\A\tick':(\A\rho',\A{\tick}')\in \Abs{A'}\Rightarrow\exists\A\rho,\A\tick:(\A\rho,\A{\tick})\in A\wedge\A\rho\A\cong\A\rho'$
  \end{enumerate}
\end{definition}
\subsection{Proof Sketches}
\subsubsection{Evaluation Preserves Equivalence}
To prove if we actually did define equivalence sensibly, we must show that the operational semantics preserves equivalence.
That is, we need to prove that starting from equivalent configurations, we end up in equivalent configurations.
\begin{clm}[Evaluation Preserves Equivalence]
  For all $\ell\in\Left$, $\tick\in\Tick$, $\rho\in\Right$, $\ell'\in\Left'$, $\tick'\in\Tick'$,
  \[\ell\semarrow_{\tick}\rho\wedge\ell\cong\ell'\Rightarrow\exists\rho':\ell'\semarrow_{\tick'}\rho'\wedge\rho\cong\rho'\]
  Thus, if $S\subseteq\State\times\Tick$ and $S'\subseteq\State'\times\Tick'$ are equivalent, $\sembracket{e}S\cong\sembracket{e}S'$.
\end{clm}
\begin{proof}[Sketch]
  To perform induction on $\semarrow_\tick$, we need to strengthen the claim.
  For convenience, we write $(\rho,\varphi)\cong(\rho',\varphi^{-1})$ to emphasize that the graph isomorphism is given by $\varphi$.

  Then we can strengthen the claim to a claim about graph isomorphisms.
  \[
    \begin{array}{l}
      \forall\ell,\tick,\rho,\ell',\tick',\varphi,\varphi^{-1}:\ell\semarrow_{\tick}\rho\wedge(\ell,\varphi)\cong(\ell',\varphi^{-1})\Rightarrow \\
      \exists\rho',\phi,\phi^{-1}:\ell'\semarrow_{\tick'}\rho'\wedge(\rho,\phi)\cong(\rho',\phi^{-1})\wedge\varphi(n)|_{n<t}=\phi(n)|_{n<t}      \\
      \text{when }t\text{ is the time component of }\ell
    \end{array}
  \]
  The important part here is that $\phi$ is an \emph{extension} of $\varphi$ in the sense that it agrees with $\varphi$ with nodes in $\underline{\ell}$.
  Thus the induction hypothesis will push through.
\end{proof}

\subsubsection{Concretization Preserves Equivalence}
For the definition of equivalence to be compatible with analysis, we want to show that if the abstract initial states are equivalent, so are the concretization of those states.
If this is true, we can obtain an overapproximation of an equivalent semantics by $\sembracket{e}\gamma(\Abs{S})\cong\sembracket{e}\gamma'(\Abs{S'})\subseteq\gamma'(\Abs{\sembracket{e}}\Abs{S'})$ from $\Abs{S}\Abs\cong\Abs{S'}$.
The first $\cong$ is from the fact that evaluation preserves equivalence and concretization preserves equivalence.
The second $\subseteq$ is from the fact that $\Abs\Step$ is a sound approximation of $\Step$.
The claim we want to prove is: If $\Abs{S}\Abs\cong\Abs{S'}$, then $\gamma(\Abs{S})\cong\gamma'(\Abs{S'})$.

\begin{clm}[Concretization Preserves Equivalence]
  For all $\Abs{S}\subseteq\AState\times\A\Tick$ and $\Abs{S'}\subseteq\AState'\times\A\Tick'$, $\Abs{S}\Abs\cong\Abs{S'}$ implies $\gamma(\Abs{S})\cong\gamma'(\Abs{S'})$.
\end{clm}
\begin{proof}[Sketch]
  We want to prove:
  \[\forall s\in\State,\A{s}'\in\AState':\A\alpha(s)\A\cong\A{s}'\Rightarrow\exists s'\in\AState':s\cong s'\wedge\A\alpha'(s')=\A{s}'\]
  If this is true, $\forall s\in\gamma(\Abs{s}):\exists s'\in\gamma(\Abs{s'}):s\cong s'$.

  Proving the same statement in the opposite side leads to $\forall s'\in\gamma(\Abs{s'}):\exists s\in\gamma(\Abs{s}):s\cong s'$, so that $\gamma(\Abs{S})\cong\gamma'(\Abs{S'})$.
  The proof of the above statement involves constructing such a $s'$ by traversal over reachable subparts $n$ of $s$, (1) translating $n$ to a reachable part $\A{n}'$ of $\A{s}'$ by $\A\varphi\circ\A\alpha$(when $\A\varphi$ is the isomorphism between $\A\alpha(s)$ and $\A{s}'$) and (2) lifting $\A{n}'$ to a reachable part $n'$ of $s'$ while tabulating the graph isomorphism $\varphi$ between $s$ and $s'$.
  Finally, (3) the unreachable parts of $s'$ are filled in with dummy values that translate to the unfilled parts of $\A{s}'$.
\end{proof}

\section{Results on Linking}
In this section, we define how linking between different time domains is performed.
First, we assume two time domains $(\Time_1,\le_1,\ATime_1,\A\alpha_1)$ and $(\Time_2,\le_2,\ATime_2,\A\alpha_2)$.
Then we can define $(\Time_+,\le_+,\ATime_+,\A\alpha_+)$ by:
\[
  \Time_+\triangleq2\mathbb{Z}\times\Time_1\uplus(2\mathbb{Z}+1)\times\Time_2\quad
  t_+\le_+t_+'\triangleq
  \text{lexicographic}\quad
  \ATime_+\triangleq\ATime_1+\ATime_2\quad
  \A\alpha_+(t_+)\triangleq
  \begin{cases}
    (0,\A\alpha_1(t_1)) & t_+=(2n,t_1)   \\
    (1,\A\alpha_2(t_2)) & t_+=(2n+1,t_2)
  \end{cases}
\]
when $X+Y$ is the separated sum of $X$ and $Y$.
A $\A\alpha_+^{-1}\in\Time_+\times\ATime_+\rightarrow\Time_+$ that satisfies the two conditions can be constructed by using some $\A\alpha_1^{-1}$, $\A\alpha_2^{-1}$, $0_1\in\Time_1$ and $0_2\in\Time_2$:
\[
  \A\alpha_+^{-1}(t_+,\A{t}_+)\triangleq
  \begin{cases}
    (2n, \A\alpha_1^{-1}(t_1,\A{t}_1))   & t_+=(2n,t_1)\wedge\A{t}_+=\A{t}_1   \\
    (2n+1, \A\alpha_2^{-1}(t_2,\A{t}_2)) & t_+=(2n+1,t_2)\wedge\A{t}_+=\A{t}_2 \\
    (2n+1,\A\alpha_2^{-1}(0_2,\A{t}_2))  & t_+=(2n,t_1)\wedge\A{t}_+=\A{t}_2   \\
    (2n+2,\A\alpha_1^{-1}(0_1,\A{t}_1))  & t_+=(2n+1,t_2)\wedge\A{t}_+=\A{t}_1
  \end{cases}
\]
Thus $(\Time_+,\le_+,\ATime_+,\A\alpha_+)$ satisfies our requirements.

The rest of this section explains how, given two initial conditions $S_1\subseteq\State_1\times\Tick_1$ and $S_2\subseteq\State_2\times\Tick_2$, the semantics starting from the \emph{injected} initial state $S_1\rhd S_2$ is equal to $S_1$ \emph{linked} with the semantics computed in advance from $S_2$.
That is, we want to prove:
\[
  \sembracket{e}(S_1\rhd S_2)=S_1\semlink\sembracket{e}S_2
\]
and
\[
  \Abs{\sembracket{e}}(\Abs{S}_1\Abs\rhd \Abs{S}_2)=\Abs{S}_2\Abs\semlink\Abs{\sembracket{e}}\Abs{S}_2
\]

Before descending into details, we want to clarify that all $C$, $\mem$, $t$ in this section are assumed to be living in the linked time domain.
Thus, when we have a $C\in\Ctx_1$, we write $C$ to also mean an element of $\Ctx_+$, with all timestamps equal to its original in $\Ctx_1$ in the second coordinate and with the first coordinate fixed to $0$.
Likewise, for a $C\in\Ctx_2$, we write $C$ to also mean an element of $\Ctx_+$, with all timestamp equal to its original in $\Ctx_2$ in the second coordinate and with the second coordinate fixed to $1$.
The same assumption applies for elements in $\Mem_1,\Mem_2,\Time_1,\Time_2$.
\subsection{Definitions}
\subsubsection{Injection and Deletion}
We want to define what it means to \emph{inject} an external $S_1\subseteq\State_1\times\Tick_1$ into an assumed $S_2\subseteq\State_2\times\Tick_2$.
Naturally, we must first define elementwise injection $\rhd$ between $(s_1,\tick_1)\in S_1$ and $(s_2,\tick_2)\in S_2$ and map this over all pairs in $S_1\times S_2$.
What properties must $(s_+,\tick_+)=(s_1,\tick_1)\rhd(s_2,\tick_2)$ satisfy?

Consider the case when we did not assume anything, that is, when $s_2=([],\varnothing,0)$.
Then first, we expect that $s_+\cong s_1$.
Second, the $\tick_+$ function under $s_+$ must preserve the transitions made by $\tick_2$ under $s_2$.
That is, if $(e,s_2)\semarrow_{\tick_2}^*(e',s_2')$, then $(s_+',\tick_+')=(s_1,\tick_1)\rhd(s_2',\tick_2)$ must satisfy
$\tick_+=\tick_+'$ and $(e,s_+)\semarrow_{\tick_+}^*(e',s_+')$.
This is because we want all transitions after injecting the exported states into the semantics calculated in advance to be valid transitions.

As the first step in defining $\rhd$, we first define the injection operator for contexts, when $\inject{C_1}{C_2}$ ``fills in the blank'' in $C_2$ with $C_1$.
The deletion operator $\delete{C_1}{C_2}$, which ``digs out'' $C_1$ from $C_2$, is also defined.
\begin{figure}[h!]
  \[
    \inject{C_{1}}{C_{2}}\triangleq
    \begin{cases}
      C_1                                           & C_{2}=[]            \\
      (x, t)\cons\inject{C_{1}}{C}                  & C_{2}=(x,t)\cons C  \\
      (M, \inject{C_{1}}{C})\cons\inject{C_{1}}{C'} & C_{2}=(M,C)\cons C'
    \end{cases}
    \delete{C_{1}}{C_{2}}\triangleq
    \begin{cases}
      []                                            & \A\alpha_+(C_{2})=\A\alpha_+(C_{1}) \\
      []                                            & C_{2}=[]                            \\
      (x,t)\cons\delete{C_{1}}{C}                   & C_{2}=(x,t)\cons C                  \\
      (M, \delete{C_{1}}{C})\cons\delete{C_{1}}{C'} & C_{2}=(M, C)\cons C'
    \end{cases}
  \]
  \caption{Definition of the injection operator $\inject{C_1}{C_2}$ and the deletion operator $\delete{C_1}{C_2}$.
    Cases are ordered by precedence.
    For example, we check for $\A\alpha_+(C_2)=\A\alpha_+(C_1)$ first when deleting $C_1$ from $C_2$.}
  \label{fig:concinject}
\end{figure}
This definition can be extended to injection and deletion between abstract contexts as well.

Naturally, we expect $\tick_+$ to increment timestamps in $\Time_+$ by using $\tick_1$ for timestamps in $2\mathbb{Z}\times\Time_1$, and by using $\tick_2$ for timestamps in $(2\mathbb{Z}+1)\times\Time_2$.
However, the context and memory will contain timestamps originating from both $\Time_1$ and $\Time_2$.
Therefore, we need to define the filter operations $C.1$ and $C.2$ which selects only timestamps from the time domain of interest.
\begin{figure}[h!]
  \[
    \begin{array}{cc}
      C.i\triangleq
      \begin{cases}
        []                  & C=[]                                \\
        (x,t')\cons C'.i    & C=(x,t)\cons C'\wedge t=(2n-1+i,t') \\
        C'.i                & C=(x,t)\cons C'\wedge t=(2n+i,t')   \\
        (M,C'.i)\cons C''.i & C=(M, C')\cons C''
      \end{cases} &
      \begin{array}{l}
        V.i\triangleq
        \begin{cases}
          C.i                  & V=C                  \\
          \langle e,C.i\rangle & V=\langle e,C\rangle
        \end{cases} \\ \\
        \mem.i\triangleq
        \displaystyle\bigcup_{(2n-1+i,t)\in\mathsf{dom}(\mem)}\{t\mapsto \mem(t).i\}
      \end{array}
    \end{array}
  \]
  \caption{Definition for the filter operations ($i=1,2$).}
  \label{fig:concfilter}
\end{figure}
This definition can be extended to filter out abstract timestamps as well.

Note that $\mem.i$ cannot be well-defined for memories that contain addresses with overlapping second coordinates.
That is, in the case when $\mem$ can be accessed with both $(0,t_1)$ and $(2,t_1)$, $\mem.1$ is not well-defined.
In this case, we leave $\mem.i$ as undefined.
This means that the mapping $\mem\mapsto\mem.i$ is a partial function that is defined for only memories with distinct second coordinates.
Luckily, in the case that $\mem$ is derived from injection, all timestamps in $2\mathbb{Z}\times\Time_1$ will be of the form $(0,\_)$, and all timestamps in $(2\mathbb{Z}+1)\times\Time_2$ will be of the form $(1,\_)$.
Thus, $\mem.i$ will always be defined for our purposes.
Of course, $\A\mem.i$ is always defined, since the only even first coordinate is $0$, and the only odd first coordinate is $1$.

Moving on, we extend $\inject{\_}{\_}$ and $\delete{\_}{\_}$:
\begin{definition}[Filling in the Blanks]
  Let $s_1=(C_1,\mem_1,t_1)\in\State_1$ and $r_2=(V_2,\mem_2,t_2)\in\Result_2$. Then we define:
  \[
    \begin{array}{cc}
      \inject{C_1}{V_2}\triangleq
      \begin{cases}
        \inject{C_1}{C_2}                  & V_2=C_2                   \\
        \langle e,\inject{C_1}{C_2}\rangle & V_2=\langle e, C_2\rangle
      \end{cases} &
      \begin{array}{l}
        \inject{C_1}{\mem_2}\triangleq
        \displaystyle\bigcup_{t\in\mathsf{dom}(\mem_2)}\{t\mapsto\inject{C_1}{\mem_2(t)}\} \\
        \inject{s_1}{r_2}\triangleq
        (\inject{C_1}{V_2},\mem_1\cup\inject{C_1}{\mem_2},t_2)
      \end{array}
    \end{array}
  \]

  \[
    \delete{C_1}{V_2}\triangleq
    \begin{cases}
      \delete{C_1}{C_2}                  & V_2=C_2                   \\
      \langle e,\delete{C_1}{C_2}\rangle & V_2=\langle e, C_2\rangle
    \end{cases}\:\:
    \delete{C_1}{\mem_2}\triangleq
    \displaystyle\bigcup_{t\in\mathsf{dom}(\mem_2)}\left\{t\mapsto\delete{C_1}{\mem_2(t)}\right\}
  \]
  Naturally, these definitions extend to their abstract counterparts.
\end{definition}

Note that $\inject{s_1}{r_2}\in\Result_+$. That is, it is time-bounded.
Now we only have to define $\tick_+$ which preserves transitions made before injection.
The definition for $\rhd$ is:
\begin{definition}[Elementwise Concrete Injection]
  Let $s_1=(C_1,\mem_1,t_1)\in\State_1$, $\tick_1\in\Tick_1$, $r_2=(V_2,\mem_2,t_2)\in\Result_2$, and $\tick_2\in\Tick_2$.
  We define $(s_1,\tick_1)\rhd(r_2,\tick_2)\triangleq(\inject{s_1}{r_2},\tick_+)\in\Result_+\times\Tick_+$, when $\tick_+$ is given by:
  \[
    \tick_+(C,\mem,t,x,v)\triangleq
    \begin{cases}
      (2n, \tick_1(C.1,\mem.1,t_1,x,v.1))                                                        & t=(2n,t_1)   \\
      (2n+1,\tick_2(\delete{C_1}{C}.2,\delete{C_1}{\mem}.2,t_2,x,\delete{C_1}{v}.2))             & t=(2n+1,t_2) \\
      \A\alpha_+^{-1}(t,\A\tick_+(\A\alpha_+(C),\A\alpha_+(\mem),\A\alpha_+(t),x,\A\alpha_+(v))) & \mem.i=\bot
    \end{cases}
  \]
  where
  \[
    \A\tick_+(\A{C},\A\mem,\A{t},x,\A{v})\triangleq
    \begin{cases}
      (0,\A\alpha_1(\tick_1)(\A{C}.1,\A\mem.1,\A{t}_1,x,\A{v}.1))                                                                                     & \A{t}=(0,\A{t}_1) \\
      (1,\A\alpha_2(\tick_2)(\delete{\A\alpha_1({C}_1)}{\A{C}}.2,\delete{\A\alpha_1({C}_1)}{\A\mem}.2,\A{t}_2,x,\delete{\A\alpha_1({C}_1)}{\A{v}}.2)) & \A{t}=(1,\A{t}_2) \\
    \end{cases}
  \]
  satisfies $\A\alpha_+\circ\tick_+=\A\tick_+\circ\A\alpha_+$.
\end{definition}

\begin{definition}[Elementwise Abstract Injection]
  Let $\A{s}_1=(\A{C}_1,\A\mem_1,\A{t}_1)\in\A\State_1$, $\A\tick_1\in\A\Tick_1$, $\A{r}_2=(\A{V}_2,\A\mem_2,\A{t}_2)\in\A\Result_2$, and $\A\tick_2\in\A\Tick_2$.
  We define $(\A{s}_1,\A\tick_1)\A\rhd(\A{r}_2,\A\tick_2)\triangleq(\inject{\A{s}_1}{\A{r}_2},\A\tick_+)\in\A\Result_+\times\A\Tick_+$, when $\A\tick_+$ is given by:
  \[
    \A\tick_+(\A{C},\A\mem,\A{t},x,\A{v})\triangleq
    \begin{cases}
      (0,\A\tick_1(\A{C}.1,\A\mem.1,\A{t}_1,x,\A{v}.1))                                                       & \A{t}=(0,\A{t}_1) \\
      (1,\A\tick_2(\delete{\A{C}_1}{\A{C}}.2,\delete{\A{C}_1}{\A\mem}.2,\A{t}_2,x,\delete{\A{C}_1}{\A{v}}.2)) & \A{t}=(1,\A{t}_2) \\
    \end{cases}
  \]
\end{definition}

Since $\tick_+$ digs out $C_1$ from the memory and context, timestamps produced by $\tick_+$ after injection will look at only the parts before injection.
Thus, it will produce the same timestamps that were produced by $\tick_2$ under $S_2$.
This is why injection into transitions computed in advance are valid as transitions beginning from injected states.

\subsubsection{Semantic Linking}
\begin{figure}[h!]
  \begin{align*}
    (s_1,\tick_1)\rhd(\rho_2,\tick_2)                  & \triangleq
    \begin{cases}
      (r_+,\tick_+)     & \rho_2=r_2\wedge(r_+,\tick_+)=(s_1,\tick_1)\rhd(r_2,\tick_2)            \\
      ((e,s_+),\tick_+) & \rho_2=\ell_2=(e,s_2)\wedge(s_+,\tick_+)=(s_1,\tick_1)\rhd(s_2,\tick_2)
    \end{cases} \\
    (s_1,\tick_1)\rhd(\ell_2\semarrow_{\tick_2}\rho_2) & \triangleq
    \ell_+\semarrow_{\tick_+}\rho_+                                                             \\
                                                       & \text{ where }
    (\ell_+,\tick_+)=(s_1,\tick_1)\rhd(\ell_2,\tick_2)\wedge
    (\rho_+,\tick_+)=(s_1,\tick_1)\rhd(\rho_2,\tick_2)
  \end{align*}
  \caption{Extension of $\rhd$ to define injection into an element of $D_2$.}
  \label{fig:extinject}
\end{figure}
\begin{definition}[Concrete Injection]
  Let $S_1\subseteq\State_1\times\Tick_1$ and $A_2\subseteq D_2$. Then we define:
  \[
    S_1\rhd A_2\triangleq\{p\rhd q|p\in S_1\wedge q\in A_2\}
  \]
\end{definition}
\begin{definition}[Abstract Injection]
  Let $\Abs{S}_1\subseteq\AState_1\times\A\Tick_1$ and $\Abs{A}_2\subseteq\A{D}_2$. Then we define:
  \[
    \Abs{S}_1\Abs\rhd\Abs{A}_2\triangleq\{\A{p}\A\rhd\A{q}|\A{p}\in\Abs{S}_1\wedge\A{q}\in\Abs{A}_2\}
  \]
\end{definition}
Now we need to define the semantic linking operator $\semlink$.
More specifically, we must define $S_1\semlink A_2$, when $S_1\subseteq\State_1\times\Tick_1$ and $A_2\subseteq D_2$.
Remember that $A_2$ is the separately computed semantics, and $S_1$ is what was missing.
Thus, we must first inject all $(s_1,\tick_1)\in S_1$ into $(\rho_2,\tick_2),\ell_2\semarrow_{\tick_2}\rho_2\in A_2$.
Next, since we have gained new information about the external environment, we must collect more that can be gleaned from $S_1$.
Thus, the definition of semantic linking is as follows:
\begin{definition}[Concrete Linking]
  Let $S_1\subseteq\State_1\times\Tick_1$ and $A_2\subseteq D_2$.
  Then:
  \[
    S_1\semlink A_2\triangleq
    \lfp(\lambda X.\mathsf{Step}(X)\cup(S_1\rhd A_2))
  \]
\end{definition}

\begin{definition}[Abstract Linking]
  Let $\Abs{S}_1\subseteq\A\State_1\times\A\Tick_1$ and $\Abs{A}_2\subseteq \A{D}_2$.
  Then:
  \[
    \Abs{S}_1\Abs\semlink\Abs{A}_2\triangleq
    \lfp(\lambda \Abs{X}.\Abs\Step(\Abs{X})\cup(\Abs{S}_1\Abs\rhd \Abs{A}_2))
  \]
\end{definition}

\subsection{Proof Sketches}
\begin{lem}[Concrete Advance, Operational]
  For all $(s_1,\tick_1)\in\State_1\times\Tick_1$, $\ell_2\in\Left_2$, $\tick_2\in\Tick_2$, $\rho_2\in\Right_2$,
  If we let $(\ell_+,\tick_+)=(s_1,\tick_1)\rhd(\ell_2,\tick_2)$ and $(\rho_+,\tick_+)=(s_1,\tick_1)\rhd(\rho_2,\tick_2)$, we have:
  \[
    \ell_2\semarrow_{\tick_2}\rho_2\Rightarrow\ell_+\semarrow_{\tick_+}\rho_+
  \]
  Thus $S_1\rhd\sembracket{e}S_2\subseteq\sembracket{e}(S_1\rhd S_2)$.
\end{lem}
\begin{proof}[Sketch]
  By induction on $\semarrow_{\tick_2}$.
  We use the fact that $\inject{C}{\mem[t\mapsto v]}=\inject{C}{\mem}[t\mapsto\inject{C}{v}]$.
\end{proof}
\begin{lem}[Concrete Advance]
  Let $S_1\subseteq\State_1\times\Tick_1$ and $S_2\subseteq\State_2\times\Tick_2$. Then:
  \[
    \sembracket{e}(S_1\rhd S_2)=S_1\semlink\sembracket{e}S_2
  \]
\end{lem}
\begin{proof}[Sketch]
  Straightforward from the previous lemma.
\end{proof}
\begin{lem}[Abstract Advance, Operational]
  For all $(\A{s}_1,\A\tick_1)\in\AState_1\times\A\Tick_1$, $\A\ell_2\in\A\Left_2$, $\A\tick_2\in\A\Tick_2$, $\A\rho_2\in\A\Right_2$,
  If we let $(\A\ell_+,\A\tick_+)=(\A{s}_1,\A\tick_1)\A\rhd(\A\ell_2,\A\tick_2)$ and $(\A\rho_+,\A\tick_+)=(\A{s}_1,\A\tick_1)\A\rhd(\A\rho_2,\A\tick_2)$, we have:
  \[
    \A\ell_2\A\semarrow_{\A\tick_2}\A\rho_2\Rightarrow\A\ell_+\A\semarrow_{\A\tick_+}\A\rho_+
  \]
  Thus $\Abs{S}_1\Abs\rhd\Abs{\sembracket{e}}\Abs{S}_2\subseteq\Abs{\sembracket{e}}(\Abs{S}_1\Abs\rhd \Abs{S}_2)$.
\end{lem}
\begin{proof}[Sketch]
  By induction on $\A\semarrow_{\A\tick_2}$.
  We use the fact that $\inject{\A{C}}{\A\mem[\A{t}\A\mapsto\A{v}]}=\inject{\A{C}}{\A\mem}[\A{t}\A\mapsto\inject{\A{C}}{\A{v}}]$.
\end{proof}
\begin{lem}[Abstract Advance]
  Let $\Abs{S}_1\subseteq\AState_1\times\A\Tick_1$ and $\Abs{S}_2\subseteq\AState_2\times\A\Tick_2$. Then:
  \[
    \Abs{\sembracket{e}}(\Abs{S}_1\Abs\rhd\Abs{S}_2)=\Abs{S}_1\Abs\semlink\Abs{\sembracket{e}}\Abs{S}_2
  \]
\end{lem}
\begin{proof}[Sketch]
  Straightforward from the previous lemma.
\end{proof}
\end{document}
%%% Local Variables: 
%%% coding: utf-8
%%% mode: latex
%%% TeX-engine: xetex
%%% End:
