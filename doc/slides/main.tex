%! TEX program = xelatex
% \PassOptionsToPackage{draft}{graphicx}
\documentclass{beamer}

\usepackage{kotex}
\usepackage{xcolor}
\usepackage{tcolorbox}

\usepackage{setspace} % setstretch

\usepackage{tabularray}
\UseTblrLibrary{booktabs}
\UseTblrLibrary{counter}

%%% Math settings
\usepackage{amssymb,amsmath,mathtools} % Before unicode-math
\usepackage[math-style=TeX,bold-style=TeX]{unicode-math}

%%% Font settings
\setmainfont{Libertinus Serif}
\setmathfont{Libertinus Math} % Before set*hangulfont
\setmathfont{TeX Gyre Pagella Math}[range={\lbrace,\rbrace},Scale=1.1]
\setmainhangulfont{Noto Serif CJK KR}
\setsanshangulfont[BoldFont={* Bold}]{KoPubWorld Dotum.ttf}

%%% PL constructs
\usepackage{ebproof}
\ebproofset{left label template=\textsc{[\inserttext]}}
\ebproofset{center=false}

% For simplebnf
\newfontfamily{\fallbackfont}{EB Garamond}
\DeclareTextFontCommand{\textfallback}{\fallbackfont}
\usepackage{newunicodechar}
\newunicodechar{⩴}{\textfallback{⩴}}

\usepackage{simplebnf}
\RenewDocumentCommand\SimpleBNFDefEq{}{\ensuremath{⩴}}

%%% Custom commands
\newcommand*{\vbar}{|}
\newcommand*{\finto}{\xrightarrow{\text{\textrm{fin}}}}
\newcommand*{\istype}{\mathrel{⩴}}
\newcommand*{\ortype}{\mathrel{|}}
\newcommand*{\cons}{::}

\def\ovbarw{1.2mu}
\def\ovbarh{1}
\newcommand*{\ovbar}[1]{\mkern \ovbarw\overline{\mkern-\ovbarw{\smash{#1}\scalebox{1}[\ovbarh]{\vphantom{i}}}\mkern-\ovbarw}\mkern \ovbarw}
\newcommand*{\A}[1]{{#1}^{\#}}
\newcommand*{\Expr}{\text{Expr}}
\newcommand*{\ExprVar}{\text{Var}}
\newcommand*{\Module}{\text{Module}}
\newcommand*{\ModVar}{\text{ModVar}}
\newcommand*{\Time}{\mathbb{T}}
\newcommand*{\ATime}{\A{\Time}}
\newcommand*{\Ctx}{\text{Ctx}}
\newcommand*{\Value}{\text{Val}}
\newcommand*{\Mem}{\text{Mem}}
\newcommand*{\Left}{\text{L}}
\newcommand*{\Right}{\text{R}}
\newcommand*{\mem}{m}
\newcommand*{\AMem}{\A{\text{Mem}}}
\newcommand*{\State}{\text{State}}
\newcommand*{\AState}{\A{\text{State}}}
\newcommand*{\Result}{\text{Result}}
\newcommand*{\AResult}{\A{\text{Result}}}
\newcommand*{\Tick}{\text{Tick}}
\newcommand*{\semarrow}{\rightsquigarrow}
\newcommand*{\semlink}{\mathbin{\rotatebox[origin=c]{180}{$\propto$}}}
\newcommand*{\link}[2]{{#1}\semlink{#2}}
\newcommand*{\mt}{\mathsf{empty}}

\newcommand*{\doubleplus}{\ensuremath{\mathbin{+\mkern-3mu+}}}
\newcommand*{\project}{\text{\texttt{:>} }}
\newcommand*{\Exp}{\mathsf{Exp}}
\newcommand*{\Imp}{\mathsf{Imp}}
\newcommand*{\Fin}{\mathsf{Fin}}
\newcommand*{\Link}{\mathsf{Link}}
\newcommand*{\sembracket}[1]{\lBrack{#1}\rBrack}
\newcommand*{\fin}[2]{{#1}\xrightarrow{\text{fin}}{#2}}
\newcommand*{\addr}{\mathsf{addr}}
\newcommand*{\tick}{\mathsf{tick}}
\newcommand*{\modctx}{\mathsf{ctx}}
\newcommand*{\mapinject}[2]{{#2}[{#1}]}
\newcommand*{\inject}[2]{{#2}\langle{#1}\rangle}
\newcommand*{\deletepre}[2]{{#2}\overline{\doubleplus}{#1}}
\newcommand*{\deletemap}[2]{{#1}\overline{[{#2}]}}
\newcommand*{\delete}[2]{{#2}{\langle{#1}\rangle}^{-1}}
\newcommand*{\filter}{\mathsf{filter}}
\newcommand*{\Let}{\mathtt{let}}

%%%%%%%%%%%%%%%%%%%%%
%  Beamer Settings  %
%%%%%%%%%%%%%%%%%%%%%
\usetheme[numbering=fraction,progressbar=frametitle]{metropolis}
\useoutertheme[subsection=false]{miniframes}
\usecolortheme{rose}

\setbeamertemplate{itemize item}[square]
\setbeamertemplate{itemize subitem}[triangle]
\setbeamertemplate{itemize subsubitem}[circle]

\usepackage{listings}
\lstdefinestyle{mystyle}{
    basicstyle=\ttfamily\footnotesize
}
\lstset{style=mystyle}

\usepackage{pgfplots}
\usetikzlibrary{shapes,arrows}

\usepackage{adjustbox}
\newcommand{\cfbox}[1]{\adjustbox{cfbox=#1}}

\title{프로그램 조각별 따로분석의 이론적 틀}
\author{이준협}
\date{2023년 8월 23일}
\institute{SIGPL 2023 여름학교}

\titlegraphic{%
  \begin{tikzpicture}[overlay,remember picture]
    \node at (current page.145) [xshift=3em, yshift=-1.3em] {
      \includegraphics[width=3em]{snu-symbol.png}
    };
    \node at (current page.35) [xshift=-3em, yshift=-1.3em] {
      \includegraphics[width=2.5em]{ropas-symbol.png}
    };
  \end{tikzpicture}%
}

\begin{document}
\maketitle
\begin{frame}[c]
  \frametitle{풀고자 한 문제}
  프로그램 전체를 분석하지 않고, 일부만 미리 분석해놓고 싶다!
  \begin{itemize}
    \item 상황 1: 외부 모듈에 대한 가정 없이 분석 후 재사용
    \item 상황 2: 예전에 다른 모듈과 합쳐서 분석했던 결과를 재사용
  \end{itemize}
  \pause
  \begin{center}
    \fbox{가정이 부족한 분석?}

    \fbox{분석 결과의 재사용?}
  \end{center}
\end{frame}
\begin{frame}[c]
  \frametitle{목표}
  부족했던 것: \textbf{의미구조 정의}부터 자연스럽게 따로분석이 이끌어지는 틀
  \begin{enumerate}
    \item 프로그램의 실행의미가 실제 \textbf{실행기}의 동작과 가깝고,
    \item 분석 디자이너가 신경 쓸 것이 많이 없는,
    \item 그러나 \textbf{정밀성}을 임의로 조절할 수 있는 틀, 그리고 안전성 증명.
  \end{enumerate}
\end{frame}
\begin{frame}{목차}
  \tableofcontents
\end{frame}
\section{모듈이 있는 언어}
\begin{frame}[c]
  \frametitle{겉모습}
  \begin{figure}[h!]
    \centering
    \begin{tabular}{rrcll}
      Identifiers & $x,M$ & $\in$         & $\ExprVar$                                         \\
      Expression  & $e$   & $\rightarrow$ & $x$                & value identifier              \\
                  &       & $\vbar$       & $\lambda x.e$      & function                      \\
                  &       & $\vbar$       & $e$ $e$            & application                   \\
                  &       & $\vbar$       & $\link{e}{e}$      & \underline{linked expression} \\
                  &       & $\vbar$       & $\varepsilon$      & empty module                  \\
                  &       & $\vbar$       & $M$                & \underline{module identifier} \\
                  &       & $\vbar$       & $\Let$ $x$ $e$ $e$ & binding expression            \\
                  &       & $\vbar$       & $\Let$ $M$ $e$ $e$ & binding module                \\
    \end{tabular}
  \end{figure}
\end{frame}
\begin{frame}[c,fragile]
  \frametitle{예시}
  \begin{tcolorbox}[sidebyside, sidebyside align=top, fontupper=\scriptsize, fontlower=\scriptsize]
    \begin{lstlisting}[basicstyle=\ttfamily,escapeinside={/*}{*/}]
(* A.ml *)
let true x y = x

(* main.ml *)
open A
let id x = x
;;
id true
\end{lstlisting}
    \tcblower
    \begin{lstlisting}[basicstyle=\ttfamily,escapeinside={/*}{*/}]
(* In our language *)
(let A =
  let true = \x.\y.x in/* $\varepsilon$ */
in/* $\varepsilon$ */)/* $\semlink$ */
  (A/* $\semlink$ */
    ((let id = \x.x
    in/* $\varepsilon$ */)/* $\semlink$ */
      (id true)))
\end{lstlisting}
  \end{tcolorbox}
\end{frame}
\section{주요 정리}
\begin{frame}[c]
  \frametitle{따로분석이란?}
  최종목표: $S$에서 출발한 $\link{e_1}{e_2}$ 의 결과 : $|\sembracket{\link{e_1}{e_2}}S|$
  \pause
  \begin{itemize}
    \item 원래는, 먼저 $S$에서 출발해 $e_1$의 결과를 계산 : $|\sembracket{e_1}S|$
    \item $e_1$의 결과에서 출발해 $e_2$의 결과를 계산 : $|\sembracket{e_2}|\sembracket{e_1}S||$
          \[|\sembracket{\link{e_1}{e_2}}S|=|\sembracket{e_2}|\sembracket{e_1}S||\]
          \pause
    \item 대신, \textbf{가정된} $S_2$에서 출발해 $e_2$을 가능한 곳까지 계산 : $\sembracket{e_2}S_2$
    \item \textbf{이후}, $e_1$의 결과가 $S_2$와 나머지로 분리되나 확인 : $|\sembracket{e_1}S|\cong S_1\rhd S_2$
    \item 부족했던 부분인 $S_1$을 합쳐서 최종 결과 계산 : $S_1\semlink\sembracket{e_2}S_2$
          \begin{center}
            \fbox{$|\sembracket{\link{e_1}{e_2}}S|\cong|S_1\semlink\sembracket{e_2}S_2|$}
          \end{center}
  \end{itemize}
\end{frame}
\section{분석을 위해 신경 쓸 것}
\begin{frame}[c]
  \frametitle{$\Time$와 $\tick$}
  실행의미는 $\Time$(Time)이라는 집합과 $\tick$이라는 함수로 매개화되어있다.
  \begin{itemize}
    \item $\Time$: 실행중 프로그램 지점을 구별해줌, 메모리 주소로도 쓰임.
    \item $\tick$: 현재 환경을 받아서, 증가된(지금껏 안 쓰인) 시간을 줌.
  \end{itemize}
\end{frame}
\begin{frame}[c]
  \frametitle{분석}
  요구사항: $\alpha:\Time\rightarrow\A\Time$
  \begin{enumerate}
    \item $\A\tick\circ\alpha=\alpha\circ\tick$인 $\A\tick$ 사용.
    \item $\forall\A{t}:\alpha^{-1}(\A{t})$는 $\Time$의 순서에 대해 윗뚜껑이 없다.
  \end{enumerate}
  분석 방법: $|\A{\sembracket{\link{e_1}{e_2}}}\A{S}|$ 어림잡기
  \begin{enumerate}
    \item 가정($\A{S}_2$)하고 분석($\A{\sembracket{e_2}}\A{S}_2$)하라
    \item 가정이 성립($\A{S}_1\A\rhd\A{S}_2\A\cong|\A{\sembracket{e_1}}\A{S}|$)하면, 합쳐라($\A{S}_1\A\semlink\A{\sembracket{e_2}}\A{S}_2$)
  \end{enumerate}
\end{frame}
\begin{frame}[c]
  \frametitle{홍보}
  \begin{center}
    포스터에 언어의 실행의미 등 더 자세한 설명이 적혀있습니다!
  \end{center}
\end{frame}
\begin{frame}[c]
  \centering\LARGE
  감사합니다
\end{frame}
\end{document}
%%% Local Variables:
%%% coding: utf-8
%%% mode: latex
%%% TeX-engine: xetex
%%% End:
